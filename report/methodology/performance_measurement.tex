\section{Performance Measurement}

%
%Q: What should we be measuring?
In a parallel MD simulation, the time for a single step is strongly dependent
on both the number of particles present and
the number of parallel processes the simulation is running on.
%
This suggests a 2D space over which the performance of the implementation
might be evaluated.

%
%Q: What will we measure?
As successful MD simulations can be carried out on well defined
systems sizes, evaluating just how large a simulation can be run
is not particularly interesting.
%
As such, several fixed system sizes will be tested, namely
$N = 2^{9} = 512$,
$N = 2^{12} = 4096$ and
$N = 2^{15} = 32768$
particles.
%
These are chosen mainly so effects can be studied where the number of cores
used is small compared to the system, large compared to the system
and comparable to the system.
%
In particular, system sizes were chosen where communication times
tended to surpass calculation times within $P = 32768$ cores
being used.
%
The minimum system size was chosen to reflect a number of cores
exceeding the number of cores on a single \hector{} node and yet well
below the total number of cores on the machine.
%
The maximum system size was chosen both for time to completion for
a single repetition, and for it's large size compared to the number of
cores on a \hector{} node.

%
%Q: How will we measure these?
A very simple model of MD is used to distil the fundamental calculation
from the implementation.
%
Specifically, the simulation will perform a loop consisting of two
individual operations over the particles representing the first
and second halves of the \velocityverlet{} integration scheme and one
pair operation representing the force evaluation section of the scheme.
%
The loop will be run several times to extract some statistical significance
from the timings.
%
In particular, loops will be repeated until at least 5 iterations
have been performed and the overall time is comparable to 1 second.

%
%Q: Will we measure IO?
No I/O will be performed during a benchmark.
%
While this is not representative of a real MD code, and indeed I/O
is itself an important factor in the performance of an MD simulation,
it is left outside the scope of this discussion.
%
I/O is implemented primarily for testing and debugging purposes.

%
%Q: How accurate will the benchmarked simulations be?
There is no pretence made of performing accurate simulations in
the benchmarks performed,
only that the scheme benchmarked performs calculations and
communications that may be used to perform an accurate MD simulation.
%
The entire step of allowing the system to come to
thermal equilibrium will therefore be ignored.
%
This is primarily to reduce the time required to perform a benchmark.
%
This is not expected to have a significant impact on performance metrics
as no particle domain specific distributions are dealt with here,
so the particles having improper positions or velocities is of
little consequence to the number of operations and communications
performed on each process.


\subsection{Benchmarking Plan}

Benchmarks will be performed on \hector{} Phase 3 using
32 cores per node with 32~GB of memory available to each node.
%
Each benchmark will be performed on the range of $[1,32768]$ cores
in powers of 2.
%
That is, benchmarks will be performed on
$1$, $2$, $4$, $8$, $\dots{}$, $32768$ cores.
%
Measurements will be made of the time taken for $n$ repetitions of a given
method where $n \ge{} 5$ and the time $t \sim{} 1$~second.
%
These measurements will be taken several times, at different times of
day and the lowest recorded time used here.

Measurements for the \individualoperation{} and \pairoperation{} method
for a particular parallelisation scheme will be made.
%
These will be measured in the cases where both MPI and calculations
are performed, measuring the total execution time of the method;
the case where MPI communications are skipped and calculations are
performed, measuring the calculation time of the method; and
the case where calculations are skipped and MPI communications are
performed, measuring the communication time of the method.

Four schemes will be tested.
%
The \replicateddata{} and \systolicloop{} schemes as outlined above
will be implemented and tested.
%
Two methods that build upon these referred to here as
the \sharedandreplicateddata{} scheme and the \replicatedsystolicloop{} scheme
will also be implemented and tested.

Each implementation is left relatively unoptimised to ensure
the clarity of the implementation although
some thought has been put into the code to avoid inappropriately slow code.
%
These are compiled using crayftn and mpich2, using the ftn compiler with
no flags specified on the command line, beyond that needed to compile
the code.
%
MPI routines are implemented using blocking calls to produce the worst
case communication time.

\begin{table}
    \begin{tabular}{|l|l|}
        \hline
        Machine & \hector{} Phase 3 \\
        Cores Per Node & 32 \\
        Memory Per Node & 32~GB \\
        \hline
        Schemes Tested & \replicateddata{}, \systolicloop{}, \\
                       & \sharedandreplicateddata{},
                         \replicatedsystolicloop{} \\
        Methods Tested & \individualoperation{}, \pairoperation{} \\
        Regimes Tested & Total Time, MPI Only, Calculation Only \\
        Core Range Tested & 1-32768 \\
        \hline
    \end{tabular}
    \caption{
        Summary of benchmarks to be performed and the hardware and
        configuration they will be performed on.
    }
    \label{table:benchmark_configuration}
\end{table}
