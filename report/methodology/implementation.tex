\section{The Abstract Distribution}
\label{sec:methodology:subsec:implementation}

%
%Q: How will the code be implemented?
The code for this dissertation is implemented in
object-oriented Fortran 2003.
%
%Q: How will we handle changing the underlying parallel pattern?
To allow for arbitrary \twobody{} potentials and integration schemes
to be used with arbitrary parallel schemes,
the routines calculating how an individual particle is stepped forward
in time and the routine determining when and where that stepping is applied
are separated.
%
Similarly, the routines calculating the force between two particles
are separated from the routine for combining all of these
\twobody{} forces to find the total instantaneous force on a particle.

\begin{figure}
    \begin{center}
    \begin{tikzpicture}[node distance=3.5cm]
        \umlclass{particle}
            {
                double precision :: pos(Ndim) \\
                double precision :: vel(Ndim) \\
                double precision :: force(Ndim) \\
                double precision :: mass
            }{}
        \umlclass[y=-5, type=abstract]{abstract\_distribution}
            {
                type(particle), allocatable :: particles(:)
            }{
                individual\_operation \\
                pair\_operation
            }
        \umlaggreg[mult2=1..N, mult1=1]
            {abstract\_distribution}{particle}
        \umlemptyclass[y=-8.9, x=-3]{replicated\_distribution}
        \umlemptyclass[y=-12.5, x=-3]{shared\_and\_replicated\_distribution}
        \umlemptyclass[y=-8.9, x=4.05]{systolic\_distribution}
        \umlemptyclass[y=-11, x=4.1]{replicated\_systolic\_distribution}
        \umlreal[geometry=-|]{replicated\_distribution}{abstract\_distribution}
        \umlreal[geometry=-|]{systolic\_distribution}{abstract\_distribution}
        \umlinherit[geometry=-|]
            {shared\_and\_replicated\_distribution}{replicated\_distribution}
        \umlreal[geometry=-|]
            {replicated\_systolic\_distribution}{abstract\_distribution}
    \end{tikzpicture}
    \end{center}
    \caption{
        Overview of the \abstractdistribution{} class and inheritence tree.
    }
\end{figure}

To address this, an \abstractdistribution{} class is implemented
which holds a list of particles and
provides a set of methods for
applying transformations over the particles.
%
The user can define functions to be passed into these methods
as parameters, and the method should perform a predefined operation
over the list of particles using these user defined functions.
%
%Q: What does an implementation look like?
An implementation of a parallel scheme inherits from the
\abstractdistribution{} class and is free to distribute
the list of particles it holds across processes as it wishes,
and should implement the required interfaces as appropriate for that scheme.
%
This allows the handling of arbitrary user defined operations while
allowing the underlying parallel scheme to be changed as necessary.

%
%Q: What interfaces will our implementations provide?
The MD loop illustration in
\FIG{fig:md_loop_flow_chart}
suggests a set of interfaces to be implemented.
%
Namely, one for capable of performing force updates and another
capable of performing an integration step.
%
Integration step can be implemented with a method that can update particles
in place.
%
The force update can be implemented with a method that can
update each particle based on some
comparison with every other particle in the system.
%
These two methods will be outlined here with an example serial
implementation provided for clarity.
%
They will be referred to as the \individualoperation{} method and
the \pairoperation{} method.


\subsection{The \individualoperation{} Method}
\label{sec:the_individual_operation_method}

This method will accept a function defining a mapping from
particles and integers to particles.
\begin{equation}
    f: p\times{}i \rightarrow{} p
\end{equation}
%
The intention is that a particle and the number of that particle in
the particle list is passed to the function, and the function
returns an updated particle based on the particle's current parameters
and its position in the list.
%
This replaces each particle in the list of particles with
a newly updated particle.

An example serial implementation is in
\LST{lst:serial_individual_operation_implementation}.
An example of using this method to initialise the list of
particles is given in
\LST{lst:individual_operation_example_usage}.

\begin{lstlisting}[
    caption=Serial \individualoperation{} implementation.,
    label=lst:serial_individual_operation_implementation,
    gobble=3
]
    subroutine individual_operation(this, f)
        class(serial_distribution), intent(inout) :: this
        procedure(one_particle_function), intent(in) :: f

        integer :: i


        do i=1, this%num_particles
            this%particles(i) = f(this%particles(i), i)
        end do
    end subroutine individual_operation
\end{lstlisting}

\begin{lstlisting}[
    caption=Example usage of the \individualoperation{} to initialise
        a list of particles.,
    label=lst:individual_operation_example_usage,
    gobble=3
]
    program ExampleIndividualOperation
        use serial_distribution_type
        use particle_type
        implicit none


        type(serial_distribution) :: dist

    
        ! Initialise a serial_distribution
        dist = new_serial_distribution(...)

        ! Apply init_particles to the list using
        ! the individual_operation.
        call dist%individual_operation(init_particles)

        ! Particle i of the particle list should now have
        ! a position (i, i, i), velocity (0, 0, 0),
        ! force (0, 0, 0) and mass 1.

    contains
        ! Initialise particle i to have position (i, i, i)
        PURE function init_particles(p_i, i)
            type(particle) :: init_particles

            type(particle), intent(in) :: p_i
            integer, intent(in) :: i

            return particle(pos=i, vel=0, force=0, mass=1)
        end function init_particles
    end program ExampleIndividualOperation
\end{lstlisting}


\subsection{The \pairoperation{} Method}
\label{sec:the_pair_operation_method}

This method accepts
a function defining a mapping from two particles to a value;
a reduction operation provided by the implementation;
a mapping from a particle and a value a new particle;
and an initial value to use for reductions.

\begin{equation}
    \begin{split}
        f &: p\times{}p \rightarrow{} v^n \\
        r &: v^n\times{}v^n \rightarrow{} v^n \\
        s &: p\times{}v^n \rightarrow{} p \\
        iv &: v^n
    \end{split}
\end{equation}

This can be considered a map/reduce operation where $f$ is the map operation
and $r$ is the reduce operation.
%
The value $iv$ is used as the initial value used in the reduce operation.
%
The function $s$ can then be used to set some value of a particle with
the results of this map/reduce operation.

The intention is that the implementation use $s$ to set the value
of particle $i$ to the value given by $iv$.
%
It will then use $f$ to compare particle
$i$ with a particle $j$, where $i \ne{} j$, and use $r$ to
reduce this new value with the current result being held for
particle $i$.
%
Performing this for every particle $j$ and reducing the results together,
the implementation should detmine the correct result for the operation.

An example serial implementation is in
\LST{lst:serial_pair_operation_implementation}.

\begin{lstlisting}[
    caption=Serial \pairoperation{} implementation.,
    label=lst:serial_pair_operation_implementation,
    gobble=3
]
    subroutine pair_operation(                     &
        this                                       &
        pair_to_value, particle_value_to_particle, &
        reduction_type, reduction_init             &
    )
        class(serial_distribution), intent(inout) &
            :: serial_distribution
        procedure(pair_to_val_function), intent(in) &
            :: pair_to_value
        procedure(set_particle_function), intent(in) &
            :: particle_value_to_particle
        procedure(reduce_op_function), intent(in) &
            :: reduction_type
        double precision, intent(in) :: reduction_init(:)

        double precision :: value(size(reduction_init))
        double precision :: tmp_value(size(reduction_init))


        do i=1, this%num_particles
            value = reduction_init

            do j=1, this%num_particles
                call pair_to_value(                       &
                    this%particles(i), this%particles(j), &
                    tmp_value                             &
                )

                value = reduction_type( &
                    value, tmp_value,   &
                )
            end do

            this%particles(i) = particle_value_to_particle(&
                this%particles(i), value                   &
            )
        end do
    end subroutine pair_operation
\end{lstlisting}

As can be seen, some design choices have been taken which limit how
general the user defined operations can be.
%
There are also some unfortunate inconsistencies in the signatures of
user defined operations.
%
These are primarily due to language restrictions.
%
However, the minimum requirement is for the distribution to be able
to implement the \velocityverlet{} algorithm,
so these problems are not a major concern.


\subsection{A Full Example}

An example of an MD simulation using this model is provided in
\LST{lst:distribution_operation}.
%
Presented is an example of creating a distribution,
initialising the particles in the simulation,
an MD update loop and
printing the list of particles out to disk.


\begin{lstlisting}[
    caption=An example MD simulation,
    label=lst:distribution_operation,
]
program MD
    ...

    dist = new_distribution(...)

    call dist%individual_operation(init_particles)

    do while(time < end_time)
        ! Perform first half of velocity Verlet scheme
        call dist%individual_operation(VV_scheme_part_1)

        ! Find forces between particles due to
        ! LJ potential
        call dist%pair_operation(                      &
            map_force, unmap_force, dist%sum, zero_arr &
        )

        ! Perform second half of velocity Verlet scheme
        call dist%individual_operation(VV_scheme_part_2)

        ! Step time forward
        time = update_time(time)
    end do


contains
    ! An overly simplistic initialisation scheme
    PURE function init_particles(particle_i, i)
        ...
        init_particles = Particle(        &
            pos=i, vel=0, mass=1, force=0 &
        )
    end function init_particles

    ! Find the force on particle1 due to particle2
    PURE subroutine map_force(        &
        particle_i, particle_j, force &
    )
        ...
        call find_LJ_force_between(        &
            particle_i, particle_j, force  &
        )
    end subroutine map_force

    ! Expect all the forces to have been summed together
    ! in the "total_force" parameter.
    ! Apply force to the particle and return it.
    PURE function unmap_force(particle_i, total_force)
        ...
        unmap_force = particle_i
        unmap_force%force = total_force
    end function unmap_force

    ! Accept a particle and return a string
    ! that can be printed.
    PURE subroutine particle_to_string( &
        particle_i, i, output_string    &
    )
        ...
        write(output_string, *) particle_i%pos
    end subroutine particle_to_string
end program MD
\end{lstlisting}


\subsection{Testing}

Interfaces for printing information on the particles
out to disk and for performing some map/reduce operations over
the particles and returning a result are also provided.
%
These are used only for testing and debugging, as they are entirely
unoptimised, and indeed, very slow.

The \individualoperation{} is tested by having each particle set
its position vector to $(i, i, i)$ where $i$ is its position in the
particle list.
%
A map/reduce operation is then performed over the particles to
sum all the positions.
%
The result is then compared to $\sum_n=1^N (n, n, n)$.
%
The test is then repeated where particles have their positions set to
$(di, di, di)$ where $d$ is an arbitrary multiplier and $i$ is their
position in the particle list, and the map/reduce result i
compared to the result of $d\sum_n=1^N (n, n, n)$.
%
This is performed for several $d$.

The \pairoperation{} is tested by having each particle's position set
to $(i, i, i)$.
%
A \pairoperation{} is performed where $f$ performed for particles $i$ and
$j$ returns the position of particle $j$.
%
The reduction operation is the element-wise summation of these results.
%
The setting operation is to set the result to the velocity parameter of
the particle.
%
The initialisation value is 0.

The result should be that the velocity parameter of each particle should
be a sum of the positions of each particle it has been compared to.
%
As a result, the velocity of particle $i$ should be equal to
$\sum_{\substack{n=1\\n\ne{}i}}^N (n, n, n)$.
%
A map/reduce operation summing all the velocities of all the particles
in the system should therefore return a value
$(N-1)\sum_{n=1}^N (n, n, n)$.


\subsection{The Replicated Distribution}

%
%Q: How is the replicated data scheme implemented?
The replicated distribution is an implementation of the \replicateddata{}
scheme.
%
In this implementation, a list of $N$ particles is kept on each process.

When the implementation is instantiated, an MPI communicator is passed to it.
%
This communicator is duplicated and several parameters, such as
the processes rank and the number of processes are determined and
set as instance variables.
%
At this point, if $P > N$, the implementation will output an error message
and exit the simulation.
%
Each process will allocate a list of $N$ particles here.

The \individualoperation{} method is implemented by having each process update
its local list of particles wth the input function.
%
No communications are performed in this method.

The \pairoperation{} method is implemented by having each process
determine a chunk of $N/P$ it should update.
%
Each process then independently updates those $N/P$ particles.
%
All processes then perform an \mpiallgatherv{} to share these
updates and ensure their list of particles is up to date.


\subsection{The Systolic Distribution}

%
%Q: How is the systolic loop scheme implemented?
The systolic distribution is an implementation of the \systolicloop{} scheme.
%
In this implementation, three lists of $N/P$ particles are allocated
on each process.
%
These lists are referred to as the local list (which corresponds to
the particles list from the abstract distribution), the send list and the
receive list.

Upon instantiation, an MPI communicator is passed to the class
which is duplicated and stored as an instance variable.
%
Several parameters, such as the current rank and the number of
processes are also set as instance variables.
%
If $P > N$, the implementation will output an error message
here and exit the simulation.
%
At this stage, the implementation determines how large a chunk of
memory it will need to store its local list, by rounding up
the result of $N/P$.
%
The three lists of particles mentioned are allocated at this point.

The \individualoperation{} method is implemented by having each process
update its local list of particles.
%
The send and receive lists are ignored in this method.

\begin{lstlisting}[
    caption=The \pairoperation{} implementation for the Systolic Loop.,
    label=lst:systolic_loop_pair_operation_implementation,
    gobble=3
]
    this%receive_list = this%particles

    do i=1, this%num_local_particles
        partial_reductions(:,i) = reduction_init
    end do

    do pulse=1, this%nprocs
        if(pulse .NE. 1) then
            this%do_systolic_pulse
        end if

        do i=1, this%num_local_particles
            do j=1, this%num_receive_particles
                if(                               &
                    this%receive_rank             &
                        .EQ. this%local_ring_rank &
                    .AND.                         &
                    i .EQ. j                      &
                ) cycle

                call pair_to_value(       &
                    this%particles(i),    &
                    this%receive_list(i), &
                    tmp_value             &
                )

                partial_reductions(:, i) = reduction_type( &
                    partial_reductions(:, i), tmp_value    &
                )
            end do
        end do
    end do

    do i=1, this%num_local_particles
        this%particles(i) = particle_value_to_value(    &
            this%particles(i), partial_reductions(:, i) &
        )
    end do
\end{lstlisting}

The \pairoperation{} method is outlined in
\LST{lst:systolic_loop_pair_operation_implementation}.
It is implemented by using the three
particle lists to perform systolic pulses.
%
On first call, this method determines the dimensions of the
reduction variable needed.
%
In the case of determining forces, this will be of size 3.
%
This is determined by finding the size of the ``reduction\_init''
variable.
%
It then allocates an array of doubles of this size multiplied by the
number of particles in its local list.
%
This list is then used to store partial reduction results across
systolic pulses.

Initially, a process will copy its local list to the receive list,
skip performing a systolic pulse for that iteration, and perform
a comparison loop over the local and receive lists, saving the
reduction results into the partial reductions array.
%
A process will then perform a systolic pulse, in which it copies its
current foreign list into its send list, and sends the send list to
the next process in the ring while waiting for data to arrive 
into the receive list from the previous process in the loop.
%
The communications are performed using an \mpisendrecv{}.

Each process will then perform a comparison loop using this new receive
array, reducing the results with the values already in the partial
reductions array.
%
After performing $P$ such comparisons and $P-1$ systolic pulses,
the results in the partial reductions array will be the correct
values to be set on the local particle list.
%
The implementation then uses the particle\_to\_val function
passed to the method to update
its list of local particles with these fully reduced values.


\subsection{Shared And Replicated Distribution}

%
%Q: How is the shared and replicated data scheme implemented?
The shared and replicated distribution is an implementation of the
shared and replcated data scheme.
%
This implementation inherits directly from the replicated data
implementation, and overloads the initialisation and pair operation
methods.

The initialisation routine is overloaded to ensure that
$P_{MPI}\times{}P_{OMP} \le{} N$, where $P_{MPI}$ is the number of MPI
processes the simulation is running on and $P_{OMP}$ is the
number of requested \openmp{} threads per MPI process.
%
If $P_{MPI}\times{}P_{OMP} > N$, the implementation will output an error message
and exit the simulation.
%
The implementation then calls the initialisation method of the
replicated data scheme.

The \pairoperation{} method is implemented exactly the same
as in the \replicateddata{} case, except a basic \openmp{}
parallel do region is defined, as outlined in
\LST{lst:shared_and_replicated_omp_region}.

\begin{lstlisting}[
    caption=\openmp{} directives in the shared and replicated distribution.,
    label=lst:shared_and_replicated_omp_region,
    gobble=3
]
    !$OMP PARALLEL DO                                  &
    !$OMP&  DEFAULT(none)                              &
    !$OMP&  PRIVATE(reduce_val, tmp_val, i, j)         &
    !$OMP&  SHARED(                                    &
    !$OMP&      reduction_identity, this, reduce_func, &
    !$OMP&      i_start, i_end                         &
    !$OMP&  )
    !
    do i=i_start, i_end
        reduce_val = reduction_init

        do j=1, this%num_particles
            if(i .EQ. j) cycle

            call pair_to_val(                         &
                this%particles(i), this%particles(j), &
                tmp_val                               &
            )

            reduce_val = reduce_func(reduce_val, tmp_val)

        end do

        this%particles(i) = val_to_particle( &
            this%particles(i), reduce_val    &
        )
    end do
    !
    !$OMP END PARALLEL DO

    call this%sync_particles
\end{lstlisting}


\subsection{Replicated Systolic Distribution}

The replicated systolic distribution is an implementation of the
replicated systolic loop.
%
In this implementation, much like the systolic distribution,
three lists of particles of size $N/S$ are allocated on each process,
referred to as the local list (corresponding to the particles list
in the abstract distribution), the send list and the receive list.

Upon instantiation, an MPI communicator is passed to the constructor.
%
The \mpidimscreate{} function is used to create a 2d decomposition
of the processes represented by the communicator passed to the
constructor.
%
These dimensions are rearranged such that the $x$ dimension of the
resulting distribution should be greater than or equal to the $y$
dimension.
%
These dimensions will correspond to the sizes of $S$ and $R$ for
the replicated systolic scheme.
%
This communicator is used to create a 2d cartesian communicator whose
dimensions are derived from the 2d decomposition that has been generated.

This cartesian communicator is then queried to find the coordinates of
the process.
%
Processes will then use \mpicommsplit{} on the cartesian communicator
to create a new communicator using their $x$ coordinate as the color.
%
This new communicator should be of size $S$ and
is stored in an instance variable as the local ring communicator.
%
Properties of the local ring communicator such as the processes rank
and the number of processes are stored as instance variables.

Processes will then use \mpicommsplit{} on the cartesian communicator
again, but this time use their rank in the local ring communicator as
the color.
%
In this manner, each process representing systolic element $i$ should
be represented by this communicator.
%
This new communicator should be of size $R$ and is stored in an instance
variable as the equivalent element communicator.
%
Properties of the equivalent element communicator such as the process rank
and the number of processes are stored as instance variables.

After this communicator setup has taken place, the instance will determine
the size of the $N/S$ local particle list, and allocate the local particle
list, the send particle list and the receive particle list.

The \individualoperation{} method is implemented by having each process update
its local particle list using the update function passed to it.
%
No communications are performed in this method and the send and receive
lists are ignored.

\begin{lstlisting}[
    caption=Replicated Systolic \pairoperation{} implementation.,
    label=lst:replicated_systolic_pair_operation_implementation,
    gobble=3
]
    ! Find pulse chunk sizes and boundaries
    call this%get_pulse_bounds(            &
        pulse_size, pulse_start, pulse_end &
    )

    ! Find initial swap ranks from pulse chunk boundaries
    call this%get_pulse_offsets(                  &
        pulse_start, pulse_end,                   &
        initial_send, initial_receive             &
    )

    ! Do the initial swap
    call MPI_Sendrecv(                                    &
        this%particles, chunk_size, this%MPI_particle,    &
        initial_send, 0,                                  &
        this%receive_list, chunk_size, this%MPI_particle, &
        initial_receive, 0,                               &
        this%local_ring_comm, MPI_STATUS_IGNORE, ierror   &
    )

    ! Initialise the partial reductions array
    do i=1, this%num_local_particles
        partial_reductions(:,i) = reduction_init
    end do

    ! Loop over systolic pulses
    do pulse=pulse_start, pulse_end

        ! Do the systolic pulse
        if(pulse .NE. pulse_star) then
            call this%do_systolic_pulse
        end if

        do i=1, this%num_local_particles
            do j=1, this%num_receive_particles

                ! Ensure a particle isn't compared
                ! with itself
                if(                               &
                    this%receive_rank             &
                        .EQ. this%local_ring_rank &
                    .AND.                         &
                    i .EQ. j                      &
                ) cycle

                ! Do comparison
                call pair_to_value(       &
                    this%particles(i),    &
                    this%receive_list(i), &
                    tmp_value             &
                )

                ! Reduce into the reductions array
                partial_reductions(:, i) = reduction_type( &
                    partial_reductions(:, i), tmp_value    &
                )
            end do
        end do
    end do


    ! Reduce over all the partial systolic pulse results
    call MPI_Allreduce(                                    &
        partial_reductions, final_partial_reductions,      &
        size(reduction_identity)*this%num_local_particles, &
        MPI_REAL_P, MPI_reduction_type,                    &
        this%equiv_elem_comm, ierror                       &
    )


    ! Set the local particles with the fully reduced results
    do i=1, this%num_local_particles
        this%particles(i) = particle_value_to_value(    &
            this%particles(i), partial_reductions(:, i) &
        )
    end do
\end{lstlisting}

The \pairoperation{} method, outlined in
\LST{lst:replicated_systolic_pair_operation_implementation},
is implemented by having each ring
determine the chunk of $S/R$ pulses it should be performing.
%
For replica $i$, for example, it should be performing pulses
$iS/R$ to $(i+1)S/R$.
%
This is realised by having each element in the ring copy its local
particles to its send list.
%
It then performs an \mpisendrecv{}, sending its send list to
the process $iS/R$ ahead and receiving to its receive list from
the process $iS/R$ behind.
%
This step looks as though $iS/R$ systolic pulses have been performed,
when this is actually performed with one communication.
%
Each process in ring $i$ will then perform $S/R$ comparison loops
with $S/R-1$ systolic pulses.

When the method is first called, an array of partial reduction
values is allocated to store the partial results across pulses.
%
This is implemented exactly as in the systolic distribution,
except instead of being of length $N/P$, it is of length $N/S$.

After performing the required pulses, each process will then perform
an \mpiallreduce{} across the equivalent element communicator.
%
This should result in a reduction over all the partial reduction
results for all of the systolic pulses performed.
%
The resulting array should be the final values to be set on each particle.
%
This array is then used to set the final values on the particles.
