We require that all parallel implementations are functionally equivalent
to the above codes.
%
This allows us to easily implement tests.
%
We simply implement tests for the serial implementation involving
performing transformations on the code and outputting the data to disk
and then analysing this output.

Further, it discourages us from making optimisations that make
assumptions about our MD algorithm.
%
We must focus solely upon parallel list comparison and updates.

Both the replicated data and systolic loop schemes were initially implemented
using MPI for parallelism.



\subsection{Replicated Data}

The replicated scheme allocates a list the size of the entire system
on each process.
%
It uses this list to keep an up-to-date copy of the system on every
process.


%
% Replicated individual_operation v0
%

\subsubsection{individual\_operation}

Individual operations are performed by having each process update
its entire local list.
This approach involves more computation than having each process
evaluate a subsection of the list and share the result with the
other processes, but it avoids a global synchronization.

This approach should take $\bigO{N}$ time.

\begin{figure}
    \input{parallel_implementation/v0/replicated.individual_operation.512.time.plt}
    \caption{Replicated Data individual\_operation Time for $N = 512$}
    \label{fig:v0_replicated_individual_operation_512_time}
\end  {figure}

\begin{figure}
    \input{parallel_implementation/v0/replicated.individual_operation.4096.time.plt}
    \caption{Replicated Data individual\_operation Time for $N = 4096$}
    \label{fig:v0_replicated_individual_operation_4096_time}
\end  {figure}

\begin{figure}
    \input{parallel_implementation/v0/replicated.individual_operation.32768.time.plt}
    \caption{Replicated Data individual\_operation Time for $N = 32768$}
    \label{fig:v0_replicated_individual_operation_32768_time}
\end  {figure}

It is plain to see from \FIG{fig:v0_replicated_individual_operation_512_time}
that the time for the individual\_operation doesn't scale with the number
of cores.
%
We see that there is no time devited to MPI operations.

There is an interesting increase of time that occurs at 2 cores and again
at 8 cores.
%
Given that \hector{} has 4 NUMA regions of 8 cores and each of those
regions is further subdivided into 4 NUMA regions of 2 cores,
it is likely this is a cause for the jumps at 2 and 8 cores.
%
This is particularly noticeable in
\FIG{fig:v0_replicated_individual_operation_32768_time}
where the jump occurs at exactly the same core count, but is noticeably larger.

It appears that after the given number of processes, the time remains
roughly constant.
%
This suggests either a latency effect with processes accessing memory
in other NUMA regionr or a memory bandwith effect.
%
Given that the effect scales roughly with the number of particles
in the system, at fixed core counts, it is most likely due to
memory bandwidth saturation.
%
Indeed, with bandwidth saturated, we would expect data transfer times
to scale linearly with the amound of data per core being transferred.
%
We may suggest that the bandwidth is not saturated in the 2 core NUMA
region when 1 core is in use, and then that the bandwidth for the 8 core
NUMA region is not saturated for 7 cores.

We also see that the time to solution scales roughly with the number
of particles in the system.

We may conclude in this implementation that the time to completion
is indeed $\bigO{N}$ as previously suggested.


%
% Replicated pair_operation v0
%

\subsubsection{pair\_operation}

Each process is assigned a set of particles for which
it will determine the forces.
the number of particles each process is assigned is roughly $N/P$
So, each process should be comparing $N/P$ particles to $N$ other
particles, suggesting a calculation term of $\bigO{N^2/P}$.

After the loop, we synchronise the updated list across processes.
This was implemented using an MPI\_Allgatherv, which
should introduce a communication term of $\bigO{(N + l)\log{P}}$
where $l$ is a constant latency.

Overall, this should take $\bigO{N^2/P + d(N+l)\log{P}}$ time
where $N$ is the number of particles,
$P$ is the number of processes and
$d$ is a constant and
$l$ is a constant latency.

\begin{figure}
    \input{parallel_implementation/v0/replicated.pair_operation.512.logtime.plt}
    \caption{Replicated Data pair\_operation Time for $N = 512$}
    \label{fig:v0_replicated_pair_operation_512_logtime}
\end  {figure}

\begin{figure}
    \input{parallel_implementation/v0/replicated.pair_operation.4096.logtime.plt}
    \caption{Replicated Data pair\_operation Time for $N = 4096$}
    \label{fig:v0_replicated_pair_operation_4096_logtime}
\end  {figure}

\begin{figure}
    \input{parallel_implementation/v0/replicated.pair_operation.32768.logtime.plt}
    \caption{Replicated Data pair\_operation Time for $N = 32768$}
    \label{fig:v0_replicated_pair_operation_32768_logtime}
\end  {figure}

In \FIG{fig:v0_replicated_pair_operation_512_logtime}, we see strong scaling results
for the pair\_operation performed normally begin to drop off as the number
of processes used is comparable to the number of particles in the system.
%
This is consistent with our predictions.
%
If $P = N/k$, then our time should be $\bigO{kN + d(N+l)\log{kN}}$.
%
We see the communication term begins to dominate our time, and
the system doesn't scale as well.

Similarly, for small $P << N$, we see our calculation term dominating
and the system appears to scale quite well.

This suggests the primary bottleneck in the replicated data approach
is in the number of particles communicated.

We expect to see this approach scale relatively well
with problem size.
%
Where a larger system is used, we may use more cores to solve the problem.
%
In particular, if there is an optimum $k$ for $P = N/k$, we see that
if we place no particular importance on core count, the time to solution
should scale roughly linearly with the problem size.

Indeed, where
\FIG{fig:v0_replicated_pair_operation_512_logtime}
and
\FIG{fig:v0_replicated_pair_operation_4096_logtime}
begin straying away from time reduction proportional to the core count,
we see that the minimum execution time of 512 particles is around
$10^{-4}$ seconds, while fo 4096 particles it is approaching
$10^{-3}$ seconds.
%
It is difficult to tell from
\FIG{fig:v0_replicated_pair_operation_32768_logtime}
where exactly the minimum execution time will be, but
extrapolating the graph with the other two in mind suggests it may
rest somewhere within $10^{-2}$ seconds.

Thus it appears the minimum execution time for this distribution
scales linearly with the number of particles for the system sizes tested.
%
Indeed, finding an optimim $k$ for $P=N/k$, it appears $k \approx{} 1$
is sufficient.



\subsection{Systolic Loop}

The systolic loop scheme is initialised by allocating 3 arrays of
size $N/P$ on every process.
%
The system is then split roughly evenly across all the processes
and is held, updated and shared using these three lists.


%
% Systolic individual_operation v0
%

\subsubsection{individual\_operation}
This is implemented by having each process update its local list
of particles.

This should take $\bigO{N/P}$ time.

\begin{figure}
    \input{parallel_implementation/v0/systolic.individual_operation.512.logtime.plt}
    \caption{Systolic Data individual\_operation Time for $N = 512$}
    \label{fig:v0_systolic_individual_operation_512_logtime}
\end  {figure}

\begin{figure}
    \input{parallel_implementation/v0/systolic.individual_operation.4096.logtime.plt}
    \caption{Systolic Data individual\_operation Time for $N = 4096$}
    \label{fig:v0_systolic_individual_operation_4096_logtime}
\end  {figure}

\begin{figure}
    \input{parallel_implementation/v0/systolic.individual_operation.32768.logtime.plt}
    \caption{Systolic Data individual\_operation Time for $N = 32768$}
    \label{fig:v0_systolic_individual_operation_32768_logtime}
\end  {figure}

As seen in
\FIG{fig:v0_systolic_individual_operation_512_logtime},
\FIG{fig:v0_systolic_individual_operation_4096_logtime} and
\FIG{fig:v0_systolic_individual_operation_512_logtime}
the current implementation satisfies this.

There appear to be a few unexpected data points in the mpi only timings,
but these are unlikely to be anything except an error.
%
As these are approaching nanosecond execution times, it is
unsurprising that they may pick up unexpected errors as the system clock
has at most a nanosecond resolution time.
%
Given the processor operates on about this time, it is also unsurprising
if some odd times may be picked up from taking measurements so
close to this time scale.


%
% Systolic pair_operation v0
%

\subsubsection{pair\_operation}
This is implemented by having each process use three lists of particles.

The first list is the processes local list of particles.

The second list will be referred to as the foreign list, and
represents a list originating from another process.

The third list is a swap list to allow a process to receive a new
foreign list list from the right
while also sending its old foreign list to the left
during a systolic pulse.

Initially, a process will copy its local list to the foreign list
and perform a partial force update on its local list using this
foreign list.
%
The system will then perform a systolic pulse.
After a systolic pulse, every foreign list should move one process
to the left in the systolic loop.
%
This is performed by copying the old foreign list into the
swap list, and sending the swap list to the left process while
receiving from the right process into the foreign list.
%
When a new foreign list is received, another partial force update
is performed on the local list.
%
This process is repeated $P-1$ times.

Each list comparison between systolic pulses should take $\bigO{(N/P)^2}$ time.
%
For a given timestep, there should be $P$ of these list comparisons
performed, giving an over calculation time of $\bigO{N^2/P}$.

Given each pulse should be passing $N/P$ particles between two processes,
we expect this to take $\bigO{N/P + l}$ time where $l$ is a constant latency.
%
With $P$ pulses on each time step, we expect a communication time of
$\bigO{(N/P + l)P}$.

This approach should take $\bigO{N^2/P + dN + dlP}$ time
where $N$ is the number of particles in the system,
$P$ is the number of processes used and
$l$ is a constant latency and
$d$ is a constant.

\begin{figure}
    \input{parallel_implementation/v0/systolic.pair_operation.512.logtime.plt}
    \caption{Systolic Data pair\_operation Time for $N = 512$}
    \label{fig:v0_systolic_pair_operation_512_logtime}
\end  {figure}

\begin{figure}
    \input{parallel_implementation/v0/systolic.pair_operation.4096.logtime.plt}
    \caption{Systolic Data pair\_operation Time for $N = 4096$}
    \label{fig:v0_systolic_pair_operation_4096_logtime}
\end  {figure}

\begin{figure}
    \input{parallel_implementation/v0/systolic.pair_operation.32768.logtime.plt}
    \caption{Systolic Data pair\_operation Time for $N = 32768$}
    \label{fig:v0_systolic_pair_operation_32768_logtime}
\end  {figure}
