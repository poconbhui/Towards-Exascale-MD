Parallel implementations are required to be functionally equivalent
to the example serial implementations outlined in \SEC{sec:methodology:subsec:implementation}.
%
Tests may then be implemented straightforwardly for the serial implementation,
and are then easily extended to the parallel implementations.

The prescribed interface discourages the use of optimisations which make
assumptions about the MD algorithm being implemented.
%
If it were known ahead of time, for example, that
only forces would be updated during a \pairoperation{},
as is the case in the \velocityverlet{} algorithm,
or that inter-atomic forces dropped to zero after a given distance,
the implementation may be able to use this information to improve
performance through specialized data layouts or communications patterns.

Optimizations for particular algorithms and MD systems are
beyond the scope of this dissertation.
%
The focus here is on general communication patterns rather than
the optimization of particular MD algorithms.

Implementations of the \replicateddata{} and \systolicloop{} schemes
using only MPI will be analysed and discussed by focusing on
the scaling results of the \individualoperation{} and \pairoperation{} methods.


\subsection{Replicated Data}

The replicated scheme allocates a list of particles the size of
the entire MD system of particles
on each process.
%
It uses this list to keep an up-to-date copy of the system of particles
on every process.

In this section, the implementation details and performance of
of the \individualoperation{} and \pairoperation{} methods
for the replicated data scheme using only MPI will be analysed and discussed.


%
% Replicated individual_operation v0
%

\subsubsection{Implementation of the \individualoperation{} Method}

The \individualoperation{} method as outlined in
\SEC{sec:the_individual_operation_method}
is implemented by having each process update its entire local list.
%
As such, it closely resembles the example serial implementation.
%
This approach involves more computation than having each process
evaluate a subsection of the list and share the result with the
other processes, but it avoids a global synchronization.
%
As each process performs an $\bigO{1}$ operation on $N$ particles with
no communications,
this implementation is expected to take a time
\begin{equation}
\label{eqn:v0_replicated_individual_operation_overall_time}
    N\bigO{1} = \bigO{N}
\end  {equation}

\begin{figure}[!h]
    \input{parallel_implementation/v0/replicated.individual_operation.512.time.plt}
    \caption{\vZeroTimeCaption{Replicated Data}{\individualoperation{}}{512}}
    \label{fig:v0_replicated_individual_operation_512_time}
\end  {figure}

\begin{figure}[!h]
    \input{parallel_implementation/v0/replicated.individual_operation.4096.time.plt}
    \caption{\vZeroTimeCaption{Replicated Data}{\individualoperation{}}{4096}}
    \label{fig:v0_replicated_individual_operation_4096_time}
\end  {figure}

\begin{figure}[!h]
    \input{parallel_implementation/v0/replicated.individual_operation.32768.time.plt}
    \caption{\vZeroTimeCaption{Replicated Data}{\individualoperation{}}{32768}}
    \label{fig:v0_replicated_individual_operation_32768_time}
\end  {figure}


\vZeroTimeExplanation
    { \FIG{fig:v0_replicated_individual_operation_512_time} }
    { \FIG{fig:v0_replicated_individual_operation_4096_time} }
    { \FIG{fig:v0_replicated_individual_operation_32768_time} }
    { \individualoperation{} }


From these, it is clear that
the time for the \individualoperation{} doesn't scale with the number
of cores and that MPI takes up no time as this method uses no MPI calls.

There is an interesting increase of time that occurs at 2 cores and again
at 8 cores.
%
Given that \hector{} has 4 NUMA regions of 8 cores and each of those
regions is further subdivided into 4 NUMA regions of 2 cores,
it is likely this is a cause for the jumps at 2 and 8 cores.
%
This is particularly noticeable in
\FIG{fig:v0_replicated_individual_operation_32768_time}
where the jump occurs at exactly the same core count, but is noticeably larger.

The jumps occur at the same numbers of processes in
\FIG{fig:v0_replicated_individual_operation_512_time},
\FIG{fig:v0_replicated_individual_operation_4096_time} and
\FIG{fig:v0_replicated_individual_operation_32768_time}
and after 8 processes, the time remains roughly constant.
%
This suggests either a latency effect with processes accessing memory
in other NUMA regions or a memory bandwith effect.
%
Given that the effect scales roughly with the number of particles
in the system, at fixed core counts, it is most likely due to
memory bandwidth saturation.
%
If bandwidth saturated, it is expected that the time taken for
data transfer would
scale linearly with the size of the data per core being transferred.
%
Therefore, it is likely that the bandwidth is not saturated in the 2 core NUMA
region when only 1 core is in use, and
similarly that the bandwidth for the 8 core NUMA region is not saturated
when only 7 cores are in use.

Above 8 cores, the overall execution time for
the \individualoperation{} method for the Replicated Data implementation
appears to scale as $\bigO{N}$
as predicted in \EQN{eqn:v0_replicated_individual_operation_overall_time}.


%
% Replicated pair_operation v0
%

\subsubsection{Implementation of the \pairoperation{} Method}

The \pairoperation{} method as outlined in
\SEC{sec:the_pair_operation_method}
is implemented by having each process evaluating the pair
comparisons for a subset of the particles and sharing the
results with the other processes.

Each process is designated a subset of the list of particles for which
it will evaluate the results of the comparison and reduction.
%
This looks similar to parallelising the outer loop of the example
serial implementation.
%
The number of particles each process is assigned is roughly $N/P$.
%
Each of these $N/P$ particles are compared to $N$ other particles
using an $\bigO{1}$ operation.
%
The time for one of the $N/P$ particles to be updated is
\begin{equation}
    N\bigO{1} = \bigO{N}
\end  {equation}
and so, the time for all $N/P$ particles to be updated is
\begin{equation}
    \frac{N}{P}\bigO{N} = \bigO{\frac{N^2}{P}}
\end  {equation}
As such, a calculation term of $\bigO{N^2/P}$ is expected.

After the each process finishes updating it's section of the list,
the updated sections of lists are shared across processes using
an MPI\_Allgatherv.
This introduces a communication term of $\bigO{(N + l)\log{P}}$
where $l$ represents a constant latency.

The overall execution time of this method should therefore be
\begin{align}
    \bigO{\frac{N^2}{P}} + \bigO{(N+l)\log{P}}
        &\approx{} \bigO{\frac{N^2}{P} + (N+l)\log{P}} \\
        &\approx{} \bigO{\frac{N^2}{P} + (N+l)\log{P}} \\
        &\approx{} \bigO{\frac{N^2}{P} + N\log{P} + \log{P}} \\
        \label{eqn:v0_replicated_pair_operation_overall_time}
        &\approx{} \bigO{\frac{N^2}{P} + N\log{P}}
\end  {align}

\begin{figure}[!h]
    \input{parallel_implementation/v0/replicated.pair_operation.512.logtime.plt}
    \caption{\vZeroTimeCaption{Replicated Data}{\pairoperation{}}{512}}
    \label{fig:v0_replicated_pair_operation_512_logtime}
\end  {figure}

\begin{figure}[!h]
    \input{parallel_implementation/v0/replicated.pair_operation.4096.logtime.plt}
    \caption{\vZeroTimeCaption{Replicated Data}{\pairoperation{}}{4096}}
    \label{fig:v0_replicated_pair_operation_4096_logtime}
\end  {figure}

\begin{figure}[!h]
    \input{parallel_implementation/v0/replicated.pair_operation.32768.logtime.plt}
    \caption{\vZeroTimeCaption{Replicated Data}{\pairoperation{}}{32768}}
    \label{fig:v0_replicated_pair_operation_32768_logtime}
\end  {figure}

\vZeroTimeExplanation
    { \FIG{fig:v0_replicated_pair_operation_512_logtime} }
    { \FIG{fig:v0_replicated_pair_operation_4096_logtime} }
    { \FIG{fig:v0_replicated_pair_operation_32768_logtime} }
    { \pairoperation{} }

In \FIG{fig:v0_replicated_pair_operation_512_logtime},
\FIG{fig:v0_replicated_pair_operation_4096_logtime} and
\FIG{fig:v0_replicated_pair_operation_32768_logtime},
it is apparent that scaling begins to drop off as the number
of processes used is comparable to the number of particles in the system.
%
From \EQN{eqn:v0_replicated_pair_operation_overall_time} when $P \ll{} N$
\begin{equation}
    \bigO{\frac{N^2}{P} + N\log{P}} \sim{} \bigO{\frac{N^2}{P}}
\end  {equation}
suggesting good scaling in this regime.
%
Similarly when $P \sim{} N$
or $P > N$
\begin{align}
    \bigO{\frac{N^2}{P} + N\log{P}}
        &\sim{} \bigO{N + N\log{P}} \\
        &\sim{} \bigO{N\log{P}}
\end  {align}
%
suggesting the communication term eventually dominating and
the overall time increasing as a function of $P$.

Indeed, examining where
\FIG{fig:v0_replicated_pair_operation_512_logtime},
\FIG{fig:v0_replicated_pair_operation_4096_logtime},
\FIG{fig:v0_replicated_pair_operation_32768_logtime} and
begin straying away from linear scaling,
it can be seen that the minimum execution time
of 512 particles is approximately $10^{-4}$ seconds,
of 4096 particles is approximately $10^{-3}$ seconds and
of 32768 particles is approximately $10^{-2}$ seconds.
%
Thus as scaling begins dropping off at $P = N$,
the minimum execution time for this distribution,
for the system sizes tested,
scales as $N\log{N}$
where $N$ is the number of particles in the system.


\section{Systolic Loop}

The \systolicloop{} scheme, as described in
\SEC{sec:background:subsec:systolic_loop},
is initialised by allocating 3 arrays of particles of
size $N/P$ on every process for storing and transferring particles
along with an array of double precision values of size $3*N/P$ for
partial results for reduction operations.
%
The system is then split roughly evenly across all the processes
and is held, updated and shared using these three lists.

In this section, the implementation details and performance of
of the \individualoperation{} and \pairoperation{} methods
for the \systolicloop{} scheme using MPI will be analysed and discussed.


%
% Systolic individual_operation v0
%

\subsection{Implementation of the \individualoperation{} Method}

The \individualoperation{} method, as outlined in
\SEC{sec:the_individual_operation_method}
for the \systolicloop{} scheme
is implemented by having each process update its local list of particles.

As each update is an $\bigO{1}$ operation and there are $N/P$ particles
to update, and as there are no MPI communications performed,
this should take a time
\begin{equation}
    \frac{N}{P}\bigO{1} = \bigO{\frac{N}{P}}
\end  {equation}

%
% Overall speedup plot
%
\begin{figure}[!h]
    \input{parallel_implementation/v0/systolic.individual_operation.logspeedup.plt}
    \caption{
        Speedup plots for the \individualoperation{} implemented with the \systolicloop{} scheme for systems of particles of size 512, 4096 and 32768.
    }
    \label{fig:v0_systolic_individual_operation_speedups}
\end{figure}


%
% Individual breakdowns
%
\begin{figure}[!h]
    \input{parallel_implementation/v0/systolic.individual_operation.512.logtime.plt}
    \caption{\vZeroTimeCaption{\systolicloop{}}{\individualoperation{}}{512}}
    \label{fig:v0_systolic_individual_operation_512_logtime}
\end  {figure}

\begin{figure}[!h]
    \input{parallel_implementation/v0/systolic.individual_operation.4096.logtime.plt}
    \caption{\vZeroTimeCaption{\systolicloop{}}{\individualoperation{}}{4096}}
    \label{fig:v0_systolic_individual_operation_4096_logtime}
\end  {figure}

\begin{figure}[!h]
    \input{parallel_implementation/v0/systolic.individual_operation.32768.logtime.plt}
    \caption{\vZeroTimeCaption{\systolicloop{}}{\individualoperation{}}{32768}}
    \label{fig:v0_systolic_individual_operation_32768_logtime}
\end  {figure}

\vZeroTimeExplanation
    {\FIG{fig:v0_systolic_individual_operation_512_logtime}}
    {\FIG{fig:v0_systolic_individual_operation_4096_logtime}}
    {\FIG{fig:v0_systolic_individual_operation_32768_logtime}}
    {\individualoperation{}}
    {\systolicloop{}}

\FIG{fig:v0_systolic_individual_operation_512_logtime},
\FIG{fig:v0_systolic_individual_operation_4096_logtime} and
\FIG{fig:v0_systolic_individual_operation_32768_logtime}
appear to follow $\bigO{N/P}$ scaling.
%
Where $P \sim{} N$, the scaling drops off slightly.
%
This could be due to the very small amount of work performed
inside the function call, where function call and loop overheads
become comparable to the calculation time.
%
This is supported by the scaling of each graph tending to drop off
as execution times reach $\sim{} 10^{-7}$~s.

The MPI only scaling times here are unexpectedly nonzero.
%
As turning calculations or MPI off in our code is implemented in
the form of global flags and conditional branches, this timing
could be a result of having to perform a function call and
evaluate a flag.
%
As previously suggested, this could have the potential to impact on
operations occuring at scales of $10^{-7}$~s, so it is plausible
that these effects may occur around $10^-{8}$~s for a much simplified
function body.


%
% Systolic pair_operation v0
%

\subsection{Implementation of the \pairoperation{} Method}

The \pairoperation{} method, as outlined in 
\SEC{sec:the_pair_operation_method}
is implemented by arranging the processes in a ring and passing
packets of particles around the ring while the processes perform
partial updates to their local particles using these packages.

This is implemented by having each process use three lists of particles,
each of size $P/N$.
%
The first list is the processes local list of particles.
%
The second list is to receive a list of particles originating from
another process.
%
The third list is used to send a receive list
to one process while simultaneously receiving a new
list of particles from another
during a systolic pulse.

When a system performs a systolic pulse,
every receive list should move one process
to the ``left'' in the \systolicloop{}.
%
This is performed by copying the old receive list into the
send list, and using an MPI\_Sendrecv to send the send list to
the ``left'' process while receiving from
the ``right'' process into the receive list.

The \systolicloop{} is initialised by having each process
copy its local list to its receive list.
%
Each process will then perform a partial update using this receive
list, and then perform a systolic pulse.
%
When a new foreign list is received, another partial force update
is performed on the local list.
%
The system will perform $P-1$ systolic pulses overall, at which point
each process should have received a list of particles originating
from each other process exactly once.

The size of both the local list and the receive list are $N/P$.
%
To perform all the partial updates for a one local particle
using a given receive list should take
\begin{equation}
    \frac{N}{P}\bigO{1} = \bigO{\frac{N}{P}}
\end  {equation}
%
Performing $N/P$ partial updates should take a time
\begin{equation}
    \frac{N}{P}\bigO{\frac{N}{P}} = \bigO{\left( \frac{N}{P} \right)^2}
\end  {equation}
%
As there are $P-1$ systolic pulses per time step, and including
the initial comparison, there should be $P$ such partial force updates
performed, giving an overall calculation time of
\begin{equation}
    P\bigO{ \left(\frac{N}{P}\right)^2 } = \bigO{\frac{N^2}{P}}
\end  {equation}

Given each pulse should be passing $N/P$ particles between two processes
with a fixed latency $l$, each pulse is expected to take a time
$\bigO{N/P + l}$.
%
With $P$ pulses on each time step, the communication per timestep should be
\begin{align}
    P\bigO{N/P + l} &= \bigO{N + lP} \\
                    &\approx{} \bigO{N + P}
\end  {align}

Combining the calculation and communication terms, the \systolicloop{} approach
should run in a time
\begin{align}
    \bigO{\frac{N^2}{P}} + \bigO{N + P}
        &\approx{} \bigO{\frac{N^2}{P} + N + P} \\
        &\approx{} \bigO{\frac{N^2}{P} + P}
\end  {align}
assuming $P \ll{} N^2$, which for this implementation will hold true.

%
% Overall speedup plot
%
\begin{figure}[!h]
    \input{parallel_implementation/v0/systolic.pair_operation.logspeedup.plt}
    \caption{
        Speedup plots for the \pairoperation{} implemented with the \systolicloop{} scheme for systems of particles of size 512, 4096 and 32768.
    }
    \label{fig:v0_systolic_pair_operation_speedups}
\end{figure}


%
% Individual breakdowns
%
\begin{figure}[!h]
    \input{parallel_implementation/v0/systolic.pair_operation.512.logtime.plt}
    \caption{\vZeroTimeCaption{\systolicloop{}}{\pairoperation{}}{512}}
    \label{fig:v0_systolic_pair_operation_512_logtime}
\end  {figure}

\begin{figure}[!h]
    \input{parallel_implementation/v0/systolic.pair_operation.4096.logtime.plt}
    \caption{\vZeroTimeCaption{\systolicloop{}}{\pairoperation{}}{4096}}
    \label{fig:v0_systolic_pair_operation_4096_logtime}
\end  {figure}

\begin{figure}[!h]
    \input{parallel_implementation/v0/systolic.pair_operation.32768.logtime.plt}
    \caption{\vZeroTimeCaption{\systolicloop{}}{\pairoperation{}}{32768}}
    \label{fig:v0_systolic_pair_operation_32768_logtime}
\end  {figure}

\vZeroTimeExplanation
{\FIG{fig:v0_systolic_pair_operation_512_logtime}}
{\FIG{fig:v0_systolic_pair_operation_4096_logtime}}
{\FIG{fig:v0_systolic_pair_operation_32768_logtime}}
{\pairoperation{}}
{\systolicloop{}}

We see in 
\FIG{fig:v0_systolic_pair_operation_512_logtime},
\FIG{fig:v0_systolic_pair_operation_4096_logtime} and
\FIG{fig:v0_systolic_pair_operation_32768_logtime}
it is clear that the system scales as $N/P$ when $P \ll{} N$.
%
However, it begins deviating significantly from linear scaling
when $N/P \approx{} 32$.

With a communication term scaling as $P$, and a calculation term
scaling as $N/P$, the communications term eventually dominates.
%
The minimum time for execution occurs in all systems tested when $N/P = 8$
and appears to scale linearly with the number of particles in the MD system.

The implimentation ultimately becomes slowed due to
communication latency caused by a large number
of systolic pulses per time step.
%
When going to much larger numbers of processes
this implementation may suffer from extra slow down due to
global synchronisation, as it was implemented using a blocking
MPI\_Sendrecv.
%
A simple remedy may be to use nonblocking communications to allow
processes to receive particles and perform partial updates while
allowing send lists to be sent in the background.

\section{Shared and Replicated Data}


\subsection{The \individualoperation{} Method}

The \individualoperation{} method is inherited directly from
the implementation outlined in
\SEC{sec:replicated_data_individual_operation_implementation}.


%
%Q: What is the speedup of the Shared And Replicated Data individual_operation method?

%
% Overall speedup plot
%
\begin{figure}[!h]
    \input{%
        parallel_implementation/v1/%
        shared_and_replicated.individual_operation.logspeedup.plt%
    }
    \caption{
        \vZeroSpeedupCaption
            {\sharedandreplicateddata{}}
            {\individualoperation{}}
            {$f(x) = 1$}
    }
    \label{fig:v1_shared_and_replicated_data_individual_operation_speedups}
\end{figure}


\vZeroSpeedupExplanation
    {\FIG{fig:v1_shared_and_replicated_data_individual_operation_speedups}}
    {\sharedandreplicateddata{}}
    {\individualoperation{}}
    {$f(x) = 1$}

Similar to the speedup graph in
\FIG{fig:v0_replicated_data_individual_operation_speedups}
for the \individualoperation{} method of the \replicateddata{}
implementation, there is no speedup to be seen here.
%
The primary point of interest is that this graph begins at 8 cores
which is a result of having the number of \openmp{} cores per
MPI process fixed at 8.

%
%Q: What is the expected calculation time of the Shared And Replicated Data individual_operation method?
The time to complete the calculations is expected to scale as $\bigO{N}$,
similarly to the \replicateddata{} implementation.
%
%Q: What is the expected communication time of the Shared And Replicated Data individual_operation method?
This implementation likewise performs no MPI communications, implying a
$\bigO{1}$ time.
%
%Q: What is the expected overall time of the Shared And Replicated Data individual_operation method?
As before, the overall time to completion is expected to be $\bigO{N}$.

%
% Individual breakdowns
%
\begin{figure}[!h]
    \input{%
        parallel_implementation/v1/%
        shared_and_replicated.individual_operation.512.time.plt%
    }
    \caption{
        \vOneSRTimeCaption
            {\sharedandreplicateddata{}}
            {\individualoperation{}}
            {512}
    }
    \label{fig:v1_shared_and_replicated_individual_operation_512_time}
\end  {figure}

\begin{figure}[!h]
    \input{%
        parallel_implementation/v1/%
        shared_and_replicated.individual_operation.4096.time.plt%
    }
    \caption{
        \vOneSRTimeCaption
            {\sharedandreplicateddata{}}
            {\individualoperation{}}
            {4096}
    }
    \label{fig:v1_shared_and_replicated_individual_operation_4096_time}
\end  {figure}

\begin{figure}[!h]
    \input{%
        parallel_implementation/v1/%
        shared_and_replicated.individual_operation.32768.time.plt%
    }
    \caption{
        \vOneSRTimeCaption
            {\sharedandreplicateddata{}}
            {\individualoperation{}}
            {32768}
    }
    \label{fig:v1_shared_and_replicated_individual_operation_32768_time}
\end  {figure}

\vOneSRTimeExplanation
    {\FIG{fig:v0_replicated_individual_operation_512_time}}
    {\FIG{fig:v0_replicated_individual_operation_4096_time}}
    {\FIG{fig:v0_replicated_individual_operation_32768_time}}
    {\individualoperation{}}
    {\replicateddata{}}


%
%Q: Where and why does scaling stop for the Shared And Replicated Data individual_operation method?
As can be seen from 
\FIG{fig:v0_replicated_individual_operation_512_time},
\FIG{fig:v0_replicated_individual_operation_4096_time} and
\FIG{fig:v0_replicated_individual_operation_32768_time}
the performace scaling does indeed scale as $\bigO{N}$, as expected
and exactly in line with the performance scaling results of the
\individualoperation{} method in the \replicateddata{} scheme.

An opportunity exists to use the \openmp{} threads available to
parallelise this particular method with little synchronisation
and communications overhead as would be present with MPI
parallelisation of this method.
%
However, this particular optimisation was left unimplemented as
the run time of this method tends to be an order of magnitude
lower than the \pairoperation{} method regardless of the
number of cores used, so any optimisation
would be insignificant to the overall run time.
%
The extra development overhead and complexity would not be justified by the
overall performance improvement to an MD loop.



\subsection{The \pairoperation{} Method}

The \pairoperation{} method is implemented in a similar manner to the
\pairoperation{} method from the \replicateddata{} scheme with the
addition of \openmp{} directives to further parallelise the
force update loop.

%
%Q: What is the speedup of the Shared And Replicated Data pair_operation method?

%
% Overall speedup plot
%
\begin{figure}[!h]
    \input{%
        parallel_implementation/v1/%
        shared_and_replicated.pair_operation.logspeedup.plt%
    }
    \caption{
        \vZeroSpeedupCaption
            {\sharedandreplicateddata{}}
            {\pairoperation{}}
            {$f(x) = x$}
    }
    \label{fig:v1_shared_and_replicated_data_pair_operation_speedups}
\end{figure}


\vZeroSpeedupExplanation
    {\FIG{fig:v1_shared_and_replicated_data_pair_operation_speedups}}
    {\sharedandreplicateddata{}}
    {\pairoperation{}}
    {$f(x) = x$}

The speedup displayed by 
\FIG{fig:v1_shared_and_replicated_data_pair_operation_speedups}
is very similar to the speedup shown for the \pairoperation{}
method of the \replicateddata{} scheme in
\FIG{fig:v0_replicated_data_pair_operation_speedups}.
%
The speedup is marginally better for higher core counts, and in
particular on larger systems of particles.
%
There is some unexpected instability in the speedup for $N=512$ particles.


%
%Q: What is the expected calculation time of the Shared And Replicated Data pair_operation method?
A copy of the full system of particles is held on each MPI process,
resulting in $P_{MPI}$ replicas of the system being created.
%
Each MPI process is then assigned $N/P_{MPI}$ particles in that system
to determine forces for.
%
Given that list of $N/P_{MPI}$ particles,
an MPI process spawns $P_{OMP}$ threads
and assigns $1/(P_{MPI} P_{OMP})$ particles to each thread.

Writing
\begin{equation}
    \label{eqn:p_eq_pmpi_pomp}
    P = P_{MPI} \times{} P_{OMP}
\end{equation}
each core therefore has $N/P$ particles
for which it must determine the forces.
%
As before, if each particles must be compared to $N$ other particles
using an $\bigO{1}$ update operation, the time to find the force for
a single particles is
\begin{equation}
    N\bigO{1} = \bigO{N}
\end{equation}
And so, the time to find the forces for $N/P$ particles is
\begin{equation}
    \label{eqn:shared_and_replicated_calculation_time}
    \frac{N}{P}\bigO{N} = \bigO{\frac{N^2}{P}}
\end{equation}

%
%Q: What is the expected communication time of the Shared And Replicated Data pair_operation method?
After the forces have been determined by each thread, the team of threads
shuts down.
%
This represents a synchronisation point for the threads within the
MPI process.
%
After this \openmp{} synchronisation point,
an MPI synchronisation point is introduced in the form of
an \mpiallgatherv{} over the $P_{MPI}$ processes
to synchronise the updated lists of particles.
%
Assuming the \mpiallgatherv{} uses a tree-like algorithm for
gathering and distributing the data, this should perform
$\log{P_{MPI}}$ communications of approximately $N$ particles
with a constant latency $l$, resulting in a communication time
\begin{equation}
    \label{eqn:shared_andreplicated_communication_time}
    \begin{split}
        \log{(P_{MPI})}\,\bigO{N + l} 
            &= \bigO{\log{(P_{MPI})}\,(N+l)} \\
            &\approx{} \bigO{N\log{P_{MPI}}}
    \end{split}
\end{equation}

Where communications in the \replicateddata{} scheme grew as $\log{P}$,
here they grow as $\log{P_{MPI}}$.
%
Using the definition of $P$ in \EQN{eqn:p_eq_pmpi_pomp} for the
\sharedandreplicateddata{} scheme, it can be seen that
\begin{equation}
    \log{P} = \log{P_{MPI}P_{OMP}} = \log{P_{MPI}} + \log{P_{OMP}}
\end{equation}
%
As a result, using $P_{OMP}$ threads per MPI process should provide
a drop in communication time of a constant $\log{P_{OMP}}$ compared
to the \replicateddata{} time.
%
Given the largest number of MPI processes used here is $32768$,
and $\log{32768} = 15$, the resulting drop in communication
time of $\log{P_{OMP}} = \log{8} = 3$ should result in a small
but noticeable improvement.


%
%Q: What is the expected overall time of the Shared And Replicated Data pair_operation method?
Combining
\EQN{eqn:shared_and_replicated_calculation_time} and
\EQN{eqn:shared_andreplicated_communication_time},
the overall execution time is expected to be
\begin{equation}
    \label{eqn:shared_and_replicated_data_overall_time}
    \bigO{\frac{N^2}{P}} + \bigO{N\log{P_{MPI}}}
        = \bigO{\frac{N^2}{P_{MPI} \times{} P_{OMP}} + N\log{P_{MPI}}}
\end{equation}


%
% Individual breakdowns
%
\begin{figure}[!h]
    \input{%
        parallel_implementation/v1/%
        shared_and_replicated.pair_operation.512.logtime.plt%
    }
    \caption{
        \vOneSRTimeCaption
            {\sharedandreplicateddata{}}
            {\pairoperation{}}
            {512}
    }
    \label{fig:v1_shared_and_replicated_pair_operation_512_logtime}
\end  {figure}

\begin{figure}[!h]
    \input{%
        parallel_implementation/v1/%
        shared_and_replicated.pair_operation.4096.logtime.plt%
    }
    \caption{
        \vOneSRTimeCaption{
            \sharedandreplicateddata{}}
            {\pairoperation{}}
            {4096}
    }
    \label{fig:v1_shared_and_replicated_pair_operation_4096_logtime}
\end  {figure}

\begin{figure}[!h]
    \input{%
        parallel_implementation/v1/%
        shared_and_replicated.pair_operation.32768.logtime.plt%
    }
    \caption{
        \vOneSRTimeCaption
            {\sharedandreplicateddata{}}
            {\pairoperation{}}
            {32768}
    }
    \label{fig:v1_shared_and_replicated_pair_operation_32768_logtime}
\end  {figure}

\vOneSRTimeExplanation
    {\FIG{fig:v1_shared_and_replicated_pair_operation_512_logtime}}
    {\FIG{fig:v1_shared_and_replicated_pair_operation_4096_logtime}}
    {\FIG{fig:v1_shared_and_replicated_pair_operation_32768_logtime}}
    {\pairoperation{}}
    {\sharedandreplicateddata{}}


%
%Q: Where and why does scaling stop for the Shared And Replicated Data pair_operation method?
From
\FIG{fig:v1_shared_and_replicated_pair_operation_512_logtime},
\FIG{fig:v1_shared_and_replicated_pair_operation_4096_logtime} and
\FIG{fig:v1_shared_and_replicated_pair_operation_32768_logtime},
it appears the communication times approach and pass the calculation
times around $P = N/4$ or $P = N/2$.
%
This is in a similar place to
\FIG{fig:v0_replicated_pair_operation_512_logtime},
\FIG{fig:v0_replicated_pair_operation_4096_logtime} and
\FIG{fig:v0_replicated_pair_operation_32768_logtime}
representing this method for the \replicateddata{} implementaiton.
%
However, the communications tend to pass slightly later and at a shallower
angle here than in the \replicateddata{} case.
%
This results in a very slightly improved speedup.

What is worth noting, however, is that this implementation performs
at least as well as the \replicateddata{} implementation, required
the inclusion of just one default \openmp{} directive and
is now capable of holding an order of magnitude more particles on
a compute node.
%
The scaling results here suggest that with more cores available to the
implementation, running system sizes at the limit of what can fit on
a node may be feasible to run.

\subsection{Replicated Systolic Loop}

\subsubsection{\individualoperation{}}

%
% Overall speedup plot
%
\begin{figure}[!h]
    \input{parallel_implementation/v1/replicated_systolic.individual_operation.logspeedup.plt}
    \caption{
        \vZeroSpeedupCaption
            {\replicatedsystolicloop{}}
            {\individualoperation{}}
    }
    \label{fig:v1_replicated_systolic_loop_individual_operation_speedups}
\end{figure}


\vZeroSpeedupExplanation
    {\FIG{fig:v1_replicated_systolic_loop_individual_operation_speedups}}
    {\replicatedsystolicloop{}}
    {\individualoperation{}}


%
% Individual breakdowns
%
\begin{figure}[!h]
    \input{parallel_implementation/v1/replicated_systolic.individual_operation.512.logtime.plt}
    \caption{\vZeroTimeCaption{\replicatedsystolicloop{}}{\individualoperation{}}{512}}
    \label{fig:v1_replicated_systolic_individual_operation_512_time}
\end  {figure}

\begin{figure}[!h]
    \input{parallel_implementation/v1/replicated_systolic.individual_operation.4096.logtime.plt}
    \caption{\vZeroTimeCaption{\replicatedsystolicloop{}}{\individualoperation{}}{4096}}
    \label{fig:v1_replicated_systolic_individual_operation_4096_time}
\end  {figure}

\begin{figure}[!h]
    \input{parallel_implementation/v1/replicated_systolic.individual_operation.32768.logtime.plt}
    \caption{\vZeroTimeCaption{\replicatedsystolicloop{}}{\individualoperation{}}{32768}}
    \label{fig:v1_replicated_systolic_individual_operation_32768_time}
\end  {figure}




\subsubsection{\pairoperation{}}

%
% Overall speedup plot
%
\begin{figure}[!h]
    \input{parallel_implementation/v1/replicated_systolic.pair_operation.logspeedup.plt}
    \caption{
        \vZeroSpeedupCaption
            {\replicatedsystolicloop{}}
            {\pairoperation{}}
    }
    \label{fig:v1_replicated_systolic_pair_operation_speedups}
\end{figure}


\vZeroSpeedupExplanation
    {\FIG{fig:v1_replicated_systolic_pair_operation_speedups}}
    {\replicatedsystolicloop{}}
    {\pairoperation{}}


%
% Individual breakdowns
%
\begin{figure}[!h]
    \input{parallel_implementation/v1/replicated_systolic.pair_operation.512.logtime.plt}
    \caption{\vZeroTimeCaption{\replicatedsystolicloop{}}{\pairoperation{}}{512}}
    \label{fig:v1_replicated_systolic_pair_operation_512_logtime}
\end  {figure}

\begin{figure}[!h]
    \input{parallel_implementation/v1/replicated_systolic.pair_operation.4096.logtime.plt}
    \caption{\vZeroTimeCaption{\replicatedsystolicloop{}}{\pairoperation{}}{4096}}
    \label{fig:v1_replicated_systolic_pair_operation_4096_logtime}
\end  {figure}

\begin{figure}[!h]
    \input{parallel_implementation/v1/replicated_systolic.pair_operation.32768.logtime.plt}
    \caption{\vZeroTimeCaption{\replicatedsystolicloop{}}{\pairoperation{}}{32768}}
    \label{fig:v1_replicated_systolic_pair_operation_32768_logtime}
\end  {figure}

