Parallel implementations are required to be functionally equivalent
to the example serial implementations outlined in \SEC{sec:methodology:subsec:implementation}.
%
Tests may then be implemented straightforwardly for the serial implementation,
and are then easily extended to the parallel implementations.

The prescribed interface discourages the use of optimisations which make
assumptions about the MD algorithm being implemented.
%
If it were known ahead of time, for example, that
only forces would be updated during a \pairoperation{},
as is the case in the \velocityverlet{} algorithm,
or that inter-atomic forces dropped to zero after a given distance,
the implementation may be able to use this information to improve
performance through specialized data layouts or communications patterns.

Optimizations for particular algorithms and MD systems are
beyond the scope of this dissertation.
%
The focus here is on general communication patterns rather than
the optimization of particular MD algorithms.

Implementations of the \replicateddata{} and \systolicloop{} schemes
using only MPI will be analysed and discussed by focusing on
the scaling results of the \individualoperation{} and \pairoperation{} methods.


\subsection{Replicated Data}

The replicated scheme allocates a list of particles the size of
the entire MD system of particles
on each process.
%
It uses this list to keep an up-to-date copy of the system of particles
on every process.

In this section, the implementation details and performance of
of the \individualoperation{} and \pairoperation{} methods
for the replicated data scheme using only MPI will be analysed and discussed.


%
% Replicated individual_operation v0
%

\subsubsection{Implementation of the \individualoperation{} Method}

The \individualoperation{} method as outlined in
\SEC{sec:the_individual_operation_method}
is implemented by having each process update its entire local list.
%
As such, it closely resembles the example serial implementation.
%
This approach involves more computation than having each process
evaluate a subsection of the list and share the result with the
other processes, but it avoids a global synchronization.
%
As each process performs an $\bigO{1}$ operation on $N$ particles with
no communications,
this implementation is expected to take a time
\begin{equation}
\label{eqn:v0_replicated_individual_operation_overall_time}
    N\bigO{1} = \bigO{N}
\end  {equation}

\begin{figure}[!h]
    \input{parallel_implementation/v0/replicated.individual_operation.512.time.plt}
    \caption{\vZeroTimeCaption{Replicated Data}{\individualoperation{}}{512}}
    \label{fig:v0_replicated_individual_operation_512_time}
\end  {figure}

\begin{figure}[!h]
    \input{parallel_implementation/v0/replicated.individual_operation.4096.time.plt}
    \caption{\vZeroTimeCaption{Replicated Data}{\individualoperation{}}{4096}}
    \label{fig:v0_replicated_individual_operation_4096_time}
\end  {figure}

\begin{figure}[!h]
    \input{parallel_implementation/v0/replicated.individual_operation.32768.time.plt}
    \caption{\vZeroTimeCaption{Replicated Data}{\individualoperation{}}{32768}}
    \label{fig:v0_replicated_individual_operation_32768_time}
\end  {figure}


\vZeroTimeExplanation
    { \FIG{fig:v0_replicated_individual_operation_512_time} }
    { \FIG{fig:v0_replicated_individual_operation_4096_time} }
    { \FIG{fig:v0_replicated_individual_operation_32768_time} }
    { \individualoperation{} }


From these, it is clear that
the time for the \individualoperation{} doesn't scale with the number
of cores and that MPI takes up no time as this method uses no MPI calls.

There is an interesting increase of time that occurs at 2 cores and again
at 8 cores.
%
Given that \hector{} has 4 NUMA regions of 8 cores and each of those
regions is further subdivided into 4 NUMA regions of 2 cores,
it is likely this is a cause for the jumps at 2 and 8 cores.
%
This is particularly noticeable in
\FIG{fig:v0_replicated_individual_operation_32768_time}
where the jump occurs at exactly the same core count, but is noticeably larger.

The jumps occur at the same numbers of processes in
\FIG{fig:v0_replicated_individual_operation_512_time},
\FIG{fig:v0_replicated_individual_operation_4096_time} and
\FIG{fig:v0_replicated_individual_operation_32768_time}
and after 8 processes, the time remains roughly constant.
%
This suggests either a latency effect with processes accessing memory
in other NUMA regions or a memory bandwith effect.
%
Given that the effect scales roughly with the number of particles
in the system, at fixed core counts, it is most likely due to
memory bandwidth saturation.
%
If bandwidth saturated, it is expected that the time taken for
data transfer would
scale linearly with the size of the data per core being transferred.
%
Therefore, it is likely that the bandwidth is not saturated in the 2 core NUMA
region when only 1 core is in use, and
similarly that the bandwidth for the 8 core NUMA region is not saturated
when only 7 cores are in use.

Above 8 cores, the overall execution time for
the \individualoperation{} method for the Replicated Data implementation
appears to scale as $\bigO{N}$
as predicted in \EQN{eqn:v0_replicated_individual_operation_overall_time}.


%
% Replicated pair_operation v0
%

\subsubsection{Implementation of the \pairoperation{} Method}

The \pairoperation{} method as outlined in
\SEC{sec:the_pair_operation_method}
is implemented by having each process evaluating the pair
comparisons for a subset of the particles and sharing the
results with the other processes.

Each process is designated a subset of the list of particles for which
it will evaluate the results of the comparison and reduction.
%
This looks similar to parallelising the outer loop of the example
serial implementation.
%
The number of particles each process is assigned is roughly $N/P$.
%
Each of these $N/P$ particles are compared to $N$ other particles
using an $\bigO{1}$ operation.
%
The time for one of the $N/P$ particles to be updated is
\begin{equation}
    N\bigO{1} = \bigO{N}
\end  {equation}
and so, the time for all $N/P$ particles to be updated is
\begin{equation}
    \frac{N}{P}\bigO{N} = \bigO{\frac{N^2}{P}}
\end  {equation}
As such, a calculation term of $\bigO{N^2/P}$ is expected.

After the each process finishes updating it's section of the list,
the updated sections of lists are shared across processes using
an MPI\_Allgatherv.
This introduces a communication term of $\bigO{(N + l)\log{P}}$
where $l$ represents a constant latency.

The overall execution time of this method should therefore be
\begin{align}
    \bigO{\frac{N^2}{P}} + \bigO{(N+l)\log{P}}
        &\approx{} \bigO{\frac{N^2}{P} + (N+l)\log{P}} \\
        &\approx{} \bigO{\frac{N^2}{P} + (N+l)\log{P}} \\
        &\approx{} \bigO{\frac{N^2}{P} + N\log{P} + \log{P}} \\
        \label{eqn:v0_replicated_pair_operation_overall_time}
        &\approx{} \bigO{\frac{N^2}{P} + N\log{P}}
\end  {align}

\begin{figure}[!h]
    \input{parallel_implementation/v0/replicated.pair_operation.512.logtime.plt}
    \caption{\vZeroTimeCaption{Replicated Data}{\pairoperation{}}{512}}
    \label{fig:v0_replicated_pair_operation_512_logtime}
\end  {figure}

\begin{figure}[!h]
    \input{parallel_implementation/v0/replicated.pair_operation.4096.logtime.plt}
    \caption{\vZeroTimeCaption{Replicated Data}{\pairoperation{}}{4096}}
    \label{fig:v0_replicated_pair_operation_4096_logtime}
\end  {figure}

\begin{figure}[!h]
    \input{parallel_implementation/v0/replicated.pair_operation.32768.logtime.plt}
    \caption{\vZeroTimeCaption{Replicated Data}{\pairoperation{}}{32768}}
    \label{fig:v0_replicated_pair_operation_32768_logtime}
\end  {figure}

\vZeroTimeExplanation
    { \FIG{fig:v0_replicated_pair_operation_512_logtime} }
    { \FIG{fig:v0_replicated_pair_operation_4096_logtime} }
    { \FIG{fig:v0_replicated_pair_operation_32768_logtime} }
    { \pairoperation{} }

In \FIG{fig:v0_replicated_pair_operation_512_logtime},
\FIG{fig:v0_replicated_pair_operation_4096_logtime} and
\FIG{fig:v0_replicated_pair_operation_32768_logtime},
it is apparent that scaling begins to drop off as the number
of processes used is comparable to the number of particles in the system.
%
From \EQN{eqn:v0_replicated_pair_operation_overall_time} when $P \ll{} N$
\begin{equation}
    \bigO{\frac{N^2}{P} + N\log{P}} \sim{} \bigO{\frac{N^2}{P}}
\end  {equation}
suggesting good scaling in this regime.
%
Similarly when $P \sim{} N$
or $P > N$
\begin{align}
    \bigO{\frac{N^2}{P} + N\log{P}}
        &\sim{} \bigO{N + N\log{P}} \\
        &\sim{} \bigO{N\log{P}}
\end  {align}
%
suggesting the communication term eventually dominating and
the overall time increasing as a function of $P$.

Indeed, examining where
\FIG{fig:v0_replicated_pair_operation_512_logtime},
\FIG{fig:v0_replicated_pair_operation_4096_logtime},
\FIG{fig:v0_replicated_pair_operation_32768_logtime} and
begin straying away from linear scaling,
it can be seen that the minimum execution time
of 512 particles is approximately $10^{-4}$ seconds,
of 4096 particles is approximately $10^{-3}$ seconds and
of 32768 particles is approximately $10^{-2}$ seconds.
%
Thus as scaling begins dropping off at $P = N$,
the minimum execution time for this distribution,
for the system sizes tested,
scales as $N\log{N}$
where $N$ is the number of particles in the system.


\subsection{Systolic Loop}

The systolic loop scheme is initialised by allocating 3 arrays of
size $N/P$ on every process.
%
The system is then split roughly evenly across all the processes
and is held, updated and shared using these three lists.


%
% Systolic individual_operation v0
%

\subsubsection{Implementation of the \individualoperation{} Method}
This is implemented by having each process update its local list
of particles.

This should take $\bigO{N/P}$ time.

\begin{figure}[!h]
    \input{parallel_implementation/v0/systolic.individual_operation.512.logtime.plt}
    \caption{\vZeroTimeCaption{Systolic Loop}{\individualoperation{}}{512}}
    \label{fig:v0_systolic_individual_operation_512_logtime}
\end  {figure}

\begin{figure}[!h]
    \input{parallel_implementation/v0/systolic.individual_operation.4096.logtime.plt}
    \caption{\vZeroTimeCaption{Systolic Loop}{\individualoperation{}}{4096}}
    \label{fig:v0_systolic_individual_operation_4096_logtime}
\end  {figure}

\begin{figure}[!h]
    \input{parallel_implementation/v0/systolic.individual_operation.32768.logtime.plt}
    \caption{\vZeroTimeCaption{Systolic Loop}{\individualoperation{}}{32768}}
    \label{fig:v0_systolic_individual_operation_32768_logtime}
\end  {figure}

As seen in
\FIG{fig:v0_systolic_individual_operation_512_logtime},
\FIG{fig:v0_systolic_individual_operation_4096_logtime} and
\FIG{fig:v0_systolic_individual_operation_32768_logtime}
the current implementation satisfies this.

There appear to be a few unexpected data points in the mpi only timings,
but these are unlikely to be anything except an error.
%
As these are approaching nanosecond execution times, it is
unsurprising that they may pick up unexpected errors as the system clock
has at most a nanosecond resolution time.
%
Given the processor operates on about this time, it is also unsurprising
if some odd times may be picked up from taking measurements so
close to this time scale.


%
% Systolic pair_operation v0
%

\subsubsection{Implementation of the \pairoperation{} Method}
This is implemented by having each process use three lists of particles,
each of size $P/N$.

The first list is the processes local list of particles.

The second list will be referred to as the foreign list, and
represents a list originating from another process.

The third list is a swap list to allow a process to receive a new
foreign list list from the right
while also sending its old foreign list to the left
during a systolic pulse.

Initially, a process will copy its local list to the foreign list
and perform a partial force update on its local list using this
foreign list.
%
The system will then perform a systolic pulse.
After a systolic pulse, every foreign list should move one process
to the left in the systolic loop.
%
This is performed by copying the old foreign list into the
swap list, and using an MPI\_sendrecv to send the swap list to
the left process while receiving from
the right process into the foreign list.
%
When a new foreign list is received, another partial force update
is performed on the local list.
%
This process is repeated $P-1$ times.

Each list comparison between systolic pulses should take $\bigO{(N/P)^2}$ time.
%
For a given timestep, there should be $P$ of these list comparisons
performed, giving an over calculation time of $\bigO{N^2/P}$.

Given each pulse should be passing $N/P$ particles between two processes,
we expect this to take $\bigO{N/P + l}$ time where $l$ is a constant latency.
%
With $P$ pulses on each time step, we expect a communication time of
$\bigO{(N/P + l)P}$.

Combining our calculation and communication terms, the systolic loop approach
should run in $\bigO{N^2/P + dN + dlP}$ time
where $N$ is the number of particles in the system,
$P$ is the number of processes used and
$l$ is a constant latency and
$d$ is a constant.

\begin{figure}[!h]
    \input{parallel_implementation/v0/systolic.pair_operation.512.logtime.plt}
    \caption{\vZeroTimeCaption{Systolic Loop}{\pairoperation{}}{512}}
    \label{fig:v0_systolic_pair_operation_512_logtime}
\end  {figure}

\begin{figure}[!h]
    \input{parallel_implementation/v0/systolic.pair_operation.4096.logtime.plt}
    \caption{\vZeroTimeCaption{Systolic Loop}{\pairoperation{}}{4096}}
    \label{fig:v0_systolic_pair_operation_4096_logtime}
\end  {figure}

\begin{figure}[!h]
    \input{parallel_implementation/v0/systolic.pair_operation.32768.logtime.plt}
    \caption{\vZeroTimeCaption{Systolic Loop}{\pairoperation{}}{32768}}
    \label{fig:v0_systolic_pair_operation_32768_logtime}
\end  {figure}

We see in 
\FIG{fig:v0_systolic_pair_operation_512_logtime},
\FIG{fig:v0_systolic_pair_operation_4096_logtime} and
\FIG{fig:v0_systolic_pair_operation_32768_logtime}
that, much like in the replicated case, that the system scales
roughly as $N/P$ when $P \ll{} N$.

With a communication term scaling as $P$, we see the point in which
communications dominate comes much sooner.
%
However, we also note that it appears when $P \approx{} N$.
%
Taking our previous approach of finding an optimum $k$ for $P = N/k$,
we find $k \approx{} 8$.

This appears to hold for our three system sizes, although there does appear
to be an unexpected jump in
\FIG{fig:v0_systolic_pair_operation_32768_logtime}
between 512 and 1024 processes.
%
It is unclear whether this is a genuine effect, or simply an error in
measurement.

Ignoring the unexpected jump in 
\FIG{fig:v0_systolic_pair_operation_32768_logtime},
we may conclude that the minimum time to completion for out system
scales linearly with the number of particles.

It would appear the implimentation ultimately becomes slowed due to
communication latency, and in particular, due to a large number
of these communications.
%
The initial scaling of this term is due to the number of particles
in the system, however,
scaling with $P$ is a result of having $P$ communications
per time step.
%
Looking at our derivation for communication times,
we conclude that this must be an effect of latency.

\section{Shared and Replicated Data}

The \sharedandreplicateddata{} scheme is implemented in exactly the same
way as the \replicateddata{} scheme outlined in 
\SEC{sec:replicated_data_implementation},
except the update loop in the \pairoperation{} method is further
parallelised using \openmp{} directives, taking advantage of shared
memory between cores.
%
In fact, this distribution class inherits directly from the
\replicateddata{} distribution class, and overloads only the
\pairoperation{} method.


The primary motivation for this is to show that mixed mode MPI and \openmp{}
paralellism is just as viable as MPI only parallelism for a \replicateddata{}
scheme along with how easily the mixed mode parallelism may be implemented
on top of an MPI implementation.
%
The importance of this is that the maximum system size per core per node
can be increased in proportion to the number of \openmp{} threads
created per MPI process, which is of particular importance for nodes with
particularly high core counts.

The \sharedandreplicateddata{} distribution is initialised by
allocating a list of particles the size of the system of particles
on every MPI process.
%
The number of \openmp{} threads used per MPI processes is fixed at 8,
as this is the suggested number for \hector{} due to the arrangement
of the \numa{} regions into groups of 8.
%
As this is a rather low number compared to the overall number of MPI
processes, the emphasis here isn't to gain any perticular performance
improvement using \openmp{}, only to show that it is viable.
%
Indeed, the mere introduction of 8 threads per MPI process should increase
the maximum system size that can be run 8 fold.

In this section,
the implementation details and performance of
the \individualoperation{} and \pairoperation{} methods
will be presented and analysed.


\subsection{\individualoperation{}}

The \individualoperation{} method is inherited directly from
the implementation outlined in
\SEC{sec:replicated_data_individual_operation_implementation}.
%
As a result, the time to completion is also expected to scale as $\bigO{N}$.

%
% Overall speedup plot
%
\begin{figure}[!h]
    \input{parallel_implementation/v1/shared_and_replicated.individual_operation.logspeedup.plt}
    \caption{
        Speedup plots for the \individualoperation{} implemented with the \sharedandreplicateddata{} scheme for systems of particles of size 512, 4096 and 32768.
    }
    \label{fig:v1_shared_and_replicated_data_individual_operation_speedups}
\end{figure}


%
% Individual breakdowns
%
\begin{figure}[!h]
    \input{parallel_implementation/v1/shared_and_replicated.individual_operation.512.time.plt}
    \caption{\vZeroTimeCaption{\sharedandreplicateddata{}}{\individualoperation{}}{512}}
    \label{fig:v1_shared_and_replicated_individual_operation_512_time}
\end  {figure}

\begin{figure}[!h]
    \input{parallel_implementation/v1/shared_and_replicated.individual_operation.4096.time.plt}
    \caption{\vZeroTimeCaption{\sharedandreplicateddata{}}{\individualoperation{}}{4096}}
    \label{fig:v1_shared_and_replicated_individual_operation_4096_time}
\end  {figure}

\begin{figure}[!h]
    \input{parallel_implementation/v1/shared_and_replicated.individual_operation.32768.time.plt}
    \caption{\vZeroTimeCaption{\sharedandreplicateddata{}}{\individualoperation{}}{32768}}
    \label{fig:v1_shared_and_replicated_individual_operation_32768_time}
\end  {figure}

\vZeroTimeExplanation
    {\FIG{fig:v0_replicated_individual_operation_512_time}}
    {\FIG{fig:v0_replicated_individual_operation_4096_time}}
    {\FIG{fig:v0_replicated_individual_operation_32768_time}}
    {\individualoperation{}}
    {\replicateddata{}}

As can be seen from 
\FIG{fig:v0_replicated_individual_operation_512_time},
\FIG{fig:v0_replicated_individual_operation_4096_time} and
\FIG{fig:v0_replicated_individual_operation_32768_time}
the performace scaling does indeed scale as $\bigO{N}$, as expected
and exactly in line with the performance scaling results of the
\individualoperation{} method in the \replicateddata{} scheme.



\subsection{\pairoperation{}}

The \pairoperation{} method is implemented in a similar manner to the
\pairoperation{} method from the \replicateddata{} scheme with the
addition of \openmp{} directives to further parallelise the
force update loop.

A copy of the system is held on each MPI process, meaning there
are $P_{MPI}$ replicas of the system created.
%
Each MPI process is then assigned $N/P_{MPI}$ particles in that system
to determine forces for.
%
Given that list of $N/P_{MPI}$ particles,
an MPI process spawns $P_{OMP}$ threads
and assigns $1/(P_{MPI} P_{OMP})$ particles to each thread.

Writing $P = P_{MPI} \times{} P_{OMP}$,
each core therefore has $N/P$ particles
for which it must determine the forces.
%
As before, if each particles must be compared to $N$ other particles
using an $\bigO{1}$ update operation, the time to find the force for
a single particles is
\begin{equation}
    N\bigO{1} = \bigO{N}
\end{equation}
And so, the time to find the forces for $N/P$ particles is
\begin{equation}
    \frac{N}{P}\bigO{N} = \bigO{\frac{N^2}{P}}
\end{equation}

After the forces have been determined by each thread, the team of threads
shuts down.
%
This represents a synchronisation point for the threads within the
MPI process.
%
After this, an MPI\_Allgatherv is used over the $P_{MPI}$ processes
to synchronise the updated lists of particles, representing a
$\bigO{N\log{P_{OMP}}}$ operation.

%
% Overall speedup plot
%
\begin{figure}[!h]
    \input{parallel_implementation/v1/shared_and_replicated.pair_operation.logspeedup.plt}
    \caption{
        Speedup plots for the \pairoperation{} implemented with the \sharedandreplicateddata{} scheme for systems of particles of size 512, 4096 and 32768.
    }
    \label{fig:v1_shared_and_replicated_data_pair_operation_speedups}
\end{figure}


%
% Individual breakdowns
%
\begin{figure}[!h]
    \input{parallel_implementation/v1/shared_and_replicated.pair_operation.512.logtime.plt}
    \caption{\vZeroTimeCaption{\sharedandreplicateddata{}}{\pairoperation{}}{512}}
    \label{fig:v1_shared_and_replicated_pair_operation_512_logtime}
\end  {figure}

\begin{figure}[!h]
    \input{parallel_implementation/v1/shared_and_replicated.pair_operation.4096.logtime.plt}
    \caption{\vZeroTimeCaption{\sharedandreplicateddata{}}{\pairoperation{}}{4096}}
    \label{fig:v1_shared_and_replicated_pair_operation_4096_logtime}
\end  {figure}

\begin{figure}[!h]
    \input{parallel_implementation/v1/shared_and_replicated.pair_operation.32768.logtime.plt}
    \caption{\vZeroTimeCaption{\sharedandreplicateddata{}}{\pairoperation{}}{32768}}
    \label{fig:v1_shared_and_replicated_pair_operation_32768_logtime}
\end  {figure}

\section{Shared and Replicated Data}

\subsection{\individualoperation{}}

%
% Overall speedup plot
%
\begin{figure}[!h]
    \input{parallel_implementation/v1/replicated_systolic.individual_operation.logspeedup.plt}
    \caption{
        Speedup plots for the \individualoperation{} implemented with the \replicatedsystolicloop{} scheme for systems of particles of size 512, 4096 and 32768.
    }
    \label{fig:v1_replicated_systolic_loop_individual_operation_speedups}
\end{figure}


%
% Individual breakdowns
%
\begin{figure}[!h]
    \input{parallel_implementation/v1/replicated_systolic.individual_operation.512.logtime.plt}
    \caption{\vZeroTimeCaption{\replicatedsystolicloop{}}{\individualoperation{}}{512}}
    \label{fig:v1_replicated_systolic_individual_operation_512_time}
\end  {figure}

\begin{figure}[!h]
    \input{parallel_implementation/v1/replicated_systolic.individual_operation.4096.logtime.plt}
    \caption{\vZeroTimeCaption{\replicatedsystolicloop{}}{\individualoperation{}}{4096}}
    \label{fig:v1_replicated_systolic_individual_operation_4096_time}
\end  {figure}

\begin{figure}[!h]
    \input{parallel_implementation/v1/replicated_systolic.individual_operation.32768.logtime.plt}
    \caption{\vZeroTimeCaption{\replicatedsystolicloop{}}{\individualoperation{}}{32768}}
    \label{fig:v1_replicated_systolic_individual_operation_32768_time}
\end  {figure}




\subsection{\pairoperation{}}

%
% Overall speedup plot
%
\begin{figure}[!h]
    \input{parallel_implementation/v1/replicated_systolic.pair_operation.logspeedup.plt}
    \caption{
        Speedup plots for the \pairoperation{} implemented with the \replicatedsystolicloop{} scheme for systems of particles of size 512, 4096 and 32768.
    }
    \label{fig:v1_replicated_systolic_pair_operation_speedups}
\end{figure}


%
% Individual breakdowns
%
\begin{figure}[!h]
    \input{parallel_implementation/v1/replicated_systolic.pair_operation.512.logtime.plt}
    \caption{\vZeroTimeCaption{\replicatedsystolicloop{}}{\pairoperation{}}{512}}
    \label{fig:v1_replicated_systolic_pair_operation_512_logtime}
\end  {figure}

\begin{figure}[!h]
    \input{parallel_implementation/v1/replicated_systolic.pair_operation.4096.logtime.plt}
    \caption{\vZeroTimeCaption{\replicatedsystolicloop{}}{\pairoperation{}}{4096}}
    \label{fig:v1_replicated_systolic_pair_operation_4096_logtime}
\end  {figure}

\begin{figure}[!h]
    \input{parallel_implementation/v1/replicated_systolic.pair_operation.32768.logtime.plt}
    \caption{\vZeroTimeCaption{\replicatedsystolicloop{}}{\pairoperation{}}{32768}}
    \label{fig:v1_replicated_systolic_pair_operation_32768_logtime}
\end  {figure}

