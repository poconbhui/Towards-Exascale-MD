\section{Systolic Loop}

%
% Systolic individual_operation v0
%

\subsection{The \individualoperation{} Method}

The \individualoperation{} method, as outlined in
\SEC{sec:the_individual_operation_method}
for the \systolicloop{} scheme
is implemented by having each process update its local list of particles.


%
% Overall speedup plot
%
\begin{figure}[!h]
    \input{%
        parallel_implementation/v0/%
        systolic.individual_operation.logspeedup.plt%
    }
    \caption{
        \vZeroSpeedupCaption
            {\systolicloop{}}
            {\individualoperation{}}
            {$f(x) = x$}
    }
    \label{fig:v0_systolic_individual_operation_speedups}
\end{figure}


\vZeroSpeedupExplanation
    {\FIG{fig:v0_systolic_individual_operation_speedups}}
    {\systolicloop{}}
    {\individualoperation{}}
    {$f(x) = x$}


%
%Q: What is the speedup of the Systolic Loop individual_operation method?

\FIG{fig:v0_systolic_individual_operation_speedups} quite good speedup
up to $P = N$.
%
It is interesting that the speedup is not perfect, given there are
no communications performed in this method.

%
%Q: What is the expected calculation time of the Systolic Loop individual_operation method?
As each update is an $\bigO{1}$ operation and there are $N/P$ particles
to update,
the calculations should take a time
\begin{equation}
    \frac{N}{P}\bigO{1} = \bigO{\frac{N}{P}}
\end  {equation}
%
%Q: What is the expected communication time of the Systolic Loop individual_operation method?
As there are no MPI communications performed, the communications should
take a time $\bigO{1}$, to allow for function calls and branch evaluations.
%
%Q: What is the expected overall time of the Systolic Loop individual_operation method?
The overall time should therefore be
\begin{equation}
    \bigO{\frac{N}{P}} + \bigO{1} = \bigO{\frac{N}{P}}
\end{equation}


%
% Individual breakdowns
%
\begin{figure}[!h]
    \input{%
        parallel_implementation/v0/%
        systolic.individual_operation.512.logtime.plt%
    }
    \caption{
        \vZeroTimeCaption
            {\systolicloop{}}
            {\individualoperation{}}
            {512}
    }
    \label{fig:v0_systolic_individual_operation_512_logtime}
\end  {figure}

\begin{figure}[!h]
    \input{%
        parallel_implementation/v0/%
        systolic.individual_operation.4096.logtime.plt%
    }
    \caption{
        \vZeroTimeCaption
            {\systolicloop{}}
            {\individualoperation{}}
            {4096}
    }
    \label{fig:v0_systolic_individual_operation_4096_logtime}
\end  {figure}

\begin{figure}[!h]
    \input{%
        parallel_implementation/v0/%
        systolic.individual_operation.32768.logtime.plt%
    }
    \caption{
        \vZeroTimeCaption
            {\systolicloop{}}
            {\individualoperation{}}
            {32768}
    }
    \label{fig:v0_systolic_individual_operation_32768_logtime}
\end  {figure}

\vZeroTimeExplanation
    {\FIG{fig:v0_systolic_individual_operation_512_logtime}}
    {\FIG{fig:v0_systolic_individual_operation_4096_logtime}}
    {\FIG{fig:v0_systolic_individual_operation_32768_logtime}}
    {\individualoperation{}}
    {\systolicloop{}}


%
%Q: Where and why does scaling stop for the Systolic Loop individual_operation method?
\FIG{fig:v0_systolic_individual_operation_512_logtime},
\FIG{fig:v0_systolic_individual_operation_4096_logtime} and
\FIG{fig:v0_systolic_individual_operation_32768_logtime}
appear to follow $\bigO{N/P}$ scaling.
%
Where $P \sim{} N$, the scaling drops off slightly.
%
This could be due to the very small amount of work performed
inside the function call, where function call and loop overheads
become comparable to the calculation time.
%
This is supported by the scaling of each graph tending to drop off
as execution times reach $\sim{} 10^{-7}$~s.

The MPI only scaling times here are unexpectedly nonzero.
%
As turning calculations or MPI off in the code is implemented in
the form of global flags and conditional branches, this timing
could be a result of having to perform a function call and
evaluate a flag.
%
As previously suggested, this could have the potential to impact on
operations occuring at scales of $10^{-7}$~s, so it is plausible
that these effects may occur around $10^-{8}$~s for a much simplified
function body.


%
% Systolic pair_operation v0
%

\subsection{The \pairoperation{} Method}

The \pairoperation{} method, as outlined in 
\SEC{sec:the_pair_operation_method}
is implemented by arranging the processes in a ring and passing
packets of particles around the ring while the processes perform
partial updates to their local particles using these packages.


%
%Q: What is the speedup of the Systolic Loop pair_operation method?

%
% Overall speedup plot
%
\begin{figure}[!h]
    \input{%
        parallel_implementation/v0/%
        systolic.pair_operation.logspeedup.plt%
    }
    \caption{
        \vZeroSpeedupCaption
            {\systolicloop{}}
            {\pairoperation{}}
            {$f(x) = x$}
    }
    \label{fig:v0_systolic_pair_operation_speedups}
\end{figure}


\vZeroSpeedupExplanation
    {\FIG{fig:v0_systolic_pair_operation_speedups}}
    {\systolicloop{}}
    {\pairoperation{}}
    {$f(x) = x$}

The speedup for each system size in 
\FIG{fig:v0_systolic_pair_operation_speedups}
appears to level off rapidly around $P = N/8$.
%
Until that point, speedup appears to be almost perfectly linear.
%
The point where the speedup graphs begin tipping over appears to
scale roughly with the number of particles in the system,
although not quite perfectly.


The \pairoperation{} method of the \systolicloop{} scheme is
implemented by having each process use three lists of particles,
each of size $N/P$.
%
The first list is the processes local list of particles.
%
The second list is used to receive a list of particles originating from
another process.
%
The third list is used to send a list of particles to another process.

When a systolic pulse is performed,
each process will copy its current ``receive''
list into its ``send'' list and then perform an \mpisendrecv{}, receiving
into the receive list from the ``left'' process and sending the send list
to the ``right'' process.
%
On each pulse, the process will compare the $N/P$ particles in its
local list to the $N/P$ particles it has just received and
add partial updates to a ``partial results'' array.

%
%Q: What is the expected calculation time of the Systolic Loop pair_operation method?
To perform all the partial updates for a one local particle
with a given receive list using an $\bigO{1}$ operation should take
\begin{equation}
    \frac{N}{P}\bigO{1} = \bigO{\frac{N}{P}}
\end  {equation}
%
Performing $N/P$ partial updates should take a time
\begin{equation}
    \frac{N}{P}\bigO{\frac{N}{P}} = \bigO{\left( \frac{N}{P} \right)^2}
\end  {equation}
%
As there are $P$ systolic pulses performed per time step,
there should be $P$ such partial updates
performed, giving an overall calculation time of
\begin{equation}
    \label{eqn:systolic_loop_pair_operation_overall_time}
    P\bigO{ \left(\frac{N}{P}\right)^2 } = \bigO{\frac{N^2}{P}}
\end  {equation}


For this particular implementation,
this method is initialised by having each process
copy its local list to its receive list.
%
Each process will then perform a partial update using this receive
list, and then perform a systolic pulse.
%
When a new foreign list is received, another partial force update
is performed on the local list.
%
The system will perform $P-1$ systolic pulses overall, at which point
each process should have received a list of particles originating
from each other process exactly once.
%
This doesn't effect the result of
\EQN{eqn:systolic_loop_pair_operation_overall_time}.


%
%Q: What is the expected communication time of the Systolic Loop pair_operation method?
Given each pulse should be passing $N/P$ particles between two processes
with a fixed latency $l$, each pulse is expected to take a time
$\bigO{N/P + l}$.
%
With $P-1$ pulses on each time step, the communication per timestep should be
\begin{equation}
    \begin{split}
        (P-1)\bigO{N/P + l}
            &= \bigO{(N/P + l)(P-1)} \\
            &= \bigO{(N + lP - N/P - l} \\
            &\approx{} \bigO{N + lP} \\
            &\approx{} \bigO{N + P}
    \end{split}
\end{equation}

%
%Q: What is the expected overall time of the Systolic Loop pair_operation method?
Combining the calculation and communication terms, the \systolicloop{} approach
should run in a time
\begin{equation}
    \begin{split}
        \bigO{\frac{N^2}{P}} + \bigO{N + P - N/P}
            &\approx{} \bigO{\frac{N^2}{P} + N + P - N/P} \\
            &= \bigO{\frac{N^2}{P} + P}
    \end{split}
\end{equation}
assuming $P \ll{} N^2$, which for this implementation will hold true.


%
% Individual breakdowns
%
\begin{figure}[!h]
    \input{%
        parallel_implementation/v0/%
        systolic.pair_operation.512.logtime.plt%
    }
    \caption{
        \vZeroTimeCaption
            {\systolicloop{}}
            {\pairoperation{}}
            {512}
    }
    \label{fig:v0_systolic_pair_operation_512_logtime}
\end  {figure}

\begin{figure}[!h]
    \input{%
        parallel_implementation/v0/%
        systolic.pair_operation.4096.logtime.plt%
    }
    \caption{
        \vZeroTimeCaption
        {\systolicloop{}}
        {\pairoperation{}}
        {4096}
    }
    \label{fig:v0_systolic_pair_operation_4096_logtime}
\end  {figure}

\begin{figure}[!h]
    \input{%
        parallel_implementation/v0/%
        systolic.pair_operation.32768.logtime.plt%
    }
    \caption{
        \vZeroTimeCaption
            {\systolicloop{}}
            {\pairoperation{}}
            {32768}
    }
    \label{fig:v0_systolic_pair_operation_32768_logtime}
\end  {figure}

\vZeroTimeExplanation
{\FIG{fig:v0_systolic_pair_operation_512_logtime}}
{\FIG{fig:v0_systolic_pair_operation_4096_logtime}}
{\FIG{fig:v0_systolic_pair_operation_32768_logtime}}
{\pairoperation{}}
{\systolicloop{}}

%
%Q: Where and why does scaling stop for the Systolic Loop pair_operation method?
From
\FIG{fig:v0_systolic_pair_operation_512_logtime},
\FIG{fig:v0_systolic_pair_operation_4096_logtime} and
\FIG{fig:v0_systolic_pair_operation_32768_logtime}
it is clear that the system scales as $N/P$ when $P \ll{} N$.
%
However, it begins deviating significantly from linear scaling
when $N/P \approx{} 32$.
%
With a communication term scaling as $P$, and a calculation term
scaling as $N/P$, the communications term dominates far before $P = N$.
%
In each case, connunications approach close to pass calculations in time
when $N/P = 8$.

The implimentation ultimately becomes slowed due to
communication latency caused by a large number
of systolic pulses per time step.
%
When going to much larger numbers of processes
this implementation may suffer from extra slow down due to
global synchronisation, as it was implemented using a blocking
\mpisendrecv{}.
%
A simple remedy may be to use nonblocking communications to allow
processes to receive particles and perform partial updates while
allowing send lists to be sent in the background.
