\subsection{Shared and Replicated Data}

The \sharedandreplicateddata{} scheme is implemented in exactly the same
way as the \replicateddata{} scheme outlined in 
\SEC{sec:replicated_data_implementation},
except the update loop in the \pairoperation{} method is further
parallelised using \openmp{} directives, taking advantage of shared
memory between cores.
%
In fact, this distribution class inherits directly from the
\replicateddata{} distribution class, and overloads only the
\pairoperation{} method.


The primary motivation for this is to show that mixed mode MPI and \openmp{}
paralellism is just as viable as MPI only parallelism for a \replicateddata{}
scheme along with how easily the mixed mode parallelism may be implemented
on top of an MPI implementation.
%
The importance of this is that the maximum system size per core per node
can be increased in proportion to the number of \openmp{} threads
created per MPI process, which is of particular importance for nodes with
particularly high core counts.

%
%Q: What data does the Shared And Replicated Data implementation use?
The \sharedandreplicateddata{} distribution is initialised by
allocating a list of particles the size of the system of particles
on every MPI process.
%
The number of \openmp{} threads used per MPI processes is fixed at 8,
as this is the suggested number for \hector{} due to the arrangement
of the \numa{} regions into groups of 8.
%
As this is a rather low number compared to the overall number of MPI
processes, the emphasis here isn't to gain any perticular performance
improvement using \openmp{} (although some is expected due to reduced
MPI communications),
only to show that it is viable.
%
Indeed, the mere introduction of 8 threads per MPI process should increase
the maximum system size that can be run 8 fold.

In this section,
the implementation details and performance of
the \individualoperation{} and \pairoperation{} methods
will be presented and analysed.


\subsubsection{\individualoperation{}}

The \individualoperation{} method is inherited directly from
the implementation outlined in
\SEC{sec:replicated_data_individual_operation_implementation}.


%
%Q: What is the speedup of the Shared And Replicated Data individual_operation method?

%
% Overall speedup plot
%
\begin{figure}[!h]
    \input{parallel_implementation/v1/shared_and_replicated.individual_operation.logspeedup.plt}
    \caption{
        \vZeroSpeedupCaption
            {\sharedandreplicateddata{}}
            {\individualoperation{}}
    }
    \label{fig:v1_shared_and_replicated_data_individual_operation_speedups}
\end{figure}


\vZeroSpeedupExplanation
    {\FIG{fig:v1_shared_and_replicated_data_individual_operation_speedups}}
    {\sharedandreplicateddata{}}
    {\individualoperation{}}

Similar to the speedup graph in
\FIG{fig:v0_replicated_data_individual_operation_speedups}
for the \individualoperation{} method of the \replicateddata{}
implementation, there is no speedup to be seen here.
%
The primary point of interest is that this graph begins at 8 cores
which is a result of having the number of \openmp{} cores per
MPI process fixed at 8.

%
%Q: What is the expected calculation time of the Shared And Replicated Data individual_operation method?
The time to complete the calculations is expected to scale as $\bigO{N}$,
similarly to the \replicateddata{} implementation.
%
%Q: What is the expected communication time of the Shared And Replicated Data individual_operation method?
This implementation likewise performs no MPI communications, implying a
$\bigO{1}$ time.
%
%Q: What is the expected overall time of the Shared And Replicated Data individual_operation method?
As before, the overall time to completion is expected to be $\bigO{N}$.

%
% Individual breakdowns
%
\begin{figure}[!h]
    \input{parallel_implementation/v1/shared_and_replicated.individual_operation.512.time.plt}
    \caption{\vOneSRTimeCaption{\sharedandreplicateddata{}}{\individualoperation{}}{512}}
    \label{fig:v1_shared_and_replicated_individual_operation_512_time}
\end  {figure}

\begin{figure}[!h]
    \input{parallel_implementation/v1/shared_and_replicated.individual_operation.4096.time.plt}
    \caption{\vOneSRTimeCaption{\sharedandreplicateddata{}}{\individualoperation{}}{4096}}
    \label{fig:v1_shared_and_replicated_individual_operation_4096_time}
\end  {figure}

\begin{figure}[!h]
    \input{parallel_implementation/v1/shared_and_replicated.individual_operation.32768.time.plt}
    \caption{\vOneSRTimeCaption{\sharedandreplicateddata{}}{\individualoperation{}}{32768}}
    \label{fig:v1_shared_and_replicated_individual_operation_32768_time}
\end  {figure}

\vOneSRTimeExplanation
    {\FIG{fig:v0_replicated_individual_operation_512_time}}
    {\FIG{fig:v0_replicated_individual_operation_4096_time}}
    {\FIG{fig:v0_replicated_individual_operation_32768_time}}
    {\individualoperation{}}
    {\replicateddata{}}


%
%Q: Where and why does scaling stop for the Shared And Replicated Data individual_operation method?
As can be seen from 
\FIG{fig:v0_replicated_individual_operation_512_time},
\FIG{fig:v0_replicated_individual_operation_4096_time} and
\FIG{fig:v0_replicated_individual_operation_32768_time}
the performace scaling does indeed scale as $\bigO{N}$, as expected
and exactly in line with the performance scaling results of the
\individualoperation{} method in the \replicateddata{} scheme.

An opportunity exists to use the \openmp{} threads available to
parallelise this particular method with little synchronisation
and communications overhead as would be present with MPI
parallelisation of this method.
%
However, this particular optimisation was left unimplemented as
the run time of this method tends to be an order of magnitude
lower than the \pairoperation{} method regardless of the
number of cores used, so any optimisation
would be insignificant to the overall run time.
%
The extra development overhead and complexity would not be justified by the
overall performance improvement to an MD loop.



\subsubsection{\pairoperation{}}

The \pairoperation{} method is implemented in a similar manner to the
\pairoperation{} method from the \replicateddata{} scheme with the
addition of \openmp{} directives to further parallelise the
force update loop.

%
%Q: What is the speedup of the Shared And Replicated Data pair_operation method?

%
% Overall speedup plot
%
\begin{figure}[!h]
    \input{parallel_implementation/v1/shared_and_replicated.pair_operation.logspeedup.plt}
    \caption{
        \vZeroSpeedupCaption
            {\sharedandreplicateddata{}}
            {\pairoperation{}}
    }
    \label{fig:v1_shared_and_replicated_data_pair_operation_speedups}
\end{figure}


\vZeroSpeedupExplanation
    {\FIG{fig:v1_shared_and_replicated_data_pair_operation_speedups}}
    {\sharedandreplicateddata{}}
    {\pairoperation{}}

The speedup displayed by 
\FIG{fig:v1_shared_and_replicated_data_pair_operation_speedups}
is very similar to the speedup shown for the \pairoperation{}
method of the \replicateddata{} scheme in
\FIG{fig:v0_replicated_data_pair_operation_speedups}.
%
The speedup is marginally better for higher core counts, and in
particular on larger systems of particles.
%
There is some unexpected instability in the speedup for $N=512$ particles.


%
%Q: What is the expected calculation time of the Shared And Replicated Data pair_operation method?
A copy of the full system of particles is held on each MPI process,
resulting in $P_{MPI}$ replicas of the system being created.
%
Each MPI process is then assigned $N/P_{MPI}$ particles in that system
to determine forces for.
%
Given that list of $N/P_{MPI}$ particles,
an MPI process spawns $P_{OMP}$ threads
and assigns $1/(P_{MPI} P_{OMP})$ particles to each thread.

Writing
\begin{equation}
    \label{eqn:p_eq_pmpi_pomp}
    P = P_{MPI} \times{} P_{OMP}
\end{equation}
each core therefore has $N/P$ particles
for which it must determine the forces.
%
As before, if each particles must be compared to $N$ other particles
using an $\bigO{1}$ update operation, the time to find the force for
a single particles is
\begin{equation}
    N\bigO{1} = \bigO{N}
\end{equation}
And so, the time to find the forces for $N/P$ particles is
\begin{equation}
    \label{eqn:shared_and_replicated_calculation_time}
    \frac{N}{P}\bigO{N} = \bigO{\frac{N^2}{P}}
\end{equation}

%
%Q: What is the expected communication time of the Shared And Replicated Data pair_operation method?
After the forces have been determined by each thread, the team of threads
shuts down.
%
This represents a synchronisation point for the threads within the
MPI process.
%
After this \openmp{} synchronisation point,
an MPI synchronisation point is introduced in the form of
an MPI\_Allgatherv over the $P_{MPI}$ processes
to synchronise the updated lists of particles.
%
Assuming the MPI\_Allgatherv uses a tree-like algorithm for
gathering and distributing the data, this should perform
$\log{P_{MPI}}$ communications of approximately $N$ particles
with a constant latency $l$, resulting in a communication time
\begin{align}
    \log{(P_{MPI})}\,\bigO{N + l} 
        &= \bigO{\log{(P_{MPI})}\,(N+l)} \\
        &\approx{} \bigO{N\log{P_{MPI}}}
        \label{eqn:shared_andreplicated_communication_time}
\end{align}

Where communications in the \replicateddata{} scheme grew as $\log{P}$,
here they grow as $\log{P_{MPI}}$.
%
Using the definition of $P$ in \EQN{eqn:p_eq_pmpi_pomp} for the
\sharedandreplicateddata{} scheme, it can be seen that
\begin{equation}
    \log{P} = \log{P_{MPI}P_{OMP}} = \log{P_{MPI}} + \log{P_{OMP}}
\end{equation}
%
As a result, using $P_{OMP}$ threads per MPI process should provide
a drop in communication time of a constant $\log{P_{OMP}}$ compared
to the \replicateddata{} time.
%
Given the largest number of MPI processes used here is $32768$,
and $\log{32768} = 15$, the resulting drop in communication
time of $\log{P_{OMP}} = \log{8} = 3$ should result in a small
but noticeable improvement.


%
%Q: What is the expected overall time of the Shared And Replicated Data pair_operation method?
Combining
\EQN{eqn:shared_and_replicated_calculation_time} and
\EQN{eqn:shared_andreplicated_communication_time},
the overall execution time is expected to be
\begin{equation}
    \label{eqn:shared_and_replicated_data_overall_time}
    \bigO{\frac{N^2}{P}} + \bigO{N\log{P_{MPI}}}
        = \bigO{\frac{N^2}{P_{MPI} \times{} P_{OMP}} + N\log{P_{MPI}}}
\end{equation}


%
% Individual breakdowns
%
\begin{figure}[!h]
    \input{parallel_implementation/v1/shared_and_replicated.pair_operation.512.logtime.plt}
    \caption{\vOneSRTimeCaption{\sharedandreplicateddata{}}{\pairoperation{}}{512}}
    \label{fig:v1_shared_and_replicated_pair_operation_512_logtime}
\end  {figure}

\begin{figure}[!h]
    \input{parallel_implementation/v1/shared_and_replicated.pair_operation.4096.logtime.plt}
    \caption{\vOneSRTimeCaption{\sharedandreplicateddata{}}{\pairoperation{}}{4096}}
    \label{fig:v1_shared_and_replicated_pair_operation_4096_logtime}
\end  {figure}

\begin{figure}[!h]
    \input{parallel_implementation/v1/shared_and_replicated.pair_operation.32768.logtime.plt}
    \caption{\vOneSRTimeCaption{\sharedandreplicateddata{}}{\pairoperation{}}{32768}}
    \label{fig:v1_shared_and_replicated_pair_operation_32768_logtime}
\end  {figure}

\vOneSRTimeExplanation
    {\FIG{fig:v1_shared_and_replicated_pair_operation_512_logtime}}
    {\FIG{fig:v1_shared_and_replicated_pair_operation_4096_logtime}}
    {\FIG{fig:v1_shared_and_replicated_pair_operation_32768_logtime}}
    {\pairoperation{}}
    {\sharedandreplicateddata{}}


%
%Q: Where and why does scaling stop for the Shared And Replicated Data pair_operation method?
From
\FIG{fig:v1_shared_and_replicated_pair_operation_512_logtime},
\FIG{fig:v1_shared_and_replicated_pair_operation_4096_logtime} and
\FIG{fig:v1_shared_and_replicated_pair_operation_32768_logtime},
it appears the communication times approach and pass the calculation
times around $P = N/4$ or $P = N/2$.
%
This is in a similar place to
\FIG{fig:v0_replicated_pair_operation_512_logtime},
\FIG{fig:v0_replicated_pair_operation_4096_logtime} and
\FIG{fig:v0_replicated_pair_operation_32768_logtime}
representing this method for the \replicateddata{} implementaiton.
%
However, the communications tend to pass slightly later and at a shallower
angle here than in the \replicateddata{} case.
%
This results in a very slightly improved speedup.

What is worth noting, however, is that this implementation performs
at least as well as the \replicateddata{} implementation, required
the inclusion of just one default \openmp{} directive and
is now capable of holding an order of magnitude more particles on
a compute node.
%
The scaling results here suggest that with more cores available to the
implementation, running system sizes at the limit of what can fit on
a node may be feasible to run.
