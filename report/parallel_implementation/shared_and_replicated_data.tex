\section{Shared and Replicated Data}

The \sharedandreplicateddata{} scheme is implemented in exactly the same
way as the \replicateddata{} scheme outlined in 
\SEC{sec:replicated_data_implementation},
except the update loop in the \pairoperation{} method is further
parallelised using \openmp{} directives, taking advantage of shared
memory between cores.
%
In fact, this distribution class inherits directly from the
\replicateddata{} distribution class, and overloads only the
\pairoperation{} method.


The primary motivation for this is to show that mixed mode MPI and \openmp{}
paralellism is just as viable as MPI only parallelism for a \replicateddata{}
scheme along with how easily the mixed mode parallelism may be implemented
on top of an MPI implementation.
%
The importance of this is that the maximum system size per core per node
can be increased in proportion to the number of \openmp{} threads
created per MPI process, which is of particular importance for nodes with
particularly high core counts.

The \sharedandreplicateddata{} distribution is initialised by
allocating a list of particles the size of the system of particles
on every MPI process.
%
The number of \openmp{} threads used per MPI processes is fixed at 8,
as this is the suggested number for \hector{} due to the arrangement
of the \numa{} regions into groups of 8.
%
As this is a rather low number compared to the overall number of MPI
processes, the emphasis here isn't to gain any perticular performance
improvement using \openmp{}, only to show that it is viable.
%
Indeed, the mere introduction of 8 threads per MPI process should increase
the maximum system size that can be run 8 fold.

In this section,
the implementation details and performance of
the \individualoperation{} and \pairoperation{} methods
will be presented and analysed.


\subsection{\individualoperation{}}

The \individualoperation{} method is inherited directly from
the implementation outlined in
\SEC{sec:replicated_data_individual_operation_implementation}.
%
As a result, the time to completion is also expected to scale as $\bigO{N}$.

%
% Overall speedup plot
%
\begin{figure}[!h]
    \input{parallel_implementation/v1/shared_and_replicated.individual_operation.logspeedup.plt}
    \caption{
        Speedup plots for the \individualoperation{} implemented with the \sharedandreplicateddata{} scheme for systems of particles of size 512, 4096 and 32768.
    }
    \label{fig:v1_shared_and_replicated_data_individual_operation_speedups}
\end{figure}


%
% Individual breakdowns
%
\begin{figure}[!h]
    \input{parallel_implementation/v1/shared_and_replicated.individual_operation.512.time.plt}
    \caption{\vZeroTimeCaption{\sharedandreplicateddata{}}{\individualoperation{}}{512}}
    \label{fig:v1_shared_and_replicated_individual_operation_512_time}
\end  {figure}

\begin{figure}[!h]
    \input{parallel_implementation/v1/shared_and_replicated.individual_operation.4096.time.plt}
    \caption{\vZeroTimeCaption{\sharedandreplicateddata{}}{\individualoperation{}}{4096}}
    \label{fig:v1_shared_and_replicated_individual_operation_4096_time}
\end  {figure}

\begin{figure}[!h]
    \input{parallel_implementation/v1/shared_and_replicated.individual_operation.32768.time.plt}
    \caption{\vZeroTimeCaption{\sharedandreplicateddata{}}{\individualoperation{}}{32768}}
    \label{fig:v1_shared_and_replicated_individual_operation_32768_time}
\end  {figure}

\vZeroTimeExplanation
    {\FIG{fig:v0_replicated_individual_operation_512_time}}
    {\FIG{fig:v0_replicated_individual_operation_4096_time}}
    {\FIG{fig:v0_replicated_individual_operation_32768_time}}
    {\individualoperation{}}
    {\replicateddata{}}

As can be seen from 
\FIG{fig:v0_replicated_individual_operation_512_time},
\FIG{fig:v0_replicated_individual_operation_4096_time} and
\FIG{fig:v0_replicated_individual_operation_32768_time}
the performace scaling does indeed scale as $\bigO{N}$, as expected
and exactly in line with the performance scaling results of the
\individualoperation{} method in the \replicateddata{} scheme.



\subsection{\pairoperation{}}

The \pairoperation{} method is implemented in a similar manner to the
\pairoperation{} method from the \replicateddata{} scheme with the
addition of \openmp{} directives to further parallelise the
force update loop.

A copy of the system is held on each MPI process, meaning there
are $P_{MPI}$ replicas of the system created.
%
Each MPI process is then assigned $N/P_{MPI}$ particles in that system
to determine forces for.
%
Given that list of $N/P_{MPI}$ particles,
an MPI process spawns $P_{OMP}$ threads
and assigns $1/(P_{MPI} P_{OMP})$ particles to each thread.

Writing $P = P_{MPI} \times{} P_{OMP}$,
each core therefore has $N/P$ particles
for which it must determine the forces.
%
As before, if each particles must be compared to $N$ other particles
using an $\bigO{1}$ update operation, the time to find the force for
a single particles is
\begin{equation}
    N\bigO{1} = \bigO{N}
\end{equation}
And so, the time to find the forces for $N/P$ particles is
\begin{equation}
    \frac{N}{P}\bigO{N} = \bigO{\frac{N^2}{P}}
\end{equation}

After the forces have been determined by each thread, the team of threads
shuts down.
%
This represents a synchronisation point for the threads within the
MPI process.
%
After this, an MPI\_Allgatherv is used over the $P_{MPI}$ processes
to synchronise the updated lists of particles, representing a
$\bigO{N\log{P_{OMP}}}$ operation.

%
% Overall speedup plot
%
\begin{figure}[!h]
    \input{parallel_implementation/v1/shared_and_replicated.pair_operation.logspeedup.plt}
    \caption{
        Speedup plots for the \pairoperation{} implemented with the \sharedandreplicateddata{} scheme for systems of particles of size 512, 4096 and 32768.
    }
    \label{fig:v1_shared_and_replicated_data_pair_operation_speedups}
\end{figure}


%
% Individual breakdowns
%
\begin{figure}[!h]
    \input{parallel_implementation/v1/shared_and_replicated.pair_operation.512.logtime.plt}
    \caption{\vZeroTimeCaption{\sharedandreplicateddata{}}{\pairoperation{}}{512}}
    \label{fig:v1_shared_and_replicated_pair_operation_512_logtime}
\end  {figure}

\begin{figure}[!h]
    \input{parallel_implementation/v1/shared_and_replicated.pair_operation.4096.logtime.plt}
    \caption{\vZeroTimeCaption{\sharedandreplicateddata{}}{\pairoperation{}}{4096}}
    \label{fig:v1_shared_and_replicated_pair_operation_4096_logtime}
\end  {figure}

\begin{figure}[!h]
    \input{parallel_implementation/v1/shared_and_replicated.pair_operation.32768.logtime.plt}
    \caption{\vZeroTimeCaption{\sharedandreplicateddata{}}{\pairoperation{}}{32768}}
    \label{fig:v1_shared_and_replicated_pair_operation_32768_logtime}
\end  {figure}
