%
% Molecular Dynamics
%
\subsection{Molecular Dynamics}

Here, we consider only classical MD simulations.
%
A classical MD simulation is performed by
solving the Newtonian equations of motion for
a molecular system.
%
As there are typically more than 3 bodies interacting in the simulation,
it is not possible to solve these equations analytically.
%
The system must be solved numerically.
%
The numerical solutions to the equations of motion
typically integrate over a timestep $h$.
%
To simulate the system for a time $t$,
one must perform $t/h$ steps.
%
An initial attempt at solving the equations of motion yield
a set of equations requiring $\bigO{N^2}$ steps to solve
for each time step.
%
Some approximations may lead to equations that may be
solved in $\bigO{N}$ steps,
at the cost of reduced accuracy.

An MD simulation typically consists of three parts:
a particle distribution scheme,
a force finding scheme and
a numerical integration scheme.
%
The particle distribution and force finding schemes are
the parts we will be most interested in here.
%
In particular, the potential particle distribution schemes possible
tend to be limited by the force finding scheme in use.
%
The particle distribution scheme also dictates
how the force update scheme may be parallelized.
%
In this study, we use the velocity Verlet integration scheme.

An MD simulation typically consists of three phases:
setting up the particle distribution,
simulating the system until it comes to thermodynamic equilibrium and
taking measurements of the system.

The timescale at which atomic forces manifest compared to
the timescale at which properties of a molecular system may emerge
requires an MD simulation takes very many very short time steps.
%
This in itself introduces a significant amount of challenge to computations.
%
As a classical molecular system can be considered a chaotic one,
we see that any significant error introduced as a result of
increasing time steps may result in
vastly different results in a simulation.
%
Therefore, great care must be taken when choosing time steps, and
indeed a numerical integrator,
that it does not introduce so much error to a simulation as to render it
an unreliable source of information.
%
Some schemes attempt to address this issue by
using different time steps for phenomenon that occur on
different time scales within the same simulation.
%
However, these won’t be discussed here.
%
Indeed, beyond ensuring our numerical integrator and time steps provide
reasonable results for our simulation,
they will be outside the scope of this discussion.


The next area of interest is
the calculation of intermolecular forces that happens on each step.
%
An initial look at the calculation of intermolecular forces suggests
a problem whose solution time scales as $\bigO{N^2}$,
where $N$ is the number of particles in the system.
%
In this dissertation, we will be interested in
tackling the calculation of these forces.
%
Approximations exist providing solutions that scale as $\bigO{N}$.
%
Here, however, we are interested primarily in
studying the direct solutions to these equations of motions,
and the study of parallel schemes which calculate them.


\subsubsection{Potentials}

In an MD simulation, each particle will experience
a potential as a result of its position relative to every other particle.
%
The gradient of that potential then determines
the force felt by that particle.
%
In a classical MD simulation, we choose an approximation of a real potential.
This can be represented as a sum of potentials
which take account of interactions between increasing numbers of particles
\begin{align}
    V =&   \sum_{i=1}^N \sum_{j=1}^N V_2(\vec{x}_i, \vec{x}_j)
    \\ & + \sum_{i=1}^N \sum_{j=1}^N \sum_{k=1}^N
            V_3(\vec{x}_i, \vec{x}_j, \vec{x}_k)
    \\ & + \sum_{i=1}^N \sum_{j=1}^N \sum_{k=1}^N \sum_{l=1}^N
            V_3(\vec{x}_i, \vec{x}_j, \vec{x}_k, \vec{x}_l)
    \\ & + ...
\end  {align}
where $V$ is the potential,
$N$ is the number of particles,
$x_i$ is the position of particle $i$ and
$V_n$ is the $n$ body potential.
%
Often, the contribution of the $V_{n+1}$ term will be
much smaller than the $V_{n}$ term.
%
As a result, the potential may be approximated to the first few terms.


\begin{figure}
    \input{background/lennard_jones_potential.plt}
    \caption{The Lennard-Jones Potential with $\epsilon = 1$ and $\delta = 1$}
\end  {figure}
%
In this dissertation, we will consider only the two body term.
For this, we will use the Lennard-Jones potential
\begin{equation}
    V_{LJ}(r) = 4\epsilon \left[
        \left( \frac{\delta}{r} \right)^{12}
        - \left( \frac{\delta}{r} \right)^{6}
    \right]
\end  {equation}
where $V_{LJ}(r)$ is the Lennard-Jones potential as
a function of the distance $r$,
$r$ is the distance between two bodies,
$\epsilon$ is the depth of the potential well and
$\delta$ is the distance at which
the potential switches from being attractive to repulsive.
Typically, $\epsilon$ and $\delta$ are determined for a system by
fitting them to experimental data.
%
Due to the simplicity of this potential,
it is well studied and there exists good data on
reasonable values for $\epsilon$ and $\delta$ for
different molecular systems.

The Lennard-Jones potential is a short range potential.
%
That is, after a certain distance,
the potential rapidly tends towards zero.
%
This is useful for force evaluation methods that
ignore interactions between distant bodies to
reduce the number of calculations performed.
%
The distance at which the potential may be cut off is important
as it sets a lower limit on how many
particles must be compared in a molecular system.


\subsubsection{Limitations}

Classical MD simulations do not account for quantum effects,
and as such make some false predictions.
%
For example, A quantum approach would yield
a different distribution of
molecular vibrational frequencies to a classical approach.
%
Typically, a classical simulation will yield incorrect results
where dealing with high speeds or low temperatures.
%
This limits the areas where it may be used for study.

There are several potential issues regarding accuracy in MD simulations.
%
An $N$ body system is a chaotic one.
%
As a result, a small changes in accuracy may lead to
largely different results.
%
Errors may be introduced by
the numerical integration scheme used,
the time step used and
approximations of the forces used.
%
While a final answer may not be accurate,
the simulation may still be useful.
%
The system may be used to derive information on
thermodynamic variables or to simulate general trends.


MD calculations tend to be computationally intense.
%
This intensity derives both from
the number of time steps required to derive useful data and
the number of operations required to evaluate the force term on each step.
%
As a result, the sizes of the largest systems studied today
tends to be in the millions.
%
While this number varies,
and different methods of evaluating forces yield
different scaling factors,
there still appears to be an upper limit on
the number of particles that can currently be simulated.
