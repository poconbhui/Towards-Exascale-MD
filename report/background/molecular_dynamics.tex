% introduction/molecular_dynamics.tex
%
% Molecular Dynamics
%
\subsection{Molecular Dynamics}

% General information

Here, we consider only classical MD simulations.
%
A classical MD simulation is performed by
solving the Newtonian equations of motion for a molecular system.
%
As there are typically more than 3 bodies interacting in the simulation,
it is not possible to solve these equations analytically.
%
The system must be solved numerically.
%
The numerical solutions to the equations of motion
typically integrate over a timestep $h$.
%
To simulate the system for a time $t$, one must perform $t/h$ steps.
%
An initial attempt at solving the equations of motion yield a set of equations
requiring $\bigO{N^2}$ steps to solve for each time step.
%
Some approximations may lead to equations that may be solved in $\bigO{N}$
steps, at the cost of reduced accuracy.


% Pieces of an MD code

An MD simulation typically consists of three parts:
a particle distribution scheme,
a force finding scheme and
a numerical integration scheme.
%
The particle distribution and force finding schemes are
the parts we will be most interested in here.
%
In particular, the potential particle distribution schemes possible
tend to be limited by the force finding scheme in use.
%
The particle distribution scheme also dictates how the force update scheme
may be parallelized.
%
In this study, we use the velocity Verlet integration scheme.

An MD simulation typically consists of three phases:
setting up the particle distribution,
simulating the system until it comes to thermodynamic equilibrium and
taking measurements of the system.


%
% Potentials
%
\subsubsection{Potentials}

In an MD simulation, each particle will experience a potential as a result
of its position relative to every other particle.
%
The gradient of that potential will then determine the force felt by
that particle.
%
In a classical MD simulation, we choose an approximation of a real
potential.
%
In this dissertation, we will consider only the two body Lennard-Jones potential
\begin{equation}
    V_{LJ}(r) = 4\epsilon{} \left[
        \left( \frac{\delta}{r} \right)^{12}
        - \left( \frac{\delta}{r} \right)^{6}
    \right]
\end  {equation}
where $V_{LJ}(r)$ is the Lennard-Jones potential as
a function of the distance $r$,
$r$ is the distance,
$\epsilon$ is the depth of the potential well and
$\delta$ is the distance at which the potential switches from being
attractive to repulsive.
%
Typically, $\epsilon$ and $\delta$ are determined for a system
by fitting them to experimental data.

\begin{figure}
    \input{background/lennard_jones_potential.plt.tex}
    \caption{The Lennard-Jones Potential with $\epsilon=1$ and $\delta=1$}
\end  {figure}




%
% Limitations
%
\subsubsection{Limitations}

Classical MD simulations do not account for quantum effects,
and as such make some false predictions.
%
For example, A quantum approach would yield
a different distribution of molecular vibrational frequencies
to a classical approach.
%
Typically, a classical simulation will yield incorrect results where
dealing with high speeds or low temperatures.
%
This limits the areas where it may be used for study.


% Some caveats and limitations

There are several potential issues regarding
accuracy in MD simulations.
%
An N body system is a chaotic one.
%
As a result, a small changes in accuracy may lead to largely different results.
%
Errors may be introduced by the numerical integration scheme used,
the time step used and
approximations of the forces used.
%
While a final answer may not be accurate, the simulation may still be useful.
%
The system may be used to derive information on thermodynamic variables
or to simulate general trends.


MD calculations tend to be computationally intense.
%
This intensity derives both from
the number of time steps required to derive useful data and
the time required to evaluate the force term on each step.
%
As a result, the sizes of systems in use today tends to be in
the millions.
%
While this number varies, and different methods of evaluating forces
yield different scaling factors, there still appears to be an upper limit
on the number of particles that can currently be simulated.
