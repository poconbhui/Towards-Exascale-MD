%
% Molecular Dynamics
%
\subsection{Molecular Dynamics}


Here, we consider only classical MD simulations.
%
%Q: What exactly is an MD simulation?
A classical MD simulation is performed by
solving the Newtonian equations of motion for
a molecular system.
%
As there are typically more than 3 bodies interacting in the simulation,
it is not possible to solve these equations analytically.
%
The system must be solved numerically.


%
%Q: What bits do we need to make a simulation?
A parallel MD simulation typically consists of three parts:
a particle distribution scheme,
a force evaluation scheme and
a numerical integration scheme.
%
The particle distribution and force finding schemes are
the parts we will be most interested in here.
%
In this study, we use the velocity Verlet integration scheme.


%
%Q: What do we need to do with those bits to get data?
An MD simulation typically consists of three phases:
setting up the particle distribution,
simulating the system until it comes to thermodynamic equilibrium and
taking measurements of the system.
%
%Q: Why do we need the system to reach thermodynamic equilibrium?
The system must be allowed to reach thermodynamic equilibrium because
the initial distribution of particles may be a statistically unlikely
configuration.
%
If measurements of the system are taken before this stage, statistically
unlikely ensembles will be sampled more frequently than they should be when
performing measurements.
%
Measurements such as temperature and energy will not be representative
of the system being studied.



%
%Q: What physical constraints does an MD simulation conform to?
An MD system,
assuming homogeneity of time,
homogeneity of space and
isotropy of space
should conserve energy, linear momentum and angular momentum.
%
%Q: Can we use these to tell anything about our simulation?
Measurements of these values can be used to
determine the ``health'' of a simulation.
%
%Q: What does the conformance to these tell us about our simulation?
If any of these values change over the course of a simulation,
it is an indicator that either the system is unphysical or
significant numerical error is creeping into the simulation.
%
%Q: If it necessarily conforms to these, is it still a valid simulation?
However, simply because these symmetries are conserved does not mean
the simulation is producing a correct result.
%
Indeed, a simulation may proceed perfectly, but not accurately reflect
the system it supposes to model.




\subsubsection{Equations Of Motion}

%
%Q: What is a potential?
In a molecular system, each particle will generate a potential field
which permeates throughout the system.
%
A given particle will have a potential energy as a sum of these fields.
%
The gradient of that potential energy then determines
the force felt by that particle.
%
%Q: How do potentials differ for different systems?
The exact form of the potential depends on the system being studied.
%
Systems studied using Coulomb interactions, for example,
will have potentials that drop to zero at relatively long distances
from a particle.
%
Systems studied using Van der Waals interactions will have
potentials that drop to zero at relatively short distances from
a particle.
%
Several intermolecular and intramolecular forces may
calculated in a simulation to more accurately model underlying behaviours.


%
%Q: What are the equations of motion for a 2-body potential?
The Hamiltonian for a system of $N$ bodies
interacting under a 2-body potential can be written
\begin{equation}
    H = \sum_{i=1}^N \frac{\vec{p}_i^2}{2 m}
        + \sum_{i=1}^N \sum_{j<i}^N V(\vec{x}_i, \vec{x}_j)
\end  {equation}
where $H$ is the Hamiltonian,
$N$ is the number of bodies,
$m_i$ is the mass of body $i$,
$p_i$ is the momentum of body $i$,
$x_i$ is the position of body $i$ and
$V(\vec{x}_i, \vec{x}_j)$ is the interaction potential between two bodies.
This formula assumes a scalar potential and
that the potential is solely a function of distance.
%
This may be easily extended include terms involving charges,
but is ignored here.
%
We may then derive the equations of motion
\begin{equation}
    m_i \vec{a}_i = -\sum_{\substack{j=1\\j\ne{}i}}^N
                    \vec{\nabla}_i V(\vec{x}_i, \vec{x}_j)
\end  {equation}


%
%Q: What is the Velocity Verlet integration scheme?
The equations of motion may be discretized and solved numerically
by using an appropriate integration scheme.
The integration scheme used in this dissertation is
the Velocity Verlet scheme
\begin{align}
\label{eqn:velocity_verlet_scheme}
    \vec{v}_i(t + \tfrac{1}{2} h) &=
        \vec{v}_i(t) + \tfrac{1}{2}\vec{a}_i h
    \\
    \vec{x}_i(t + h) &=
        \vec{x}_i(t) + \vec{v}_i(t + \tfrac{1}{2} h) h
    \\
    m_i \vec{a}_i(t + h) &=
        - \sum_{\substack{j=1\\j\ne{}i}}^N
            \vec{\nabla}_i V(\vec{x}_i(t+h), \vec{x}_j(t+h))
    \\
    \vec{v}_i(t+h) &=
        \vec{v}_i(t + \tfrac{1}{2} h) + \tfrac{1}{2} \vec{a}_i(t + h) h
\end  {align}
where $t$ is the time,
$h$ is the step size,
$\vec{v}_i(t)$ is the velocity of body $i$ at time $t$,
$m_i$ is the mass of body $i$,
$\vec{a}_i(t)$ is the acceleration of body $i$ at time $t$,
$\vec{x}_i(t)$ is the position of body $i$ at time $t$ and
$V$ and $N$ are as described above.
%
%Q:What is the error in the VV scheme?
The local error for this scheme is $\bigO{h^4}$ while the local error
is $\bigO{h^3}$.
%
%Q: What is the time to solution for the VV scheme?
As can be seen, for a 2-body potential,
this produces an $\bigO{N^2}$ alforithm.

%
%Q: What time step do we need for the VV scheme?
The timescale at which atomic forces manifest compared to
the timescale at which properties of a molecular system may emerge
requires out numerical integrator use very short steps and be run for
a very large number of them.
%
This in itself introduces a significant amount of challenge to computations.
%
As a classical molecular system can be considered a chaotic one,
we see that any significant error introduced as a result of
increasing time steps may result in
vastly different results in a simulation.
%
Therefore, great care must be taken when choosing time steps, and
indeed a numerical integrator,
that it does not introduce so much error to a simulation as to render it
an unreliable source of information.
%
Some schemes attempt to address this issue by
using different time steps for phenomenon that occur on
different time scales within the same simulation.
%
However, these won’t be discussed here.
%
Indeed, beyond ensuring our numerical integrator and time steps provide
reasonable results for our simulation,
they will be outside the scope of this discussion.


%
%Q: What is the Lennard-Jones potential?
In this dissertation, we will consider the two body Lennard-Jones potential
\begin{equation}
    V_{LJ}(r) = 4\epsilon \left[
        \left( \frac{\delta}{r} \right)^{12}
        - \left( \frac{\delta}{r} \right)^{6}
    \right]
\end  {equation}
where $V_{LJ}(r)$ is the Lennard-Jones potential as
a function of the distance $r$,
$r$ is the distance between two bodies,
$\epsilon$ is the depth of the potential well and
$\delta$ is the distance at which
the potential switches from being attractive to repulsive.
Typically, $\epsilon$ and $\delta$ are determined for a system by
fitting them to experimental data.
%
%Q: What is the LJ potential useful for?
The Lennard-Jones is useful for modelling simple particle interactions.
%
Due to the simplicity of this potential,
it is well studied and there exists good data on
reasonable values for $\epsilon$ and $\delta$ for
different molecular systems.
%
\begin{figure}
    \input{background/lennard_jones_potential.plt}
    \caption{The Lennard-Jones Potential with $\epsilon = 1$ and $\delta = 1$}
\end  {figure}

%
%Q: What are the equations of motion for the LJ potential under VV?
Using the Lennard Jones potential, we can write
\begin{equation}
    V(\vec{x}_i(t), \vec{x}_j(t)) = V_{LJ}(|\vec{x}_i(t) - \vec{x}_j(t)|)
\end  {equation}
and writing $\vec{r}_{ij} = \vec{x}_j - \vec{x}_i$,
$r_{ij} = |\vec{r}_{ij}|$, $\hat{r}_{ij} = \frac{\vec{r}_{ij}}{r_{ij}}$
\begin{equation}
    \vec{\nabla}_i V(r_{ij}) = 4\epsilon \left[
        - \frac{12}{r_{ij}} \left( \frac{\delta}{r_{ij}} \right)^{12}
        + \frac{6}{r_{ij}} \left( \frac{\delta}{r_{ij}} \right)^{6}
    \right]
\end  {equation}
which may be used in \EQN{eqn:velocity_verlet_scheme} to describe
the discretized equations of motion of $N$ bodies interacting under the
Lennard-Jones potential.
%
%Q: How do these equations affect our time to solution?
One may imagine a situation where two bodies are distant enough
that the potential difference between them is negligible compared
to the contributions from closer bodies.
%
In this case, it may be possible to approximate this potential difference
to zero, and in fact, not calculate it.
%
By doing this, we may reduce the time complexity of our simulation.
%
This process is called truncation.


\subsubsection{Truncation of Forces}
%
%Q: What is the importance of the effective length of our potentials?
As can be seen from the form of the Lennard-Jones potential,
after a given distance from a particle, the potential tends towards
zero.
%
This presents the opportunity to consider only interactions between
``close'' particles.
%
This is achieved by setting a cutoff distance and evaluating potentials
between particles within this distance from each other.
%
Several schemes for tailoring a potential to a cutoff exist, and several
schemes for limiting calculations performed to ``close'' particles
exist.
%
The Verlet List scheme, for example, involves using some bookkeeping to
maintain lists of particles that are within the cutoff.
%
Typically, Verlet List implementations scale near $\bigO{N}$ with the
number of particles in the simulation, but $\bigO{N^2}$ in memory used.
%
Another example is a domain distributed parallel decomposition.
%
Particles that are physically close are placed on processes that are ``close''.
%
In this manner, processes may limit communications to other close processes.
%
Finally, direct solutions are performed between these close particles.

%
%Q: When we truncate our forces, what are the effects?
Overzealous truncation of forces may overly approximate the forces in play.
%
In particular, some molecular systems may be of interest, but the forces
are too long ranged to allow for simulations to be performed in
a reasonable timeframe.
% W%here is this ok and where is it not ok?

%
%Q: What happens if we have periodic systems?
Truncation may be used to effectively simulate very large
systems by implementing periodic boundary conditions, but ensuring
that no particle interacts with the same particle twice.
%
With this approach, one hopes to simulate what looks like an infinite system
without introducing periodic effects.
%
Of course, as we have done this by using periodic boundary conditions,
we can not completely remove periodic effects.
%
%Q: When we shrink our system sizes, what are the effects?
These periodic systems must be of a minimum size to avoid introducing
large finite size effects.
%
A system will often, however, experience some finite size effects
regardless of the system size simply because the system is not actually
infinite.

%
%Q: What happens if we have long range potentials?
Of course, we can't reasonably truncate our potentials if they are long ranged.
%
For non-periodic systems, we must directly solve the equations of motion.
%
Further, we must approach the problem from a very different angle if we
are using a periodic system.
%
Schemes, such as Ewald summation, exist to tackle the problem.
%
This will not be discussed here, but it is worth noting the limitations of
long ranged potentials and periodic systems as we are effectively using
truncation to describe short ranged potentials in periodic systems as
long ranged forces in many non-periodic systems.


\subsubsection{Approximations and Sensitivity}

Classical MD simulations do not account for quantum effects,
and as such make some false predictions.
%
For example, A quantum approach would yield
a different distribution of
molecular vibrational frequencies to a classical approach.
%
Typically, a classical simulation will yield incorrect results
where dealing with high speeds or low temperatures.
%
This limits the areas where it may be used for study.

There are several potential issues regarding accuracy in MD simulations.
%
An $N$ body system is a chaotic one.
%
As a result, a small change in accuracy may lead to
largely different results.
%
Errors may be introduced by
the numerical integration scheme used,
the time step used and
approximations of the forces used.
%
While a final answer may not be accurate,
the simulation may still be useful.
%
The system may be used to derive information on
thermodynamic variables or to simulate general trends.

Indeed, here, we seek to improve the time to solution for direct solutions
to allow us to increase the cutoff distance to two-body potentials.
%
By increasing this cutoff distance, we may slightly improve the accuracy
of our simulation.
%
However, given the system is chaotic, we will
still get an ``incorrect'' answer.
%
We hope only to reduce this particular source of error.
