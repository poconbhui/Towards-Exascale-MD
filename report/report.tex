% % % % % % % % % % % % % % % % % % % %
%                                     %
% Towards Exascale Molecular Dynamics %
%                                     %
% % % % % % % % % % % % % % % % % % % %
%          Padraig O Conbhui          %
% % % % % % % % % % % % % % % % % % % %

\documentclass[12pt,a4paper]{report}

\ifx\macrosHeader\undefined

\newcommand{\bigO}[1]{\mathcal{O}{(#1)}}
\newcommand{\subsubsubsection}[1]{{\bf #1} \vspace{0.5\baselineskip}}

\newcommand{\EQN}[1]{Eqn. (\ref{#1})}
\newcommand{\LST}[1]{Listing \ref{#1}}

\def\macrosHeader{0}
\fi

\ifx\packagesHeader\undefined

\usepackage{amsmath}

\usepackage{hyperref}

\usepackage{graphics}
\usepackage{graphicx}

\ifx\noepcc\undefined
    \usepackage{epcc}
\fi

\ifx\noxcolor\undefined
    \usepackage[usenames, svgnames]{xcolor}
\fi
\usepackage{listings}

\usepackage{enumitem}

\usepackage{tikz}
\usepackage{tikz-uml}
\usetikzlibrary{shapes, arrows}


\def\packagesHeader{0}
\fi

\ifx\configurationHeader\undefined

%
% Set author, title and date
%
\author{\padraigoconbhui{}}
\title{\towardsexascalemd{}}
\date{\today}

\hypersetup{
    pdfauthor={\plainpadraigoconbhui{}},
    pdftitle={\towardsexascalemd{}}
}

%
% Set up hyperref package
%
\hypersetup{
    hidelinks
}



\lstset{frame=single, language=Fortran}
\lstset{
    commentstyle=\color{Blue},
    keywordstyle=\bf \color{Green},
    numbers=left,
    stepnumber=1
}
\lstset{morecomment=[s][\color{LimeGreen}]{PURE}{\ }}
\lstset{morecomment=[s][\color{LimeGreen}]{&}{\ }}


\usepackage{amsmath}
\usepackage{parskip}

\usepackage[font=footnotesize]{caption}

\usepackage{epcc}


% Styles gratefully stolen from
% http://www.texample.net/tikz/examples/simple-flow-chart/
\tikzstyle{redcolor} = [red!80]
\tikzstyle{bluecolor} = [blue!20]
\tikzstyle{redfill} = [fill=red!80]
\tikzstyle{bluefill} = [fill=blue!20]
\tikzstyle{decision} = [diamond, draw, bluefill, 
    text width=4.5em, text badly centered, node distance=3cm, inner sep=0pt]
\tikzstyle{block} = [rectangle, draw, bluefill, 
    text width=5em, text centered, rounded corners, minimum height=4em]
\tikzstyle{line} = [draw, -latex', rounded corners, thick]


\def\configurationHeader{0}
\fi


\begin{document}


%
% Begin roman numerals
%

\pagenumbering{roman}


%
% Make front page
%

\makeEPCCtitle

\thispagestyle{empty}

\vspace{12cm}

\begin{center}

\large{MSc in High Performance Computing}

\large{The University of Edinburgh}

\large{Year of Presentation: \the\year}

\end{center}

\newpage

%
% Empty page
%
\thispagestyle{empty}
\clearpage\mbox{}\clearpage


%
% Project Abstract
%

\begin{abstract}
Computer experiments are becoming increasingly important in
scientific research.
%
In fields such as material science and biophysics, computer models
of atomic and molecular systems present a rich source of
information.
%
Molecular Dynamics (MD) is one tool for generating these computer models.

There is a constant strive to simulate ever larger molecular systems
using MD.
%
Given a simple formulation of an atomic system suggests a
$\bigO{N^2}$ algorithm, increasing simulation sizes is not
a straightforward task.
%
To this end, many approximations of they physics involved exist that
can dramatically reduce the time complexity of these simulations.
%
However, these approximations represent potential sources of error
in the already chaotic equations of motion for these systems.

This work attempts to determine just how far parallelisation alone
can improve an MD application.
%
In particular, it strives to find where the bottlenecks exist for
four parallel schemes which integrate the equations of motion for
an MD simulation without approximations.
%
Two of these schemes are the familiar \replicateddata{} and
\systolicloop{} schemes.
%
The third is referred to in this work as the \sharedandreplicateddata{}
scheme, representing a hybrid replicated and shared memory scheme.
%
The fourth, presented here, is referred to as the \replicatedsystolicloop{},
which is based on the \systolicloop{} scheme, taking inspiration from
the \replicateddata{} scheme by using replica systolic loops to
reduce the number of pulses any one systolic loop performs.

The bottleneck in the \replicateddata{} scheme was found to be
the amount of data being transferred and the number of MPI processes
being communicated to.
%
This was addressed by the \sharedandreplicateddata{} scheme by
using shared memory to reduce the number of MPI communications
necesary.
%
This scheme also addressed some of the memory usage issues involved
with the \replicateddata{} scheme.

The bottleneck in the \systolicloop{} scheme was found to be the
number of pulses performed, as a result of communication latency.
%
This was addressed by the \replicatedsystolicloop{} scheme by
using replica loops to reduce the number of systolic pulses each
loop performed.

\end{abstract}


% Restart numbering
\pagenumbering{roman}


%
% Content list pages
%

\tableofcontents
\listoftables
\listoffigures


%
% Title page
%

\begin{titlepage}
\vspace*{2in}
% an acknowledgements section is completely optional but if you decide
% not to include it you should still include an empty {titlepage}
% environment as this initialises things like section and page numbering.

%
% Acknowledgements
%

\section*{Acknowledgements}

I would like to thank my supervisor, Dr. Antonia Collis, without whom
I would still be completely incapable of writing a reasonable scientific
document.
%
Her guidance throughout the project, steering me back on track when I
needed steering, and giving me the advice I needed when I needed it,
were instrumental to the existance of this document.

Thanks go out to friends and family for their love and support
over the last 12 months. 

I must also thank my parents, in particular,
for bankrolling me all the way here.



\end{titlepage}


% Start regular page numbering
\pagenumbering{arabic}


%
% Include chapters
%

\section{Introduction}

Parallel implementations are required to be functionally equivalent
to the example serial implementations outlined in \SEC{sec:methodology:subsec:implementation}.
%
Tests may then be implemented straightforwardly for the serial implementation,
and are then easily extended to the parallel implementations.

The prescribed interface discourages the use of optimisations which make
assumptions about the MD algorithm being implemented.
%
If it were known ahead of time, for example, that
only forces would be updated during a \pairoperation{},
as is the case in the \velocityverlet{} algorithm,
or that inter-atomic forces dropped to zero after a given distance,
the implementation may be able to use this information to improve
performance through specialized data layouts or communications patterns.

Optimizations for particular algorithms and MD systems are
beyond the scope of this dissertation.
%
The focus here is on general communication patterns rather than
the optimization of particular MD algorithms.

Implementations of the \replicateddata{} and \systolicloop{} schemes
using only MPI will be analysed and discussed by focusing on
the scaling results of the \individualoperation{} and \pairoperation{} methods.


\subsection{Replicated Data}

The replicated scheme allocates a list of particles the size of
the entire MD system of particles
on each process.
%
It uses this list to keep an up-to-date copy of the system of particles
on every process.

In this section, the implementation details and performance of
of the \individualoperation{} and \pairoperation{} methods
for the replicated data scheme using only MPI will be analysed and discussed.


%
% Replicated individual_operation v0
%

\subsubsection{Implementation of the \individualoperation{} Method}

The \individualoperation{} method as outlined in
\SEC{sec:the_individual_operation_method}
is implemented by having each process update its entire local list.
%
As such, it closely resembles the example serial implementation.
%
This approach involves more computation than having each process
evaluate a subsection of the list and share the result with the
other processes, but it avoids a global synchronization.
%
As each process performs an $\bigO{1}$ operation on $N$ particles with
no communications,
this implementation is expected to take a time
\begin{equation}
\label{eqn:v0_replicated_individual_operation_overall_time}
    N\bigO{1} = \bigO{N}
\end  {equation}

\begin{figure}[!h]
    \input{parallel_implementation/v0/replicated.individual_operation.512.time.plt}
    \caption{\vZeroTimeCaption{Replicated Data}{\individualoperation{}}{512}}
    \label{fig:v0_replicated_individual_operation_512_time}
\end  {figure}

\begin{figure}[!h]
    \input{parallel_implementation/v0/replicated.individual_operation.4096.time.plt}
    \caption{\vZeroTimeCaption{Replicated Data}{\individualoperation{}}{4096}}
    \label{fig:v0_replicated_individual_operation_4096_time}
\end  {figure}

\begin{figure}[!h]
    \input{parallel_implementation/v0/replicated.individual_operation.32768.time.plt}
    \caption{\vZeroTimeCaption{Replicated Data}{\individualoperation{}}{32768}}
    \label{fig:v0_replicated_individual_operation_32768_time}
\end  {figure}


\vZeroTimeExplanation
    { \FIG{fig:v0_replicated_individual_operation_512_time} }
    { \FIG{fig:v0_replicated_individual_operation_4096_time} }
    { \FIG{fig:v0_replicated_individual_operation_32768_time} }
    { \individualoperation{} }


From these, it is clear that
the time for the \individualoperation{} doesn't scale with the number
of cores and that MPI takes up no time as this method uses no MPI calls.

There is an interesting increase of time that occurs at 2 cores and again
at 8 cores.
%
Given that \hector{} has 4 NUMA regions of 8 cores and each of those
regions is further subdivided into 4 NUMA regions of 2 cores,
it is likely this is a cause for the jumps at 2 and 8 cores.
%
This is particularly noticeable in
\FIG{fig:v0_replicated_individual_operation_32768_time}
where the jump occurs at exactly the same core count, but is noticeably larger.

The jumps occur at the same numbers of processes in
\FIG{fig:v0_replicated_individual_operation_512_time},
\FIG{fig:v0_replicated_individual_operation_4096_time} and
\FIG{fig:v0_replicated_individual_operation_32768_time}
and after 8 processes, the time remains roughly constant.
%
This suggests either a latency effect with processes accessing memory
in other NUMA regions or a memory bandwith effect.
%
Given that the effect scales roughly with the number of particles
in the system, at fixed core counts, it is most likely due to
memory bandwidth saturation.
%
If bandwidth saturated, it is expected that the time taken for
data transfer would
scale linearly with the size of the data per core being transferred.
%
Therefore, it is likely that the bandwidth is not saturated in the 2 core NUMA
region when only 1 core is in use, and
similarly that the bandwidth for the 8 core NUMA region is not saturated
when only 7 cores are in use.

Above 8 cores, the overall execution time for
the \individualoperation{} method for the Replicated Data implementation
appears to scale as $\bigO{N}$
as predicted in \EQN{eqn:v0_replicated_individual_operation_overall_time}.


%
% Replicated pair_operation v0
%

\subsubsection{Implementation of the \pairoperation{} Method}

The \pairoperation{} method as outlined in
\SEC{sec:the_pair_operation_method}
is implemented by having each process evaluating the pair
comparisons for a subset of the particles and sharing the
results with the other processes.

Each process is designated a subset of the list of particles for which
it will evaluate the results of the comparison and reduction.
%
This looks similar to parallelising the outer loop of the example
serial implementation.
%
The number of particles each process is assigned is roughly $N/P$.
%
Each of these $N/P$ particles are compared to $N$ other particles
using an $\bigO{1}$ operation.
%
The time for one of the $N/P$ particles to be updated is
\begin{equation}
    N\bigO{1} = \bigO{N}
\end  {equation}
and so, the time for all $N/P$ particles to be updated is
\begin{equation}
    \frac{N}{P}\bigO{N} = \bigO{\frac{N^2}{P}}
\end  {equation}
As such, a calculation term of $\bigO{N^2/P}$ is expected.

After the each process finishes updating it's section of the list,
the updated sections of lists are shared across processes using
an MPI\_Allgatherv.
This introduces a communication term of $\bigO{(N + l)\log{P}}$
where $l$ represents a constant latency.

The overall execution time of this method should therefore be
\begin{align}
    \bigO{\frac{N^2}{P}} + \bigO{(N+l)\log{P}}
        &\approx{} \bigO{\frac{N^2}{P} + (N+l)\log{P}} \\
        &\approx{} \bigO{\frac{N^2}{P} + (N+l)\log{P}} \\
        &\approx{} \bigO{\frac{N^2}{P} + N\log{P} + \log{P}} \\
        \label{eqn:v0_replicated_pair_operation_overall_time}
        &\approx{} \bigO{\frac{N^2}{P} + N\log{P}}
\end  {align}

\begin{figure}[!h]
    \input{parallel_implementation/v0/replicated.pair_operation.512.logtime.plt}
    \caption{\vZeroTimeCaption{Replicated Data}{\pairoperation{}}{512}}
    \label{fig:v0_replicated_pair_operation_512_logtime}
\end  {figure}

\begin{figure}[!h]
    \input{parallel_implementation/v0/replicated.pair_operation.4096.logtime.plt}
    \caption{\vZeroTimeCaption{Replicated Data}{\pairoperation{}}{4096}}
    \label{fig:v0_replicated_pair_operation_4096_logtime}
\end  {figure}

\begin{figure}[!h]
    \input{parallel_implementation/v0/replicated.pair_operation.32768.logtime.plt}
    \caption{\vZeroTimeCaption{Replicated Data}{\pairoperation{}}{32768}}
    \label{fig:v0_replicated_pair_operation_32768_logtime}
\end  {figure}

\vZeroTimeExplanation
    { \FIG{fig:v0_replicated_pair_operation_512_logtime} }
    { \FIG{fig:v0_replicated_pair_operation_4096_logtime} }
    { \FIG{fig:v0_replicated_pair_operation_32768_logtime} }
    { \pairoperation{} }

In \FIG{fig:v0_replicated_pair_operation_512_logtime},
\FIG{fig:v0_replicated_pair_operation_4096_logtime} and
\FIG{fig:v0_replicated_pair_operation_32768_logtime},
it is apparent that scaling begins to drop off as the number
of processes used is comparable to the number of particles in the system.
%
From \EQN{eqn:v0_replicated_pair_operation_overall_time} when $P \ll{} N$
\begin{equation}
    \bigO{\frac{N^2}{P} + N\log{P}} \sim{} \bigO{\frac{N^2}{P}}
\end  {equation}
suggesting good scaling in this regime.
%
Similarly when $P \sim{} N$
or $P > N$
\begin{align}
    \bigO{\frac{N^2}{P} + N\log{P}}
        &\sim{} \bigO{N + N\log{P}} \\
        &\sim{} \bigO{N\log{P}}
\end  {align}
%
suggesting the communication term eventually dominating and
the overall time increasing as a function of $P$.

Indeed, examining where
\FIG{fig:v0_replicated_pair_operation_512_logtime},
\FIG{fig:v0_replicated_pair_operation_4096_logtime},
\FIG{fig:v0_replicated_pair_operation_32768_logtime} and
begin straying away from linear scaling,
it can be seen that the minimum execution time
of 512 particles is approximately $10^{-4}$ seconds,
of 4096 particles is approximately $10^{-3}$ seconds and
of 32768 particles is approximately $10^{-2}$ seconds.
%
Thus as scaling begins dropping off at $P = N$,
the minimum execution time for this distribution,
for the system sizes tested,
scales as $N\log{N}$
where $N$ is the number of particles in the system.


\subsection{Systolic Loop}

The systolic loop scheme is initialised by allocating 3 arrays of
size $N/P$ on every process.
%
The system is then split roughly evenly across all the processes
and is held, updated and shared using these three lists.


%
% Systolic individual_operation v0
%

\subsubsection{Implementation of the \individualoperation{} Method}
This is implemented by having each process update its local list
of particles.

This should take $\bigO{N/P}$ time.

\begin{figure}[!h]
    \input{parallel_implementation/v0/systolic.individual_operation.512.logtime.plt}
    \caption{\vZeroTimeCaption{Systolic Loop}{\individualoperation{}}{512}}
    \label{fig:v0_systolic_individual_operation_512_logtime}
\end  {figure}

\begin{figure}[!h]
    \input{parallel_implementation/v0/systolic.individual_operation.4096.logtime.plt}
    \caption{\vZeroTimeCaption{Systolic Loop}{\individualoperation{}}{4096}}
    \label{fig:v0_systolic_individual_operation_4096_logtime}
\end  {figure}

\begin{figure}[!h]
    \input{parallel_implementation/v0/systolic.individual_operation.32768.logtime.plt}
    \caption{\vZeroTimeCaption{Systolic Loop}{\individualoperation{}}{32768}}
    \label{fig:v0_systolic_individual_operation_32768_logtime}
\end  {figure}

As seen in
\FIG{fig:v0_systolic_individual_operation_512_logtime},
\FIG{fig:v0_systolic_individual_operation_4096_logtime} and
\FIG{fig:v0_systolic_individual_operation_32768_logtime}
the current implementation satisfies this.

There appear to be a few unexpected data points in the mpi only timings,
but these are unlikely to be anything except an error.
%
As these are approaching nanosecond execution times, it is
unsurprising that they may pick up unexpected errors as the system clock
has at most a nanosecond resolution time.
%
Given the processor operates on about this time, it is also unsurprising
if some odd times may be picked up from taking measurements so
close to this time scale.


%
% Systolic pair_operation v0
%

\subsubsection{Implementation of the \pairoperation{} Method}
This is implemented by having each process use three lists of particles,
each of size $P/N$.

The first list is the processes local list of particles.

The second list will be referred to as the foreign list, and
represents a list originating from another process.

The third list is a swap list to allow a process to receive a new
foreign list list from the right
while also sending its old foreign list to the left
during a systolic pulse.

Initially, a process will copy its local list to the foreign list
and perform a partial force update on its local list using this
foreign list.
%
The system will then perform a systolic pulse.
After a systolic pulse, every foreign list should move one process
to the left in the systolic loop.
%
This is performed by copying the old foreign list into the
swap list, and using an MPI\_sendrecv to send the swap list to
the left process while receiving from
the right process into the foreign list.
%
When a new foreign list is received, another partial force update
is performed on the local list.
%
This process is repeated $P-1$ times.

Each list comparison between systolic pulses should take $\bigO{(N/P)^2}$ time.
%
For a given timestep, there should be $P$ of these list comparisons
performed, giving an over calculation time of $\bigO{N^2/P}$.

Given each pulse should be passing $N/P$ particles between two processes,
we expect this to take $\bigO{N/P + l}$ time where $l$ is a constant latency.
%
With $P$ pulses on each time step, we expect a communication time of
$\bigO{(N/P + l)P}$.

Combining our calculation and communication terms, the systolic loop approach
should run in $\bigO{N^2/P + dN + dlP}$ time
where $N$ is the number of particles in the system,
$P$ is the number of processes used and
$l$ is a constant latency and
$d$ is a constant.

\begin{figure}[!h]
    \input{parallel_implementation/v0/systolic.pair_operation.512.logtime.plt}
    \caption{\vZeroTimeCaption{Systolic Loop}{\pairoperation{}}{512}}
    \label{fig:v0_systolic_pair_operation_512_logtime}
\end  {figure}

\begin{figure}[!h]
    \input{parallel_implementation/v0/systolic.pair_operation.4096.logtime.plt}
    \caption{\vZeroTimeCaption{Systolic Loop}{\pairoperation{}}{4096}}
    \label{fig:v0_systolic_pair_operation_4096_logtime}
\end  {figure}

\begin{figure}[!h]
    \input{parallel_implementation/v0/systolic.pair_operation.32768.logtime.plt}
    \caption{\vZeroTimeCaption{Systolic Loop}{\pairoperation{}}{32768}}
    \label{fig:v0_systolic_pair_operation_32768_logtime}
\end  {figure}

We see in 
\FIG{fig:v0_systolic_pair_operation_512_logtime},
\FIG{fig:v0_systolic_pair_operation_4096_logtime} and
\FIG{fig:v0_systolic_pair_operation_32768_logtime}
that, much like in the replicated case, that the system scales
roughly as $N/P$ when $P \ll{} N$.

With a communication term scaling as $P$, we see the point in which
communications dominate comes much sooner.
%
However, we also note that it appears when $P \approx{} N$.
%
Taking our previous approach of finding an optimum $k$ for $P = N/k$,
we find $k \approx{} 8$.

This appears to hold for our three system sizes, although there does appear
to be an unexpected jump in
\FIG{fig:v0_systolic_pair_operation_32768_logtime}
between 512 and 1024 processes.
%
It is unclear whether this is a genuine effect, or simply an error in
measurement.

Ignoring the unexpected jump in 
\FIG{fig:v0_systolic_pair_operation_32768_logtime},
we may conclude that the minimum time to completion for out system
scales linearly with the number of particles.

It would appear the implimentation ultimately becomes slowed due to
communication latency, and in particular, due to a large number
of these communications.
%
The initial scaling of this term is due to the number of particles
in the system, however,
scaling with $P$ is a result of having $P$ communications
per time step.
%
Looking at our derivation for communication times,
we conclude that this must be an effect of latency.

\section{Shared and Replicated Data}

The \sharedandreplicateddata{} scheme is implemented in exactly the same
way as the \replicateddata{} scheme outlined in 
\SEC{sec:replicated_data_implementation},
except the update loop in the \pairoperation{} method is further
parallelised using \openmp{} directives, taking advantage of shared
memory between cores.
%
In fact, this distribution class inherits directly from the
\replicateddata{} distribution class, and overloads only the
\pairoperation{} method.


The primary motivation for this is to show that mixed mode MPI and \openmp{}
paralellism is just as viable as MPI only parallelism for a \replicateddata{}
scheme along with how easily the mixed mode parallelism may be implemented
on top of an MPI implementation.
%
The importance of this is that the maximum system size per core per node
can be increased in proportion to the number of \openmp{} threads
created per MPI process, which is of particular importance for nodes with
particularly high core counts.

The \sharedandreplicateddata{} distribution is initialised by
allocating a list of particles the size of the system of particles
on every MPI process.
%
The number of \openmp{} threads used per MPI processes is fixed at 8,
as this is the suggested number for \hector{} due to the arrangement
of the \numa{} regions into groups of 8.
%
As this is a rather low number compared to the overall number of MPI
processes, the emphasis here isn't to gain any perticular performance
improvement using \openmp{}, only to show that it is viable.
%
Indeed, the mere introduction of 8 threads per MPI process should increase
the maximum system size that can be run 8 fold.

In this section,
the implementation details and performance of
the \individualoperation{} and \pairoperation{} methods
will be presented and analysed.


\subsection{\individualoperation{}}

The \individualoperation{} method is inherited directly from
the implementation outlined in
\SEC{sec:replicated_data_individual_operation_implementation}.
%
As a result, the time to completion is also expected to scale as $\bigO{N}$.

%
% Overall speedup plot
%
\begin{figure}[!h]
    \input{parallel_implementation/v1/shared_and_replicated.individual_operation.logspeedup.plt}
    \caption{
        Speedup plots for the \individualoperation{} implemented with the \sharedandreplicateddata{} scheme for systems of particles of size 512, 4096 and 32768.
    }
    \label{fig:v1_shared_and_replicated_data_individual_operation_speedups}
\end{figure}


%
% Individual breakdowns
%
\begin{figure}[!h]
    \input{parallel_implementation/v1/shared_and_replicated.individual_operation.512.time.plt}
    \caption{\vZeroTimeCaption{\sharedandreplicateddata{}}{\individualoperation{}}{512}}
    \label{fig:v1_shared_and_replicated_individual_operation_512_time}
\end  {figure}

\begin{figure}[!h]
    \input{parallel_implementation/v1/shared_and_replicated.individual_operation.4096.time.plt}
    \caption{\vZeroTimeCaption{\sharedandreplicateddata{}}{\individualoperation{}}{4096}}
    \label{fig:v1_shared_and_replicated_individual_operation_4096_time}
\end  {figure}

\begin{figure}[!h]
    \input{parallel_implementation/v1/shared_and_replicated.individual_operation.32768.time.plt}
    \caption{\vZeroTimeCaption{\sharedandreplicateddata{}}{\individualoperation{}}{32768}}
    \label{fig:v1_shared_and_replicated_individual_operation_32768_time}
\end  {figure}

\vZeroTimeExplanation
    {\FIG{fig:v0_replicated_individual_operation_512_time}}
    {\FIG{fig:v0_replicated_individual_operation_4096_time}}
    {\FIG{fig:v0_replicated_individual_operation_32768_time}}
    {\individualoperation{}}
    {\replicateddata{}}

As can be seen from 
\FIG{fig:v0_replicated_individual_operation_512_time},
\FIG{fig:v0_replicated_individual_operation_4096_time} and
\FIG{fig:v0_replicated_individual_operation_32768_time}
the performace scaling does indeed scale as $\bigO{N}$, as expected
and exactly in line with the performance scaling results of the
\individualoperation{} method in the \replicateddata{} scheme.



\subsection{\pairoperation{}}

The \pairoperation{} method is implemented in a similar manner to the
\pairoperation{} method from the \replicateddata{} scheme with the
addition of \openmp{} directives to further parallelise the
force update loop.

A copy of the system is held on each MPI process, meaning there
are $P_{MPI}$ replicas of the system created.
%
Each MPI process is then assigned $N/P_{MPI}$ particles in that system
to determine forces for.
%
Given that list of $N/P_{MPI}$ particles,
an MPI process spawns $P_{OMP}$ threads
and assigns $1/(P_{MPI} P_{OMP})$ particles to each thread.

Writing $P = P_{MPI} \times{} P_{OMP}$,
each core therefore has $N/P$ particles
for which it must determine the forces.
%
As before, if each particles must be compared to $N$ other particles
using an $\bigO{1}$ update operation, the time to find the force for
a single particles is
\begin{equation}
    N\bigO{1} = \bigO{N}
\end{equation}
And so, the time to find the forces for $N/P$ particles is
\begin{equation}
    \frac{N}{P}\bigO{N} = \bigO{\frac{N^2}{P}}
\end{equation}

After the forces have been determined by each thread, the team of threads
shuts down.
%
This represents a synchronisation point for the threads within the
MPI process.
%
After this, an MPI\_Allgatherv is used over the $P_{MPI}$ processes
to synchronise the updated lists of particles, representing a
$\bigO{N\log{P_{OMP}}}$ operation.

%
% Overall speedup plot
%
\begin{figure}[!h]
    \input{parallel_implementation/v1/shared_and_replicated.pair_operation.logspeedup.plt}
    \caption{
        Speedup plots for the \pairoperation{} implemented with the \sharedandreplicateddata{} scheme for systems of particles of size 512, 4096 and 32768.
    }
    \label{fig:v1_shared_and_replicated_data_pair_operation_speedups}
\end{figure}


%
% Individual breakdowns
%
\begin{figure}[!h]
    \input{parallel_implementation/v1/shared_and_replicated.pair_operation.512.logtime.plt}
    \caption{\vZeroTimeCaption{\sharedandreplicateddata{}}{\pairoperation{}}{512}}
    \label{fig:v1_shared_and_replicated_pair_operation_512_logtime}
\end  {figure}

\begin{figure}[!h]
    \input{parallel_implementation/v1/shared_and_replicated.pair_operation.4096.logtime.plt}
    \caption{\vZeroTimeCaption{\sharedandreplicateddata{}}{\pairoperation{}}{4096}}
    \label{fig:v1_shared_and_replicated_pair_operation_4096_logtime}
\end  {figure}

\begin{figure}[!h]
    \input{parallel_implementation/v1/shared_and_replicated.pair_operation.32768.logtime.plt}
    \caption{\vZeroTimeCaption{\sharedandreplicateddata{}}{\pairoperation{}}{32768}}
    \label{fig:v1_shared_and_replicated_pair_operation_32768_logtime}
\end  {figure}

\section{Shared and Replicated Data}

\subsection{\individualoperation{}}

%
% Overall speedup plot
%
\begin{figure}[!h]
    \input{parallel_implementation/v1/replicated_systolic.individual_operation.logspeedup.plt}
    \caption{
        Speedup plots for the \individualoperation{} implemented with the \replicatedsystolicloop{} scheme for systems of particles of size 512, 4096 and 32768.
    }
    \label{fig:v1_replicated_systolic_loop_individual_operation_speedups}
\end{figure}


%
% Individual breakdowns
%
\begin{figure}[!h]
    \input{parallel_implementation/v1/replicated_systolic.individual_operation.512.logtime.plt}
    \caption{\vZeroTimeCaption{\replicatedsystolicloop{}}{\individualoperation{}}{512}}
    \label{fig:v1_replicated_systolic_individual_operation_512_time}
\end  {figure}

\begin{figure}[!h]
    \input{parallel_implementation/v1/replicated_systolic.individual_operation.4096.logtime.plt}
    \caption{\vZeroTimeCaption{\replicatedsystolicloop{}}{\individualoperation{}}{4096}}
    \label{fig:v1_replicated_systolic_individual_operation_4096_time}
\end  {figure}

\begin{figure}[!h]
    \input{parallel_implementation/v1/replicated_systolic.individual_operation.32768.logtime.plt}
    \caption{\vZeroTimeCaption{\replicatedsystolicloop{}}{\individualoperation{}}{32768}}
    \label{fig:v1_replicated_systolic_individual_operation_32768_time}
\end  {figure}




\subsection{\pairoperation{}}

%
% Overall speedup plot
%
\begin{figure}[!h]
    \input{parallel_implementation/v1/replicated_systolic.pair_operation.logspeedup.plt}
    \caption{
        Speedup plots for the \pairoperation{} implemented with the \replicatedsystolicloop{} scheme for systems of particles of size 512, 4096 and 32768.
    }
    \label{fig:v1_replicated_systolic_pair_operation_speedups}
\end{figure}


%
% Individual breakdowns
%
\begin{figure}[!h]
    \input{parallel_implementation/v1/replicated_systolic.pair_operation.512.logtime.plt}
    \caption{\vZeroTimeCaption{\replicatedsystolicloop{}}{\pairoperation{}}{512}}
    \label{fig:v1_replicated_systolic_pair_operation_512_logtime}
\end  {figure}

\begin{figure}[!h]
    \input{parallel_implementation/v1/replicated_systolic.pair_operation.4096.logtime.plt}
    \caption{\vZeroTimeCaption{\replicatedsystolicloop{}}{\pairoperation{}}{4096}}
    \label{fig:v1_replicated_systolic_pair_operation_4096_logtime}
\end  {figure}

\begin{figure}[!h]
    \input{parallel_implementation/v1/replicated_systolic.pair_operation.32768.logtime.plt}
    \caption{\vZeroTimeCaption{\replicatedsystolicloop{}}{\pairoperation{}}{32768}}
    \label{fig:v1_replicated_systolic_pair_operation_32768_logtime}
\end  {figure}



%%
% Molecular Dynamics
%
\subsection{Molecular Dynamics}

Here, we consider only classical MD simulations.
%
A classical MD simulation is performed by
solving the Newtonian equations of motion for
a molecular system.
%
As there are typically more than 3 bodies interacting in the simulation,
it is not possible to solve these equations analytically.
%
The system must be solved numerically.
%
The numerical solutions to the equations of motion
typically integrate over a timestep $h$.
%
To simulate the system for a time $t$,
one must perform $t/h$ steps.
%
An initial attempt at solving the equations of motion yield
a set of equations requiring $\bigO{N^2}$ steps to solve
for each time step.
%
Some approximations may lead to equations that may be
solved in $\bigO{N}$ steps,
at the cost of reduced accuracy.

An MD simulation typically consists of three parts:
a particle distribution scheme,
a force finding scheme and
a numerical integration scheme.
%
The particle distribution and force finding schemes are
the parts we will be most interested in here.
%
In particular, the potential particle distribution schemes possible
tend to be limited by the force finding scheme in use.
%
The particle distribution scheme also dictates
how the force update scheme may be parallelized.
%
In this study, we use the velocity Verlet integration scheme.

An MD simulation typically consists of three phases:
setting up the particle distribution,
simulating the system until it comes to thermodynamic equilibrium and
taking measurements of the system.

The timescale at which atomic forces manifest compared to
the timescale at which properties of a molecular system may emerge
requires an MD simulation takes very many very short time steps.
%
This in itself introduces a significant amount of challenge to computations.
%
As a classical molecular system can be considered a chaotic one,
we see that any significant error introduced as a result of
increasing time steps may result in
vastly different results in a simulation.
%
Therefore, great care must be taken when choosing time steps, and
indeed a numerical integrator,
that it does not introduce so much error to a simulation as to render it
an unreliable source of information.
%
Some schemes attempt to address this issue by
using different time steps for phenomenon that occur on
different time scales within the same simulation.
%
However, these won’t be discussed here.
%
Indeed, beyond ensuring our numerical integrator and time steps provide
reasonable results for our simulation,
they will be outside the scope of this discussion.


The next area of interest is
the calculation of intermolecular forces that happens on each step.
%
An initial look at the calculation of intermolecular forces suggests
a problem whose solution time scales as $\bigO{N^2}$,
where $N$ is the number of particles in the system.
%
In this dissertation, we will be interested in
tackling the calculation of these forces.
%
Approximations exist providing solutions that scale as $\bigO{N}$.
%
Here, however, we are interested primarily in
studying the direct solutions to these equations of motions,
and the study of parallel schemes which calculate them.


\subsubsection{Potentials}

In an MD simulation, each particle will experience
a potential as a result of its position relative to every other particle.
%
The gradient of that potential then determines
the force felt by that particle.
%
In a classical MD simulation, we choose an approximation of a real potential.
This can be represented as a sum of potentials
which take account of interactions between increasing numbers of particles
\begin{align}
    V =&   \sum_{i=1}^N \sum_{j=1}^N V_2(\vec{x}_i, \vec{x}_j)
    \\ & + \sum_{i=1}^N \sum_{j=1}^N \sum_{k=1}^N
            V_3(\vec{x}_i, \vec{x}_j, \vec{x}_k)
    \\ & + \sum_{i=1}^N \sum_{j=1}^N \sum_{k=1}^N \sum_{l=1}^N
            V_3(\vec{x}_i, \vec{x}_j, \vec{x}_k, \vec{x}_l)
    \\ & + ...
\end  {align}
where $V$ is the potential,
$N$ is the number of particles,
$x_i$ is the position of particle $i$ and
$V_n$ is the $n$ body potential.
%
Often, the contribution of the $V_{n+1}$ term will be
much smaller than the $V_{n}$ term.
%
As a result, the potential may be approximated to the first few terms.


\begin{figure}
    \input{background/lennard_jones_potential.plt}
    \caption{The Lennard-Jones Potential with $\epsilon = 1$ and $\delta = 1$}
\end  {figure}
%
In this dissertation, we will consider only the two body term.
For this, we will use the Lennard-Jones potential
\begin{equation}
    V_{LJ}(r) = 4\epsilon \left[
        \left( \frac{\delta}{r} \right)^{12}
        - \left( \frac{\delta}{r} \right)^{6}
    \right]
\end  {equation}
where $V_{LJ}(r)$ is the Lennard-Jones potential as
a function of the distance $r$,
$r$ is the distance between two bodies,
$\epsilon$ is the depth of the potential well and
$\delta$ is the distance at which
the potential switches from being attractive to repulsive.
Typically, $\epsilon$ and $\delta$ are determined for a system by
fitting them to experimental data.
%
Due to the simplicity of this potential,
it is well studied and there exists good data on
reasonable values for $\epsilon$ and $\delta$ for
different molecular systems.

The Lennard-Jones potential is a short range potential.
%
That is, after a certain distance,
the potential rapidly tends towards zero.
%
This is useful for force evaluation methods that
ignore interactions between distant bodies to
reduce the number of calculations performed.
%
The distance at which the potential may be cut off is important
as it sets a lower limit on how many
particles must be compared in a molecular system.


\subsubsection{Limitations}

Classical MD simulations do not account for quantum effects,
and as such make some false predictions.
%
For example, A quantum approach would yield
a different distribution of
molecular vibrational frequencies to a classical approach.
%
Typically, a classical simulation will yield incorrect results
where dealing with high speeds or low temperatures.
%
This limits the areas where it may be used for study.

There are several potential issues regarding accuracy in MD simulations.
%
An $N$ body system is a chaotic one.
%
As a result, a small changes in accuracy may lead to
largely different results.
%
Errors may be introduced by
the numerical integration scheme used,
the time step used and
approximations of the forces used.
%
While a final answer may not be accurate,
the simulation may still be useful.
%
The system may be used to derive information on
thermodynamic variables or to simulate general trends.


MD calculations tend to be computationally intense.
%
This intensity derives both from
the number of time steps required to derive useful data and
the number of operations required to evaluate the force term on each step.
%
As a result, the sizes of the largest systems studied today
tends to be in the millions.
%
While this number varies,
and different methods of evaluating forces yield
different scaling factors,
there still appears to be an upper limit on
the number of particles that can currently be simulated.


%%
% Objectives
%
\subsection{Objectives}

In particular, we are interested in three distribution schemes:
A replicated distribution scheme,
a domain decomposed distribution scheme,
a systolic loop distribution scheme.
%
We will explore the implementation and optimization of these three schemes
along with their suitability for evaluating both long and short range forces.
%
These implementations will be evaluated on performance in terms of the
minimum time to solution.
%
As useful systems may be simulated with
fixed numbers of particles,
we are not particularly interested in the weak scaling performance of
the system.
%
As such, we will focus on strong scaling results for 4 system sizes:
$10^4$ particles, $10^5$ particles, $10^6$ particles and $10^7$ particles.
%
For each optimization performed or change made, it will be studied
in the context of these benchmarks.

We will focus as well on the errors produced by a particular approximation.
This will be done by comparing results to
a scheme using a direct solution of the equations of motion
where feasible.
%
For this, it may be necessary to perform calculations using several
different time steps to accurately judge the errors introduced.
%
However, as we are limited on time, it may not be possible to generate
reliable enough statistics to make any hard conclusions.
%
We may only compare implementations.


\section{Background}

%
% Molecular Dynamics
%
\subsection{Molecular Dynamics}

Here, we consider only classical MD simulations.
%
A classical MD simulation is performed by
solving the Newtonian equations of motion for
a molecular system.
%
As there are typically more than 3 bodies interacting in the simulation,
it is not possible to solve these equations analytically.
%
The system must be solved numerically.
%
The numerical solutions to the equations of motion
typically integrate over a timestep $h$.
%
To simulate the system for a time $t$,
one must perform $t/h$ steps.
%
An initial attempt at solving the equations of motion yield
a set of equations requiring $\bigO{N^2}$ steps to solve
for each time step.
%
Some approximations may lead to equations that may be
solved in $\bigO{N}$ steps,
at the cost of reduced accuracy.

An MD simulation typically consists of three parts:
a particle distribution scheme,
a force finding scheme and
a numerical integration scheme.
%
The particle distribution and force finding schemes are
the parts we will be most interested in here.
%
In particular, the potential particle distribution schemes possible
tend to be limited by the force finding scheme in use.
%
The particle distribution scheme also dictates
how the force update scheme may be parallelized.
%
In this study, we use the velocity Verlet integration scheme.

An MD simulation typically consists of three phases:
setting up the particle distribution,
simulating the system until it comes to thermodynamic equilibrium and
taking measurements of the system.

The timescale at which atomic forces manifest compared to
the timescale at which properties of a molecular system may emerge
requires an MD simulation takes very many very short time steps.
%
This in itself introduces a significant amount of challenge to computations.
%
As a classical molecular system can be considered a chaotic one,
we see that any significant error introduced as a result of
increasing time steps may result in
vastly different results in a simulation.
%
Therefore, great care must be taken when choosing time steps, and
indeed a numerical integrator,
that it does not introduce so much error to a simulation as to render it
an unreliable source of information.
%
Some schemes attempt to address this issue by
using different time steps for phenomenon that occur on
different time scales within the same simulation.
%
However, these won’t be discussed here.
%
Indeed, beyond ensuring our numerical integrator and time steps provide
reasonable results for our simulation,
they will be outside the scope of this discussion.


The next area of interest is
the calculation of intermolecular forces that happens on each step.
%
An initial look at the calculation of intermolecular forces suggests
a problem whose solution time scales as $\bigO{N^2}$,
where $N$ is the number of particles in the system.
%
In this dissertation, we will be interested in
tackling the calculation of these forces.
%
Approximations exist providing solutions that scale as $\bigO{N}$.
%
Here, however, we are interested primarily in
studying the direct solutions to these equations of motions,
and the study of parallel schemes which calculate them.


\subsubsection{Potentials}

In an MD simulation, each particle will experience
a potential as a result of its position relative to every other particle.
%
The gradient of that potential then determines
the force felt by that particle.
%
In a classical MD simulation, we choose an approximation of a real potential.
This can be represented as a sum of potentials
which take account of interactions between increasing numbers of particles
\begin{align}
    V =&   \sum_{i=1}^N \sum_{j=1}^N V_2(\vec{x}_i, \vec{x}_j)
    \\ & + \sum_{i=1}^N \sum_{j=1}^N \sum_{k=1}^N
            V_3(\vec{x}_i, \vec{x}_j, \vec{x}_k)
    \\ & + \sum_{i=1}^N \sum_{j=1}^N \sum_{k=1}^N \sum_{l=1}^N
            V_3(\vec{x}_i, \vec{x}_j, \vec{x}_k, \vec{x}_l)
    \\ & + ...
\end  {align}
where $V$ is the potential,
$N$ is the number of particles,
$x_i$ is the position of particle $i$ and
$V_n$ is the $n$ body potential.
%
Often, the contribution of the $V_{n+1}$ term will be
much smaller than the $V_{n}$ term.
%
As a result, the potential may be approximated to the first few terms.


\begin{figure}
    \input{background/lennard_jones_potential.plt}
    \caption{The Lennard-Jones Potential with $\epsilon = 1$ and $\delta = 1$}
\end  {figure}
%
In this dissertation, we will consider only the two body term.
For this, we will use the Lennard-Jones potential
\begin{equation}
    V_{LJ}(r) = 4\epsilon \left[
        \left( \frac{\delta}{r} \right)^{12}
        - \left( \frac{\delta}{r} \right)^{6}
    \right]
\end  {equation}
where $V_{LJ}(r)$ is the Lennard-Jones potential as
a function of the distance $r$,
$r$ is the distance between two bodies,
$\epsilon$ is the depth of the potential well and
$\delta$ is the distance at which
the potential switches from being attractive to repulsive.
Typically, $\epsilon$ and $\delta$ are determined for a system by
fitting them to experimental data.
%
Due to the simplicity of this potential,
it is well studied and there exists good data on
reasonable values for $\epsilon$ and $\delta$ for
different molecular systems.

The Lennard-Jones potential is a short range potential.
%
That is, after a certain distance,
the potential rapidly tends towards zero.
%
This is useful for force evaluation methods that
ignore interactions between distant bodies to
reduce the number of calculations performed.
%
The distance at which the potential may be cut off is important
as it sets a lower limit on how many
particles must be compared in a molecular system.


\subsubsection{Limitations}

Classical MD simulations do not account for quantum effects,
and as such make some false predictions.
%
For example, A quantum approach would yield
a different distribution of
molecular vibrational frequencies to a classical approach.
%
Typically, a classical simulation will yield incorrect results
where dealing with high speeds or low temperatures.
%
This limits the areas where it may be used for study.

There are several potential issues regarding accuracy in MD simulations.
%
An $N$ body system is a chaotic one.
%
As a result, a small changes in accuracy may lead to
largely different results.
%
Errors may be introduced by
the numerical integration scheme used,
the time step used and
approximations of the forces used.
%
While a final answer may not be accurate,
the simulation may still be useful.
%
The system may be used to derive information on
thermodynamic variables or to simulate general trends.


MD calculations tend to be computationally intense.
%
This intensity derives both from
the number of time steps required to derive useful data and
the number of operations required to evaluate the force term on each step.
%
As a result, the sizes of the largest systems studied today
tends to be in the millions.
%
While this number varies,
and different methods of evaluating forces yield
different scaling factors,
there still appears to be an upper limit on
the number of particles that can currently be simulated.


esubsection{Parallel Schemes}
\FIG{fig:v0_systolic_pair_operation_512_logtime},
\FIG{fig:v0_systolic_pair_operation_4096_logtime} and
\FIG{fig:v0_systolic_pair_operation_32768_logtime}

%
%Q: Can we parallelize the force calculation?
As discussed, the \velocityverlet{} scheme
immediately provides an algorithm for
performing MD simulations of long ranged forces in isolated systems.
%
We would like to explore parallel implementations of this algorithm.
%
In this dissertation, we begin by focusing on two particular schemes
that directly evaluate the \velocityverlet{} algorithm:
\replicateddata{} and the \systolicloop{}.



\subsubsection{Replicated Data}
\label{sec:background:subsec:replicated_data}

%
%Q: What is the replicated data scheme?
In the \replicateddata{} scheme,
the entire system of particles is replicated across all processors.
%
This is useful for the force update, as each particle must have a view of
every other particle to determine its instantaneous force.
%
The general approach is, in $P$ replicated processes, for each processor
to determine the force for $1/P$ of the particles.
%
Each process may then either update the particles whose forces it has
determined and then share them, or share them and then update the entire
list of particles.
%
In this way, it may provide an $\bigO{N^2/P}$ algorithm.

%
%Q: What are the scaling challenges of replicated data?
Ideally, we would like to simply use $P = N^2$ processes and produce an
$\bigO{1}$ algorithm, however, this is not possible for several reasons.
%
An immediate concern is that each process must have the same view of the
system at all times.
%
As a result, all processes must share their updated particles at
every time step and receive updated particles from other processes.
%
This results in a large amount of communications occurring very frequently,
with a size growing as $\bigO{N}$ and as $\bigO{\log{P}}$,
yielding a communication time of $\bigO{N \log{P}}$.

Further problems arising from this algorithm is that every process has
an individual copy of the system.
%
For large systems running on a large number of cores,
this can stretch the memory of the machine it is running on.
%
Imagining a system of particles where each particle contains double
precision floating point (8~B) data for position, velocity, force and mass,
amounting to 80~B per particle, a 32~GB system could hold around $10^8$
particles.
%
However, if the system has around 10 cores per 32~GB of memory,
with the replicated scheme,
it can now hold a maximum of around $10^7$ particles
as each core will be holding a full system.
%
Essentially, the maximum size of the system becomes inversely proportional
to the number of cores used per GB of memory.
%
This increased use of memory can cause other issues such as poor caching
due to the increased memory use.

We further encounter the problem that all processes must communicate
with every other process at every time step.
%
This puts global synchronisation points into the simulation.
%
Frequent global synchronisation and global data exchange can incur
significant overhead for the calculation.

%
%Q: Why are the scaling challenges of replicated data concerning for exascale?
These drawbacks are particularly important when considering exascale systems
where we expect large numbers of cores per node.
%
As we expect memory per node to grow at a much lower rate, a \replicateddata{}
scheme may have difficulties utilising every core on a node.
%
Indeed, even in current systems, nodes are often requested and only a
subset of cores used due to memory per core considerations in applications.
%
Even if all the cores in the node are utilised, global synchronisation
points do not scale well to large numbers of processes.



\subsubsection{Systolic Loop}
\label{sec:background:subsec:systolic_loop}

%
%Q: What is the systolic loop scheme?
In the \systolicloop{} scheme, the system is split up and distributed among
all the processes.
%
Force updates are performed by passing list data around a ring in pulses.

Initially, a process will make a copy of its list of particles,
and then use that to perform a partial force update on its local list.
%
It will then pass the copied list to the ``right'' process and
receive a new list from the ``left'' process and use that to perform
another partial force update on its local list.
%
This process continues until a process receives its own system again from
the ``left'' process, signifying that it has at some stage received
a copy of every list on every process.
%
At this point, it should have added the force contributions from
every particle in the system to the particles in its local list.
%
From here, a process may update the position and velocity of these local
particles, thereby stepping the system forward.
%
As we are performing an $\bigO{\left(\frac{N}{P}\right)^2}$ operation
when comparing particles and perform this $\bigO{P}$ times, we
get an $\bigO{N^2/P}$ algorithm.


%
%Q: What are the scaling challenges of systolic loop?
As the system is distributed over a number of processes, this allows
us to work on very large systems.
%
Taking the systems and figures from the previous section,
rather than being limited to $10^7$ particles for the system,
we are limited to $10^7$ particles per core.
%
As a result, we have an upper limit on particles per core, but no
upper limit on the actual system being studied.
%
Further, communications between processes is more limited than the
replicated case.
%
Here, we have each process communicating with only two other processes
and sending only $N/P$ particles.
%
By this mechanism, we simultaneously reduce the volume of data
being communicated between every process and
remove a hard global synchronisation
when compared to the \replicateddata{} scheme.
%
However, we must perform $P$ communications.

Ideally, we could set $P = N$ and have a $\bigO{1}$ operation performed
$\bigO{P = N}$ times.
%
The primary limitation here is the overhead of performing the systolic
pulses.
%
As communications have a given latency, we would like to reduce the number
of communications performed.
%
In a \systolicloop{}, the number of operations performed reduces with
the square of the processes, but the communications performed scales linearly.
%
As such, we quickly reach a point where latency dominates over computation.
%
Even for very large $N$, a modest $P$ can incur large latency overheads.
%
%Q: Why are the scaling challenges of systolic loop concerning for exascale?
As increasing the number of processes to a large value
can quickly result in calculations having significant latency,
we see an immediate concern for exascale system.


\chapter{Methodology}

Parallel implementations are required to be functionally equivalent
to the example serial implementations outlined in \SEC{sec:methodology:subsec:implementation}.
%
Tests may then be implemented straightforwardly for the serial implementation,
and are then easily extended to the parallel implementations.

The prescribed interface discourages the use of optimisations which make
assumptions about the MD algorithm being implemented.
%
If it were known ahead of time, for example, that
only forces would be updated during a \pairoperation{},
as is the case in the \velocityverlet{} algorithm,
or that inter-atomic forces dropped to zero after a given distance,
the implementation may be able to use this information to improve
performance through specialized data layouts or communications patterns.

Optimizations for particular algorithms and MD systems are
beyond the scope of this dissertation.
%
The focus here is on general communication patterns rather than
the optimization of particular MD algorithms.

Implementations of the \replicateddata{} and \systolicloop{} schemes
using only MPI will be analysed and discussed by focusing on
the scaling results of the \individualoperation{} and \pairoperation{} methods.


\subsection{Replicated Data}

The replicated scheme allocates a list of particles the size of
the entire MD system of particles
on each process.
%
It uses this list to keep an up-to-date copy of the system of particles
on every process.

In this section, the implementation details and performance of
of the \individualoperation{} and \pairoperation{} methods
for the replicated data scheme using only MPI will be analysed and discussed.


%
% Replicated individual_operation v0
%

\subsubsection{Implementation of the \individualoperation{} Method}

The \individualoperation{} method as outlined in
\SEC{sec:the_individual_operation_method}
is implemented by having each process update its entire local list.
%
As such, it closely resembles the example serial implementation.
%
This approach involves more computation than having each process
evaluate a subsection of the list and share the result with the
other processes, but it avoids a global synchronization.
%
As each process performs an $\bigO{1}$ operation on $N$ particles with
no communications,
this implementation is expected to take a time
\begin{equation}
\label{eqn:v0_replicated_individual_operation_overall_time}
    N\bigO{1} = \bigO{N}
\end  {equation}

\begin{figure}[!h]
    \input{parallel_implementation/v0/replicated.individual_operation.512.time.plt}
    \caption{\vZeroTimeCaption{Replicated Data}{\individualoperation{}}{512}}
    \label{fig:v0_replicated_individual_operation_512_time}
\end  {figure}

\begin{figure}[!h]
    \input{parallel_implementation/v0/replicated.individual_operation.4096.time.plt}
    \caption{\vZeroTimeCaption{Replicated Data}{\individualoperation{}}{4096}}
    \label{fig:v0_replicated_individual_operation_4096_time}
\end  {figure}

\begin{figure}[!h]
    \input{parallel_implementation/v0/replicated.individual_operation.32768.time.plt}
    \caption{\vZeroTimeCaption{Replicated Data}{\individualoperation{}}{32768}}
    \label{fig:v0_replicated_individual_operation_32768_time}
\end  {figure}


\vZeroTimeExplanation
    { \FIG{fig:v0_replicated_individual_operation_512_time} }
    { \FIG{fig:v0_replicated_individual_operation_4096_time} }
    { \FIG{fig:v0_replicated_individual_operation_32768_time} }
    { \individualoperation{} }


From these, it is clear that
the time for the \individualoperation{} doesn't scale with the number
of cores and that MPI takes up no time as this method uses no MPI calls.

There is an interesting increase of time that occurs at 2 cores and again
at 8 cores.
%
Given that \hector{} has 4 NUMA regions of 8 cores and each of those
regions is further subdivided into 4 NUMA regions of 2 cores,
it is likely this is a cause for the jumps at 2 and 8 cores.
%
This is particularly noticeable in
\FIG{fig:v0_replicated_individual_operation_32768_time}
where the jump occurs at exactly the same core count, but is noticeably larger.

The jumps occur at the same numbers of processes in
\FIG{fig:v0_replicated_individual_operation_512_time},
\FIG{fig:v0_replicated_individual_operation_4096_time} and
\FIG{fig:v0_replicated_individual_operation_32768_time}
and after 8 processes, the time remains roughly constant.
%
This suggests either a latency effect with processes accessing memory
in other NUMA regions or a memory bandwith effect.
%
Given that the effect scales roughly with the number of particles
in the system, at fixed core counts, it is most likely due to
memory bandwidth saturation.
%
If bandwidth saturated, it is expected that the time taken for
data transfer would
scale linearly with the size of the data per core being transferred.
%
Therefore, it is likely that the bandwidth is not saturated in the 2 core NUMA
region when only 1 core is in use, and
similarly that the bandwidth for the 8 core NUMA region is not saturated
when only 7 cores are in use.

Above 8 cores, the overall execution time for
the \individualoperation{} method for the Replicated Data implementation
appears to scale as $\bigO{N}$
as predicted in \EQN{eqn:v0_replicated_individual_operation_overall_time}.


%
% Replicated pair_operation v0
%

\subsubsection{Implementation of the \pairoperation{} Method}

The \pairoperation{} method as outlined in
\SEC{sec:the_pair_operation_method}
is implemented by having each process evaluating the pair
comparisons for a subset of the particles and sharing the
results with the other processes.

Each process is designated a subset of the list of particles for which
it will evaluate the results of the comparison and reduction.
%
This looks similar to parallelising the outer loop of the example
serial implementation.
%
The number of particles each process is assigned is roughly $N/P$.
%
Each of these $N/P$ particles are compared to $N$ other particles
using an $\bigO{1}$ operation.
%
The time for one of the $N/P$ particles to be updated is
\begin{equation}
    N\bigO{1} = \bigO{N}
\end  {equation}
and so, the time for all $N/P$ particles to be updated is
\begin{equation}
    \frac{N}{P}\bigO{N} = \bigO{\frac{N^2}{P}}
\end  {equation}
As such, a calculation term of $\bigO{N^2/P}$ is expected.

After the each process finishes updating it's section of the list,
the updated sections of lists are shared across processes using
an MPI\_Allgatherv.
This introduces a communication term of $\bigO{(N + l)\log{P}}$
where $l$ represents a constant latency.

The overall execution time of this method should therefore be
\begin{align}
    \bigO{\frac{N^2}{P}} + \bigO{(N+l)\log{P}}
        &\approx{} \bigO{\frac{N^2}{P} + (N+l)\log{P}} \\
        &\approx{} \bigO{\frac{N^2}{P} + (N+l)\log{P}} \\
        &\approx{} \bigO{\frac{N^2}{P} + N\log{P} + \log{P}} \\
        \label{eqn:v0_replicated_pair_operation_overall_time}
        &\approx{} \bigO{\frac{N^2}{P} + N\log{P}}
\end  {align}

\begin{figure}[!h]
    \input{parallel_implementation/v0/replicated.pair_operation.512.logtime.plt}
    \caption{\vZeroTimeCaption{Replicated Data}{\pairoperation{}}{512}}
    \label{fig:v0_replicated_pair_operation_512_logtime}
\end  {figure}

\begin{figure}[!h]
    \input{parallel_implementation/v0/replicated.pair_operation.4096.logtime.plt}
    \caption{\vZeroTimeCaption{Replicated Data}{\pairoperation{}}{4096}}
    \label{fig:v0_replicated_pair_operation_4096_logtime}
\end  {figure}

\begin{figure}[!h]
    \input{parallel_implementation/v0/replicated.pair_operation.32768.logtime.plt}
    \caption{\vZeroTimeCaption{Replicated Data}{\pairoperation{}}{32768}}
    \label{fig:v0_replicated_pair_operation_32768_logtime}
\end  {figure}

\vZeroTimeExplanation
    { \FIG{fig:v0_replicated_pair_operation_512_logtime} }
    { \FIG{fig:v0_replicated_pair_operation_4096_logtime} }
    { \FIG{fig:v0_replicated_pair_operation_32768_logtime} }
    { \pairoperation{} }

In \FIG{fig:v0_replicated_pair_operation_512_logtime},
\FIG{fig:v0_replicated_pair_operation_4096_logtime} and
\FIG{fig:v0_replicated_pair_operation_32768_logtime},
it is apparent that scaling begins to drop off as the number
of processes used is comparable to the number of particles in the system.
%
From \EQN{eqn:v0_replicated_pair_operation_overall_time} when $P \ll{} N$
\begin{equation}
    \bigO{\frac{N^2}{P} + N\log{P}} \sim{} \bigO{\frac{N^2}{P}}
\end  {equation}
suggesting good scaling in this regime.
%
Similarly when $P \sim{} N$
or $P > N$
\begin{align}
    \bigO{\frac{N^2}{P} + N\log{P}}
        &\sim{} \bigO{N + N\log{P}} \\
        &\sim{} \bigO{N\log{P}}
\end  {align}
%
suggesting the communication term eventually dominating and
the overall time increasing as a function of $P$.

Indeed, examining where
\FIG{fig:v0_replicated_pair_operation_512_logtime},
\FIG{fig:v0_replicated_pair_operation_4096_logtime},
\FIG{fig:v0_replicated_pair_operation_32768_logtime} and
begin straying away from linear scaling,
it can be seen that the minimum execution time
of 512 particles is approximately $10^{-4}$ seconds,
of 4096 particles is approximately $10^{-3}$ seconds and
of 32768 particles is approximately $10^{-2}$ seconds.
%
Thus as scaling begins dropping off at $P = N$,
the minimum execution time for this distribution,
for the system sizes tested,
scales as $N\log{N}$
where $N$ is the number of particles in the system.


\subsection{Systolic Loop}

The systolic loop scheme is initialised by allocating 3 arrays of
size $N/P$ on every process.
%
The system is then split roughly evenly across all the processes
and is held, updated and shared using these three lists.


%
% Systolic individual_operation v0
%

\subsubsection{Implementation of the \individualoperation{} Method}
This is implemented by having each process update its local list
of particles.

This should take $\bigO{N/P}$ time.

\begin{figure}[!h]
    \input{parallel_implementation/v0/systolic.individual_operation.512.logtime.plt}
    \caption{\vZeroTimeCaption{Systolic Loop}{\individualoperation{}}{512}}
    \label{fig:v0_systolic_individual_operation_512_logtime}
\end  {figure}

\begin{figure}[!h]
    \input{parallel_implementation/v0/systolic.individual_operation.4096.logtime.plt}
    \caption{\vZeroTimeCaption{Systolic Loop}{\individualoperation{}}{4096}}
    \label{fig:v0_systolic_individual_operation_4096_logtime}
\end  {figure}

\begin{figure}[!h]
    \input{parallel_implementation/v0/systolic.individual_operation.32768.logtime.plt}
    \caption{\vZeroTimeCaption{Systolic Loop}{\individualoperation{}}{32768}}
    \label{fig:v0_systolic_individual_operation_32768_logtime}
\end  {figure}

As seen in
\FIG{fig:v0_systolic_individual_operation_512_logtime},
\FIG{fig:v0_systolic_individual_operation_4096_logtime} and
\FIG{fig:v0_systolic_individual_operation_32768_logtime}
the current implementation satisfies this.

There appear to be a few unexpected data points in the mpi only timings,
but these are unlikely to be anything except an error.
%
As these are approaching nanosecond execution times, it is
unsurprising that they may pick up unexpected errors as the system clock
has at most a nanosecond resolution time.
%
Given the processor operates on about this time, it is also unsurprising
if some odd times may be picked up from taking measurements so
close to this time scale.


%
% Systolic pair_operation v0
%

\subsubsection{Implementation of the \pairoperation{} Method}
This is implemented by having each process use three lists of particles,
each of size $P/N$.

The first list is the processes local list of particles.

The second list will be referred to as the foreign list, and
represents a list originating from another process.

The third list is a swap list to allow a process to receive a new
foreign list list from the right
while also sending its old foreign list to the left
during a systolic pulse.

Initially, a process will copy its local list to the foreign list
and perform a partial force update on its local list using this
foreign list.
%
The system will then perform a systolic pulse.
After a systolic pulse, every foreign list should move one process
to the left in the systolic loop.
%
This is performed by copying the old foreign list into the
swap list, and using an MPI\_sendrecv to send the swap list to
the left process while receiving from
the right process into the foreign list.
%
When a new foreign list is received, another partial force update
is performed on the local list.
%
This process is repeated $P-1$ times.

Each list comparison between systolic pulses should take $\bigO{(N/P)^2}$ time.
%
For a given timestep, there should be $P$ of these list comparisons
performed, giving an over calculation time of $\bigO{N^2/P}$.

Given each pulse should be passing $N/P$ particles between two processes,
we expect this to take $\bigO{N/P + l}$ time where $l$ is a constant latency.
%
With $P$ pulses on each time step, we expect a communication time of
$\bigO{(N/P + l)P}$.

Combining our calculation and communication terms, the systolic loop approach
should run in $\bigO{N^2/P + dN + dlP}$ time
where $N$ is the number of particles in the system,
$P$ is the number of processes used and
$l$ is a constant latency and
$d$ is a constant.

\begin{figure}[!h]
    \input{parallel_implementation/v0/systolic.pair_operation.512.logtime.plt}
    \caption{\vZeroTimeCaption{Systolic Loop}{\pairoperation{}}{512}}
    \label{fig:v0_systolic_pair_operation_512_logtime}
\end  {figure}

\begin{figure}[!h]
    \input{parallel_implementation/v0/systolic.pair_operation.4096.logtime.plt}
    \caption{\vZeroTimeCaption{Systolic Loop}{\pairoperation{}}{4096}}
    \label{fig:v0_systolic_pair_operation_4096_logtime}
\end  {figure}

\begin{figure}[!h]
    \input{parallel_implementation/v0/systolic.pair_operation.32768.logtime.plt}
    \caption{\vZeroTimeCaption{Systolic Loop}{\pairoperation{}}{32768}}
    \label{fig:v0_systolic_pair_operation_32768_logtime}
\end  {figure}

We see in 
\FIG{fig:v0_systolic_pair_operation_512_logtime},
\FIG{fig:v0_systolic_pair_operation_4096_logtime} and
\FIG{fig:v0_systolic_pair_operation_32768_logtime}
that, much like in the replicated case, that the system scales
roughly as $N/P$ when $P \ll{} N$.

With a communication term scaling as $P$, we see the point in which
communications dominate comes much sooner.
%
However, we also note that it appears when $P \approx{} N$.
%
Taking our previous approach of finding an optimum $k$ for $P = N/k$,
we find $k \approx{} 8$.

This appears to hold for our three system sizes, although there does appear
to be an unexpected jump in
\FIG{fig:v0_systolic_pair_operation_32768_logtime}
between 512 and 1024 processes.
%
It is unclear whether this is a genuine effect, or simply an error in
measurement.

Ignoring the unexpected jump in 
\FIG{fig:v0_systolic_pair_operation_32768_logtime},
we may conclude that the minimum time to completion for out system
scales linearly with the number of particles.

It would appear the implimentation ultimately becomes slowed due to
communication latency, and in particular, due to a large number
of these communications.
%
The initial scaling of this term is due to the number of particles
in the system, however,
scaling with $P$ is a result of having $P$ communications
per time step.
%
Looking at our derivation for communication times,
we conclude that this must be an effect of latency.

\section{Shared and Replicated Data}

The \sharedandreplicateddata{} scheme is implemented in exactly the same
way as the \replicateddata{} scheme outlined in 
\SEC{sec:replicated_data_implementation},
except the update loop in the \pairoperation{} method is further
parallelised using \openmp{} directives, taking advantage of shared
memory between cores.
%
In fact, this distribution class inherits directly from the
\replicateddata{} distribution class, and overloads only the
\pairoperation{} method.


The primary motivation for this is to show that mixed mode MPI and \openmp{}
paralellism is just as viable as MPI only parallelism for a \replicateddata{}
scheme along with how easily the mixed mode parallelism may be implemented
on top of an MPI implementation.
%
The importance of this is that the maximum system size per core per node
can be increased in proportion to the number of \openmp{} threads
created per MPI process, which is of particular importance for nodes with
particularly high core counts.

The \sharedandreplicateddata{} distribution is initialised by
allocating a list of particles the size of the system of particles
on every MPI process.
%
The number of \openmp{} threads used per MPI processes is fixed at 8,
as this is the suggested number for \hector{} due to the arrangement
of the \numa{} regions into groups of 8.
%
As this is a rather low number compared to the overall number of MPI
processes, the emphasis here isn't to gain any perticular performance
improvement using \openmp{}, only to show that it is viable.
%
Indeed, the mere introduction of 8 threads per MPI process should increase
the maximum system size that can be run 8 fold.

In this section,
the implementation details and performance of
the \individualoperation{} and \pairoperation{} methods
will be presented and analysed.


\subsection{\individualoperation{}}

The \individualoperation{} method is inherited directly from
the implementation outlined in
\SEC{sec:replicated_data_individual_operation_implementation}.
%
As a result, the time to completion is also expected to scale as $\bigO{N}$.

%
% Overall speedup plot
%
\begin{figure}[!h]
    \input{parallel_implementation/v1/shared_and_replicated.individual_operation.logspeedup.plt}
    \caption{
        Speedup plots for the \individualoperation{} implemented with the \sharedandreplicateddata{} scheme for systems of particles of size 512, 4096 and 32768.
    }
    \label{fig:v1_shared_and_replicated_data_individual_operation_speedups}
\end{figure}


%
% Individual breakdowns
%
\begin{figure}[!h]
    \input{parallel_implementation/v1/shared_and_replicated.individual_operation.512.time.plt}
    \caption{\vZeroTimeCaption{\sharedandreplicateddata{}}{\individualoperation{}}{512}}
    \label{fig:v1_shared_and_replicated_individual_operation_512_time}
\end  {figure}

\begin{figure}[!h]
    \input{parallel_implementation/v1/shared_and_replicated.individual_operation.4096.time.plt}
    \caption{\vZeroTimeCaption{\sharedandreplicateddata{}}{\individualoperation{}}{4096}}
    \label{fig:v1_shared_and_replicated_individual_operation_4096_time}
\end  {figure}

\begin{figure}[!h]
    \input{parallel_implementation/v1/shared_and_replicated.individual_operation.32768.time.plt}
    \caption{\vZeroTimeCaption{\sharedandreplicateddata{}}{\individualoperation{}}{32768}}
    \label{fig:v1_shared_and_replicated_individual_operation_32768_time}
\end  {figure}

\vZeroTimeExplanation
    {\FIG{fig:v0_replicated_individual_operation_512_time}}
    {\FIG{fig:v0_replicated_individual_operation_4096_time}}
    {\FIG{fig:v0_replicated_individual_operation_32768_time}}
    {\individualoperation{}}
    {\replicateddata{}}

As can be seen from 
\FIG{fig:v0_replicated_individual_operation_512_time},
\FIG{fig:v0_replicated_individual_operation_4096_time} and
\FIG{fig:v0_replicated_individual_operation_32768_time}
the performace scaling does indeed scale as $\bigO{N}$, as expected
and exactly in line with the performance scaling results of the
\individualoperation{} method in the \replicateddata{} scheme.



\subsection{\pairoperation{}}

The \pairoperation{} method is implemented in a similar manner to the
\pairoperation{} method from the \replicateddata{} scheme with the
addition of \openmp{} directives to further parallelise the
force update loop.

A copy of the system is held on each MPI process, meaning there
are $P_{MPI}$ replicas of the system created.
%
Each MPI process is then assigned $N/P_{MPI}$ particles in that system
to determine forces for.
%
Given that list of $N/P_{MPI}$ particles,
an MPI process spawns $P_{OMP}$ threads
and assigns $1/(P_{MPI} P_{OMP})$ particles to each thread.

Writing $P = P_{MPI} \times{} P_{OMP}$,
each core therefore has $N/P$ particles
for which it must determine the forces.
%
As before, if each particles must be compared to $N$ other particles
using an $\bigO{1}$ update operation, the time to find the force for
a single particles is
\begin{equation}
    N\bigO{1} = \bigO{N}
\end{equation}
And so, the time to find the forces for $N/P$ particles is
\begin{equation}
    \frac{N}{P}\bigO{N} = \bigO{\frac{N^2}{P}}
\end{equation}

After the forces have been determined by each thread, the team of threads
shuts down.
%
This represents a synchronisation point for the threads within the
MPI process.
%
After this, an MPI\_Allgatherv is used over the $P_{MPI}$ processes
to synchronise the updated lists of particles, representing a
$\bigO{N\log{P_{OMP}}}$ operation.

%
% Overall speedup plot
%
\begin{figure}[!h]
    \input{parallel_implementation/v1/shared_and_replicated.pair_operation.logspeedup.plt}
    \caption{
        Speedup plots for the \pairoperation{} implemented with the \sharedandreplicateddata{} scheme for systems of particles of size 512, 4096 and 32768.
    }
    \label{fig:v1_shared_and_replicated_data_pair_operation_speedups}
\end{figure}


%
% Individual breakdowns
%
\begin{figure}[!h]
    \input{parallel_implementation/v1/shared_and_replicated.pair_operation.512.logtime.plt}
    \caption{\vZeroTimeCaption{\sharedandreplicateddata{}}{\pairoperation{}}{512}}
    \label{fig:v1_shared_and_replicated_pair_operation_512_logtime}
\end  {figure}

\begin{figure}[!h]
    \input{parallel_implementation/v1/shared_and_replicated.pair_operation.4096.logtime.plt}
    \caption{\vZeroTimeCaption{\sharedandreplicateddata{}}{\pairoperation{}}{4096}}
    \label{fig:v1_shared_and_replicated_pair_operation_4096_logtime}
\end  {figure}

\begin{figure}[!h]
    \input{parallel_implementation/v1/shared_and_replicated.pair_operation.32768.logtime.plt}
    \caption{\vZeroTimeCaption{\sharedandreplicateddata{}}{\pairoperation{}}{32768}}
    \label{fig:v1_shared_and_replicated_pair_operation_32768_logtime}
\end  {figure}

\section{Shared and Replicated Data}

\subsection{\individualoperation{}}

%
% Overall speedup plot
%
\begin{figure}[!h]
    \input{parallel_implementation/v1/replicated_systolic.individual_operation.logspeedup.plt}
    \caption{
        Speedup plots for the \individualoperation{} implemented with the \replicatedsystolicloop{} scheme for systems of particles of size 512, 4096 and 32768.
    }
    \label{fig:v1_replicated_systolic_loop_individual_operation_speedups}
\end{figure}


%
% Individual breakdowns
%
\begin{figure}[!h]
    \input{parallel_implementation/v1/replicated_systolic.individual_operation.512.logtime.plt}
    \caption{\vZeroTimeCaption{\replicatedsystolicloop{}}{\individualoperation{}}{512}}
    \label{fig:v1_replicated_systolic_individual_operation_512_time}
\end  {figure}

\begin{figure}[!h]
    \input{parallel_implementation/v1/replicated_systolic.individual_operation.4096.logtime.plt}
    \caption{\vZeroTimeCaption{\replicatedsystolicloop{}}{\individualoperation{}}{4096}}
    \label{fig:v1_replicated_systolic_individual_operation_4096_time}
\end  {figure}

\begin{figure}[!h]
    \input{parallel_implementation/v1/replicated_systolic.individual_operation.32768.logtime.plt}
    \caption{\vZeroTimeCaption{\replicatedsystolicloop{}}{\individualoperation{}}{32768}}
    \label{fig:v1_replicated_systolic_individual_operation_32768_time}
\end  {figure}




\subsection{\pairoperation{}}

%
% Overall speedup plot
%
\begin{figure}[!h]
    \input{parallel_implementation/v1/replicated_systolic.pair_operation.logspeedup.plt}
    \caption{
        Speedup plots for the \pairoperation{} implemented with the \replicatedsystolicloop{} scheme for systems of particles of size 512, 4096 and 32768.
    }
    \label{fig:v1_replicated_systolic_pair_operation_speedups}
\end{figure}


%
% Individual breakdowns
%
\begin{figure}[!h]
    \input{parallel_implementation/v1/replicated_systolic.pair_operation.512.logtime.plt}
    \caption{\vZeroTimeCaption{\replicatedsystolicloop{}}{\pairoperation{}}{512}}
    \label{fig:v1_replicated_systolic_pair_operation_512_logtime}
\end  {figure}

\begin{figure}[!h]
    \input{parallel_implementation/v1/replicated_systolic.pair_operation.4096.logtime.plt}
    \caption{\vZeroTimeCaption{\replicatedsystolicloop{}}{\pairoperation{}}{4096}}
    \label{fig:v1_replicated_systolic_pair_operation_4096_logtime}
\end  {figure}

\begin{figure}[!h]
    \input{parallel_implementation/v1/replicated_systolic.pair_operation.32768.logtime.plt}
    \caption{\vZeroTimeCaption{\replicatedsystolicloop{}}{\pairoperation{}}{32768}}
    \label{fig:v1_replicated_systolic_pair_operation_32768_logtime}
\end  {figure}


\section{The Abstract Distribution}
\label{sec:methodology:subsec:implementation}

%
%Q: How will the code be implemented?
The code for this dissertation is implemented in
object-oriented Fortran 2003.
%
%Q: How will we handle changing the underlying parallel pattern?
To allow for arbitrary \twobody{} potentials and integration schemes
to be used with arbitrary parallel schemes,
the routines calculating how an individual particle is stepped forward
in time and the routine determining when and where that stepping is applied
are separated.
%
Similarly, the routines calculating the force between two particles
are separated from the routine for combining all of these
\twobody{} forces to find the total instantaneous force on a particle.

\begin{figure}
    \begin{center}
    \begin{tikzpicture}[node distance=3.5cm]
        \umlclass{particle}
            {
                double precision :: pos(Ndim) \\
                double precision :: vel(Ndim) \\
                double precision :: force(Ndim) \\
                double precision :: mass
            }{}
        \umlclass[y=-5, type=abstract]{abstract\_distribution}
            {
                type(particle), allocatable :: particles(:)
            }{
                individual\_operation \\
                pair\_operation
            }
        \umlaggreg[mult2=1..N, mult1=1]
            {abstract\_distribution}{particle}
        \umlemptyclass[y=-8.9, x=-3]{replicated\_distribution}
        \umlemptyclass[y=-12.5, x=-3]{shared\_and\_replicated\_distribution}
        \umlemptyclass[y=-8.9, x=4.05]{systolic\_distribution}
        \umlemptyclass[y=-11, x=4.1]{replicated\_systolic\_distribution}
        \umlreal[geometry=-|]{replicated\_distribution}{abstract\_distribution}
        \umlreal[geometry=-|]{systolic\_distribution}{abstract\_distribution}
        \umlinherit[geometry=-|]
            {shared\_and\_replicated\_distribution}{replicated\_distribution}
        \umlreal[geometry=-|]
            {replicated\_systolic\_distribution}{abstract\_distribution}
    \end{tikzpicture}
    \end{center}
    \caption{
        Overview of the \abstractdistribution{} class and inheritence tree.
    }
\end{figure}

To address this, an \abstractdistribution{} class is implemented
which holds a list of particles and
provides a set of methods for
applying transformations over the particles.
%
The user can define functions to be passed into these methods
as parameters, and the method should perform a predefined operation
over the list of particles using these user defined functions.
%
%Q: What does an implementation look like?
An implementation of a parallel scheme inherits from the
\abstractdistribution{} class and is free to distribute
the list of particles it holds across processes as it wishes,
and should implement the required interfaces as appropriate for that scheme.
%
This allows the handling of arbitrary user defined operations while
allowing the underlying parallel scheme to be changed as necessary.

%
%Q: What interfaces will our implementations provide?
The MD loop illustration in
\FIG{fig:md_loop_flow_chart}
suggests a set of interfaces to be implemented.
%
Namely, one for capable of performing force updates and another
capable of performing an integration step.
%
Integration step can be implemented with a method that can update particles
in place.
%
The force update can be implemented with a method that can
update each particle based on some
comparison with every other particle in the system.
%
These two methods will be outlined here with an example serial
implementation provided for clarity.
%
They will be referred to as the \individualoperation{} method and
the \pairoperation{} method.


\subsection{The \individualoperation{} Method}
\label{sec:the_individual_operation_method}

This method will accept a function defining a mapping from
particles and integers to particles.
\begin{equation}
    f: p\times{}i \rightarrow{} p
\end{equation}
%
The intention is that a particle and the number of that particle in
the particle list is passed to the function, and the function
returns an updated particle based on the particle's current parameters
and its position in the list.
%
This replaces each particle in the list of particles with
a newly updated particle.

An example serial implementation is in
\LST{lst:serial_individual_operation_implementation}.
An example of using this method to initialise the list of
particles is given in
\LST{lst:individual_operation_example_usage}.

\begin{lstlisting}[
    caption=Serial \individualoperation{} implementation.,
    label=lst:serial_individual_operation_implementation,
    gobble=3
]
    subroutine individual_operation(this, f)
        class(serial_distribution), intent(inout) :: this
        procedure(one_particle_function), intent(in) :: f

        integer :: i


        do i=1, this%num_particles
            this%particles(i) = f(this%particles(i), i)
        end do
    end subroutine individual_operation
\end{lstlisting}

\begin{lstlisting}[
    caption=Example usage of the \individualoperation{} to initialise
        a list of particles.,
    label=lst:individual_operation_example_usage,
    gobble=3
]
    program ExampleIndividualOperation
        use serial_distribution_type
        use particle_type
        implicit none


        type(serial_distribution) :: dist

    
        ! Initialise a serial_distribution
        dist = new_serial_distribution(...)

        ! Apply init_particles to the list using
        ! the individual_operation.
        call dist%individual_operation(init_particles)

        ! Particle i of the particle list should now have
        ! a position (i, i, i), velocity (0, 0, 0),
        ! force (0, 0, 0) and mass 1.

    contains
        ! Initialise particle i to have position (i, i, i)
        PURE function init_particles(p_i, i)
            type(particle) :: init_particles

            type(particle), intent(in) :: p_i
            integer, intent(in) :: i

            return particle(pos=i, vel=0, force=0, mass=1)
        end function init_particles
    end program ExampleIndividualOperation
\end{lstlisting}


\subsection{The \pairoperation{} Method}
\label{sec:the_pair_operation_method}

This method accepts
a function defining a mapping from two particles to a value;
a reduction operation provided by the implementation;
a mapping from a particle and a value a new particle;
and an initial value to use for reductions.

\begin{equation}
    \begin{split}
        f &: p\times{}p \rightarrow{} v^n \\
        r &: v^n\times{}v^n \rightarrow{} v^n \\
        s &: p\times{}v^n \rightarrow{} p \\
        iv &: v^n
    \end{split}
\end{equation}

This can be considered a map/reduce operation where $f$ is the map operation
and $r$ is the reduce operation.
%
The value $iv$ is used as the initial value used in the reduce operation.
%
The function $s$ can then be used to set some value of a particle with
the results of this map/reduce operation.

The intention is that the implementation use $s$ to set the value
of particle $i$ to the value given by $iv$.
%
It will then use $f$ to compare particle
$i$ with a particle $j$, where $i \ne{} j$, and use $r$ to
reduce this new value with the current result being held for
particle $i$.
%
Performing this for every particle $j$ and reducing the results together,
the implementation should detmine the correct result for the operation.

An example serial implementation is in
\LST{lst:serial_pair_operation_implementation}.

\begin{lstlisting}[
    caption=Serial \pairoperation{} implementation.,
    label=lst:serial_pair_operation_implementation,
    gobble=3
]
    subroutine pair_operation(                     &
        this                                       &
        pair_to_value, particle_value_to_particle, &
        reduction_type, reduction_init             &
    )
        class(serial_distribution), intent(inout) &
            :: serial_distribution
        procedure(pair_to_val_function), intent(in) &
            :: pair_to_value
        procedure(set_particle_function), intent(in) &
            :: particle_value_to_particle
        procedure(reduce_op_function), intent(in) &
            :: reduction_type
        double precision, intent(in) :: reduction_init(:)

        double precision :: value(size(reduction_init))
        double precision :: tmp_value(size(reduction_init))


        do i=1, this%num_particles
            value = reduction_init

            do j=1, this%num_particles
                call pair_to_value(                       &
                    this%particles(i), this%particles(j), &
                    tmp_value                             &
                )

                value = reduction_type( &
                    value, tmp_value,   &
                )
            end do

            this%particles(i) = particle_value_to_particle(&
                this%particles(i), value                   &
            )
        end do
    end subroutine pair_operation
\end{lstlisting}

As can be seen, some design choices have been taken which limit how
general the user defined operations can be.
%
There are also some unfortunate inconsistencies in the signatures of
user defined operations.
%
These are primarily due to language restrictions.
%
However, the minimum requirement is for the distribution to be able
to implement the \velocityverlet{} algorithm,
so these problems are not a major concern.


\subsection{A Full Example}

An example of an MD simulation using this model is provided in
\LST{lst:distribution_operation}.
%
Presented is an example of creating a distribution,
initialising the particles in the simulation,
an MD update loop and
printing the list of particles out to disk.


\begin{lstlisting}[
    caption=An example MD simulation,
    label=lst:distribution_operation,
]
program MD
    ...

    dist = new_distribution(...)

    call dist%individual_operation(init_particles)

    do while(time < end_time)
        ! Perform first half of velocity Verlet scheme
        call dist%individual_operation(VV_scheme_part_1)

        ! Find forces between particles due to
        ! LJ potential
        call dist%pair_operation(                      &
            map_force, unmap_force, dist%sum, zero_arr &
        )

        ! Perform second half of velocity Verlet scheme
        call dist%individual_operation(VV_scheme_part_2)

        ! Step time forward
        time = update_time(time)
    end do


contains
    ! An overly simplistic initialisation scheme
    PURE function init_particles(particle_i, i)
        ...
        init_particles = Particle(        &
            pos=i, vel=0, mass=1, force=0 &
        )
    end function init_particles

    ! Find the force on particle1 due to particle2
    PURE subroutine map_force(        &
        particle_i, particle_j, force &
    )
        ...
        call find_LJ_force_between(        &
            particle_i, particle_j, force  &
        )
    end subroutine map_force

    ! Expect all the forces to have been summed together
    ! in the "total_force" parameter.
    ! Apply force to the particle and return it.
    PURE function unmap_force(particle_i, total_force)
        ...
        unmap_force = particle_i
        unmap_force%force = total_force
    end function unmap_force

    ! Accept a particle and return a string
    ! that can be printed.
    PURE subroutine particle_to_string( &
        particle_i, i, output_string    &
    )
        ...
        write(output_string, *) particle_i%pos
    end subroutine particle_to_string
end program MD
\end{lstlisting}


\subsection{Testing}

Interfaces for printing information on the particles
out to disk and for performing some map/reduce operations over
the particles and returning a result are also provided.
%
These are used only for testing and debugging, as they are entirely
unoptimised, and indeed, very slow.

The \individualoperation{} is tested by having each particle set
its position vector to $(i, i, i)$ where $i$ is its position in the
particle list.
%
A map/reduce operation is then performed over the particles to
sum all the positions.
%
The result is then compared to $\sum_n=1^N (n, n, n)$.
%
The test is then repeated where particles have their positions set to
$(di, di, di)$ where $d$ is an arbitrary multiplier and $i$ is their
position in the particle list, and the map/reduce result i
compared to the result of $d\sum_n=1^N (n, n, n)$.
%
This is performed for several $d$.

The \pairoperation{} is tested by having each particle's position set
to $(i, i, i)$.
%
A \pairoperation{} is performed where $f$ performed for particles $i$ and
$j$ returns the position of particle $j$.
%
The reduction operation is the element-wise summation of these results.
%
The setting operation is to set the result to the velocity parameter of
the particle.
%
The initialisation value is 0.

The result should be that the velocity parameter of each particle should
be a sum of the positions of each particle it has been compared to.
%
As a result, the velocity of particle $i$ should be equal to
$\sum_{\substack{n=1\\n\ne{}i}}^N (n, n, n)$.
%
A map/reduce operation summing all the velocities of all the particles
in the system should therefore return a value
$(N-1)\sum_{n=1}^N (n, n, n)$.


\subsection{The Replicated Distribution}

%
%Q: How is the replicated data scheme implemented?
The replicated distribution is an implementation of the \replicateddata{}
scheme.
%
In this implementation, a list of $N$ particles is kept on each process.

When the implementation is instantiated, an MPI communicator is passed to it.
%
This communicator is duplicated and several parameters, such as
the processes rank and the number of processes are determined and
set as instance variables.
%
At this point, if $P > N$, the implementation will output an error message
and exit the simulation.
%
Each process will allocate a list of $N$ particles here.

The \individualoperation{} method is implemented by having each process update
its local list of particles wth the input function.
%
No communications are performed in this method.

The \pairoperation{} method is implemented by having each process
determine a chunk of $N/P$ it should update.
%
Each process then independently updates those $N/P$ particles.
%
All processes then perform an \mpiallgatherv{} to share these
updates and ensure their list of particles is up to date.


\subsection{The Systolic Distribution}

%
%Q: How is the systolic loop scheme implemented?
The systolic distribution is an implementation of the \systolicloop{} scheme.
%
In this implementation, three lists of $N/P$ particles are allocated
on each process.
%
These lists are referred to as the local list (which corresponds to
the particles list from the abstract distribution), the send list and the
receive list.

Upon instantiation, an MPI communicator is passed to the class
which is duplicated and stored as an instance variable.
%
Several parameters, such as the current rank and the number of
processes are also set as instance variables.
%
If $P > N$, the implementation will output an error message
here and exit the simulation.
%
At this stage, the implementation determines how large a chunk of
memory it will need to store its local list, by rounding up
the result of $N/P$.
%
The three lists of particles mentioned are allocated at this point.

The \individualoperation{} method is implemented by having each process
update its local list of particles.
%
The send and receive lists are ignored in this method.

\begin{lstlisting}[
    caption=The \pairoperation{} implementation for the Systolic Loop.,
    label=lst:systolic_loop_pair_operation_implementation,
    gobble=3
]
    this%receive_list = this%particles

    do i=1, this%num_local_particles
        partial_reductions(:,i) = reduction_init
    end do

    do pulse=1, this%nprocs
        if(pulse .NE. 1) then
            this%do_systolic_pulse
        end if

        do i=1, this%num_local_particles
            do j=1, this%num_receive_particles
                if(                               &
                    this%receive_rank             &
                        .EQ. this%local_ring_rank &
                    .AND.                         &
                    i .EQ. j                      &
                ) cycle

                call pair_to_value(       &
                    this%particles(i),    &
                    this%receive_list(i), &
                    tmp_value             &
                )

                partial_reductions(:, i) = reduction_type( &
                    partial_reductions(:, i), tmp_value    &
                )
            end do
        end do
    end do

    do i=1, this%num_local_particles
        this%particles(i) = particle_value_to_value(    &
            this%particles(i), partial_reductions(:, i) &
        )
    end do
\end{lstlisting}

The \pairoperation{} method is outlined in
\LST{lst:systolic_loop_pair_operation_implementation}.
It is implemented by using the three
particle lists to perform systolic pulses.
%
On first call, this method determines the dimensions of the
reduction variable needed.
%
In the case of determining forces, this will be of size 3.
%
This is determined by finding the size of the ``reduction\_init''
variable.
%
It then allocates an array of doubles of this size multiplied by the
number of particles in its local list.
%
This list is then used to store partial reduction results across
systolic pulses.

Initially, a process will copy its local list to the receive list,
skip performing a systolic pulse for that iteration, and perform
a comparison loop over the local and receive lists, saving the
reduction results into the partial reductions array.
%
A process will then perform a systolic pulse, in which it copies its
current foreign list into its send list, and sends the send list to
the next process in the ring while waiting for data to arrive 
into the receive list from the previous process in the loop.
%
The communications are performed using an \mpisendrecv{}.

Each process will then perform a comparison loop using this new receive
array, reducing the results with the values already in the partial
reductions array.
%
After performing $P$ such comparisons and $P-1$ systolic pulses,
the results in the partial reductions array will be the correct
values to be set on the local particle list.
%
The implementation then uses the particle\_to\_val function
passed to the method to update
its list of local particles with these fully reduced values.


\subsection{Shared And Replicated Distribution}

%
%Q: How is the shared and replicated data scheme implemented?
The shared and replicated distribution is an implementation of the
shared and replcated data scheme.
%
This implementation inherits directly from the replicated data
implementation, and overloads the initialisation and pair operation
methods.

The initialisation routine is overloaded to ensure that
$P_{MPI}*P_{OMP} \le{} N$, where $P_{MPI}$ is the number of MPI
processes the simulation is running on and $P){OMP}$ is the
number of requested \openmp{} threads per MPI process.
%
If $P_{MPI}*P_{OMP} > N$, the implementation will output an error message
and exit the simulation.
%
The implementation then calls the initialisation method of the
replicated data scheme.

The \pairoperation{} method is implemented exactly the same
as in the \replicateddata{} case, except a basic \openmp{}
parallel do region is defined, as outlined in
\LST{lst:shared_and_replicated_omp_region}.

\begin{lstlisting}[
    caption=\openmp{} directives in the shared and replicated distribution.,
    label=lst:shared_and_replicated_omp_region,
    gobble=3
]
    !$OMP PARALLEL DO                                  &
    !$OMP&  DEFAULT(none)                              &
    !$OMP&  PRIVATE(reduce_val, tmp_val, i, j)         &
    !$OMP&  SHARED(                                    &
    !$OMP&      reduction_identity, this, reduce_func, &
    !$OMP&      i_start, i_end                         &
    !$OMP&  )
    !
    do i=i_start, i_end
        reduce_val = reduction_init

        do j=1, this%num_particles
            if(i .EQ. j) cycle

            call pair_to_val(                         &
                this%particles(i), this%particles(j), &
                tmp_val                               &
            )

            reduce_val = reduce_func(reduce_val, tmp_val)

        end do

        this%particles(i) = val_to_particle( &
            this%particles(i), reduce_val    &
        )
    end do
    !
    !$OMP END PARALLEL DO

    call this%sync_particles
\end{lstlisting}


\subsection{Replicated Systolic Distribution}

The replicated systolic distribution is an implementation of the
replicated systolic loop.
%
In this implementation, much like the systolic distribution,
three lists of particles of size $N/S$ are allocated on each process,
referred to as the local list (corresponding to the particles list
in the abstract distribution), the send list and the receive list.

Upon instantiation, an MPI communicator is passed to the constructor.
%
The \mpidimscreate{} function is used to create a 2d decomposition
of the processes represented by the communicator passed to the
constructor.
%
These dimensions are rearranged such that the $x$ dimension of the
resulting distribution should be greater than or equal to the $y$
dimension.
%
These dimensions will correspond to the sizes of $S$ and $R$ for
the replicated systolic scheme.
%
This communicator is used to create a 2d cartesian communicator whose
dimensions are derived from the 2d decomposition that has been generated.

This cartesian communicator is then queried to find the coordinates of
the process.
%
Processes will then use \mpicommsplit{} on the cartesian communicator
to create a new communicator using their $x$ coordinate as the color.
%
This new communicator should be of size $S$ and
is stored in an instance variable as the local ring communicator.
%
Properties of the local ring communicator such as the processes rank
and the number of processes are stored as instance variables.

Processes will then use \mpicommsplit{} on the cartesian communicator
again, but this time use their rank in the local ring communicator as
the color.
%
In this manner, each process representing systolic element $i$ should
be represented by this communicator.
%
This new communicator should be of size $R$ and is stored in an instance
variable as the equivalent element communicator.
%
Properties of the equivalent element communicator such as the process rank
and the number of processes are stored as instance variables.

After this communicator setup has taken place, the instance will determine
the size of the $N/S$ local particle list, and allocate the local particle
list, the send particle list and the receive particle list.

The \individualoperation{} method is implemented by having each process update
its local particle list using the update function passed to it.
%
No communications are performed in this method and the send and receive
lists are ignored.

\begin{lstlisting}[
    caption=Replicated Systolic \pairoperation{} implementation.,
    label=lst:replicated_systolic_pair_operation_implementation,
    gobble=3
]
    ! Find pulse chunk sizes and boundaries
    call this%get_pulse_bounds(            &
        pulse_size, pulse_start, pulse_end &
    )

    ! Find initial swap ranks from pulse chunk boundaries
    call this%get_pulse_offsets(                  &
        pulse_start, pulse_end,                   &
        initial_send, initial_receive             &
    )

    ! Do the initial swap
    call MPI_Sendrecv(                                    &
        this%particles, chunk_size, this%MPI_particle,    &
        initial_send, 0,                                  &
        this%receive_list, chunk_size, this%MPI_particle, &
        initial_receive, 0,                               &
        this%local_ring_comm, MPI_STATUS_IGNORE, ierror   &
    )

    ! Initialise the partial reductions array
    do i=1, this%num_local_particles
        partial_reductions(:,i) = reduction_init
    end do

    ! Loop over systolic pulses
    do pulse=pulse_start, pulse_end

        ! Do the systolic pulse
        if(pulse .NE. pulse_star) then
            call this%do_systolic_pulse
        end if

        do i=1, this%num_local_particles
            do j=1, this%num_receive_particles

                ! Ensure a particle isn't compared
                ! with itself
                if(                               &
                    this%receive_rank             &
                        .EQ. this%local_ring_rank &
                    .AND.                         &
                    i .EQ. j                      &
                ) cycle

                ! Do comparison
                call pair_to_value(       &
                    this%particles(i),    &
                    this%receive_list(i), &
                    tmp_value             &
                )

                ! Reduce into the reductions array
                partial_reductions(:, i) = reduction_type( &
                    partial_reductions(:, i), tmp_value    &
                )
            end do
        end do
    end do


    ! Reduce over all the partial systolic pulse results
    call MPI_Allreduce(                                    &
        partial_reductions, final_partial_reductions,      &
        size(reduction_identity)*this%num_local_particles, &
        MPI_REAL_P, MPI_reduction_type,                    &
        this%equiv_elem_comm, ierror                       &
    )


    ! Set the local particles with the fully reduced results
    do i=1, this%num_local_particles
        this%particles(i) = particle_value_to_value(    &
            this%particles(i), partial_reductions(:, i) &
        )
    end do
\end{lstlisting}

The \pairoperation{} method, outlined in
\LST{lst:replicated_systolic_pair_operation_implementation},
is implemented by having each ring
determine the chunk of $S/R$ pulses it should be performing.
%
For replica $i$, for example, it should be performing pulses
$iS/R$ to $(i+1)S/R$.
%
This is realised by having each element in the ring copy its local
particles to its send list.
%
It then performs an \mpisendrecv{}, sending its send list to
the process $iS/R$ ahead and receiving to its receive list from
the process $iS/R$ behind.
%
This step looks as though $iS/R$ systolic pulses have been performed,
when this is actually performed with one communication.
%
Each process in ring $i$ will then perform $S/R$ comparison loops
with $S/R-1$ systolic pulses.

When the method is first called, an array of partial reduction
values is allocated to store the partial results across pulses.
%
This is implemented exactly as in the systolic distribution,
except instead of being of length $N/P$, it is of length $N/S$.

After performing the required pulses, each process will then perform
an \mpiallreduce{} across the equivalent element communicator.
%
This should result in a reduction over all the partial reduction
results for all of the systolic pulses performed.
%
The resulting array should be the final values to be set on each particle.
%
This array is then used to set the final values on the particles.

\section{Performance Measurement}

%
%Q: What should we be measuring?
In a parallel MD simulation, the time for a single step is strongly dependent
on both the number of particles present and
the number of parallel processes the simulation is running on.
%
This suggests a 2D space over which the performance of the implementation
might be evaluated.

%
%Q: What will we measure?
As successful MD simulations can be carried out on well defined
systems sizes, interesting results can be derived with relatively
small systems.
%
As such, several fixed system sizes will be tested, namely
$N = 2^{9} = 512$,
$N = 2^{12} = 4096$ and
$N = 2^{15} = 32768$
particles.
%
These are chosen mainly so effects can be studied where the number of cores
used is small compared to the system, large compared to the system
and comparable to the system.

The system sizes were also chosen as communication times
tended to surpass calculation times within $P = 32768$ cores
being used.
%
The minimum system size was chosen to reflect a number of particles
exceeding the number of cores on a single \hector{} node and yet far
below the total number of cores on the machine.
%
The maximum system size was chosen both for time to completion for
a single repetition, and for its size being comparable to the number of
cores on \hector{}.

%
%Q: How will we measure these?
A very simple model of MD is used to distil the fundamental calculation
from the implementation.
%
Specifically, the simulation will perform a loop consisting of two
individual operations over the particles representing the first
and second halves of the \velocityverlet{} integration scheme and one
pair operation representing the force evaluation section of the scheme.
%
The loop will be run several times to extract some statistical significance
from the timings.
%
In particular, loops will be required to perform at least 5 iterations
and have a run time comparable to 1 second.

%
%Q: Will we measure IO?
No I/O will be performed during a benchmark.
%
While this is not representative of a real MD code, and indeed I/O
is itself an important factor in the performance of an MD simulation,
it is left outside the scope of this discussion.
%
I/O is implemented primarily for testing and debugging purposes.

%
%Q: How accurate will the benchmarked simulations be?
As the implementations use particle list decomposition rather
than domain decomposition, the placement of particles on
processes is not tied to the simulation.
%
This means, to accurately measure the run time of the system,
it is not necessary to perform a physically accurate simulation.
%
As such, each method may be run separately.
%
As the \individualoperation{} and \pairoperation{} methods will
be tested separately, the MD simulation steps outlined in
\FIG{fig:phases_of_md_simulation}
will not be followed in full.
%
In particular, the initialisation and equilibriation steps will
be skipped.
%
The particle distribution may be initialised with arbitrary data.


\subsection{Benchmarking Plan}

Benchmarks will be performed on \hector{} Phase 3 using
32 cores per node with 32~GB of memory available to each node.
%
Each benchmark will be performed on the range of $[1,32768]$ cores
in powers of 2.
%
That is, benchmarks will be performed on
$1$, $2$, $4$, $8$, $\dots{}$, $32768$ cores.
%
Measurements will be made of the time taken for $n$ repetitions of a given
method where $n \ge{} 5$ and the time $t \sim{} 1$~second.
%
This is performed by timing 5 repetitions of the code.
%
If this time is less than 0.1~seconds, the repetition count is
multiplied by ten, and the code timed again.
%
This is repeated until until the time measured is greater than 0.1~seconds.
%
These measurements will be taken several times, at different times of
day and the lowest recorded time used here.

Measurements for implementations of
the \individualoperation{} and \pairoperation{} method
for each parallelisation scheme will be made.
%
These will be measured in the cases where both MPI and calculations
are performed, measuring the total execution time of the method;
the case where MPI communications are skipped and calculations are
performed, measuring the calculation time of the method; and
the case where calculations are skipped and MPI communications are
performed, measuring the communication time of the method.

Four distributions will be benchmarked:
%
the replicated distribution; the systolic distribution;
the shared and replicated distribution; and
the replicated systolic distribution.

Each implementation is left relatively unoptimised to ensure
the clarity of the implementation although
some thought has been put into the code to avoid inappropriately slow code.
%
These are compiled using crayftn and mpich2, using the ftn compiler with
no flags specified on the command line, beyond that needed to compile
the code.
%
MPI routines are implemented using blocking calls to produce the worst
case communication time.

\begin{table}
    \begin{tabular}{|l|l|}
        \hline
        Machine & \hector{} Phase 3 \\
        Cores Per Node & 32 \\
        Memory Per Node & 32~GB \\
        \hline
        Schemes Tested & \replicateddata{}, \systolicloop{}, \\
                       & \sharedandreplicateddata{},
                         \replicatedsystolicloop{} \\
        Methods Tested & \individualoperation{}, \pairoperation{} \\
        Regimes Tested & Total Time, MPI Only, Calculation Only \\
        Core Range Tested & 1-32768 \\
        Minimum Measurements & 5 repetitions and 0.1 seconds \\
        \hline
    \end{tabular}
    \caption{
        Summary of benchmarks to be performed and the hardware and
        configuration they will be performed on.
    }
    \label{table:benchmark_configuration}
\end{table}


\chapter{Parallel Implementations}
\label{sec:parallel_implementations}

Parallel implementations are required to be functionally equivalent
to the example serial implementations outlined in \SEC{sec:methodology:subsec:implementation}.
%
Tests may then be implemented straightforwardly for the serial implementation,
and are then easily extended to the parallel implementations.

The prescribed interface discourages the use of optimisations which make
assumptions about the MD algorithm being implemented.
%
If it were known ahead of time, for example, that
only forces would be updated during a \pairoperation{},
as is the case in the \velocityverlet{} algorithm,
or that inter-atomic forces dropped to zero after a given distance,
the implementation may be able to use this information to improve
performance through specialized data layouts or communications patterns.

Optimizations for particular algorithms and MD systems are
beyond the scope of this dissertation.
%
The focus here is on general communication patterns rather than
the optimization of particular MD algorithms.

Implementations of the \replicateddata{} and \systolicloop{} schemes
using only MPI will be analysed and discussed by focusing on
the scaling results of the \individualoperation{} and \pairoperation{} methods.


\subsection{Replicated Data}

The replicated scheme allocates a list of particles the size of
the entire MD system of particles
on each process.
%
It uses this list to keep an up-to-date copy of the system of particles
on every process.

In this section, the implementation details and performance of
of the \individualoperation{} and \pairoperation{} methods
for the replicated data scheme using only MPI will be analysed and discussed.


%
% Replicated individual_operation v0
%

\subsubsection{Implementation of the \individualoperation{} Method}

The \individualoperation{} method as outlined in
\SEC{sec:the_individual_operation_method}
is implemented by having each process update its entire local list.
%
As such, it closely resembles the example serial implementation.
%
This approach involves more computation than having each process
evaluate a subsection of the list and share the result with the
other processes, but it avoids a global synchronization.
%
As each process performs an $\bigO{1}$ operation on $N$ particles with
no communications,
this implementation is expected to take a time
\begin{equation}
\label{eqn:v0_replicated_individual_operation_overall_time}
    N\bigO{1} = \bigO{N}
\end  {equation}

\begin{figure}[!h]
    \input{parallel_implementation/v0/replicated.individual_operation.512.time.plt}
    \caption{\vZeroTimeCaption{Replicated Data}{\individualoperation{}}{512}}
    \label{fig:v0_replicated_individual_operation_512_time}
\end  {figure}

\begin{figure}[!h]
    \input{parallel_implementation/v0/replicated.individual_operation.4096.time.plt}
    \caption{\vZeroTimeCaption{Replicated Data}{\individualoperation{}}{4096}}
    \label{fig:v0_replicated_individual_operation_4096_time}
\end  {figure}

\begin{figure}[!h]
    \input{parallel_implementation/v0/replicated.individual_operation.32768.time.plt}
    \caption{\vZeroTimeCaption{Replicated Data}{\individualoperation{}}{32768}}
    \label{fig:v0_replicated_individual_operation_32768_time}
\end  {figure}


\vZeroTimeExplanation
    { \FIG{fig:v0_replicated_individual_operation_512_time} }
    { \FIG{fig:v0_replicated_individual_operation_4096_time} }
    { \FIG{fig:v0_replicated_individual_operation_32768_time} }
    { \individualoperation{} }


From these, it is clear that
the time for the \individualoperation{} doesn't scale with the number
of cores and that MPI takes up no time as this method uses no MPI calls.

There is an interesting increase of time that occurs at 2 cores and again
at 8 cores.
%
Given that \hector{} has 4 NUMA regions of 8 cores and each of those
regions is further subdivided into 4 NUMA regions of 2 cores,
it is likely this is a cause for the jumps at 2 and 8 cores.
%
This is particularly noticeable in
\FIG{fig:v0_replicated_individual_operation_32768_time}
where the jump occurs at exactly the same core count, but is noticeably larger.

The jumps occur at the same numbers of processes in
\FIG{fig:v0_replicated_individual_operation_512_time},
\FIG{fig:v0_replicated_individual_operation_4096_time} and
\FIG{fig:v0_replicated_individual_operation_32768_time}
and after 8 processes, the time remains roughly constant.
%
This suggests either a latency effect with processes accessing memory
in other NUMA regions or a memory bandwith effect.
%
Given that the effect scales roughly with the number of particles
in the system, at fixed core counts, it is most likely due to
memory bandwidth saturation.
%
If bandwidth saturated, it is expected that the time taken for
data transfer would
scale linearly with the size of the data per core being transferred.
%
Therefore, it is likely that the bandwidth is not saturated in the 2 core NUMA
region when only 1 core is in use, and
similarly that the bandwidth for the 8 core NUMA region is not saturated
when only 7 cores are in use.

Above 8 cores, the overall execution time for
the \individualoperation{} method for the Replicated Data implementation
appears to scale as $\bigO{N}$
as predicted in \EQN{eqn:v0_replicated_individual_operation_overall_time}.


%
% Replicated pair_operation v0
%

\subsubsection{Implementation of the \pairoperation{} Method}

The \pairoperation{} method as outlined in
\SEC{sec:the_pair_operation_method}
is implemented by having each process evaluating the pair
comparisons for a subset of the particles and sharing the
results with the other processes.

Each process is designated a subset of the list of particles for which
it will evaluate the results of the comparison and reduction.
%
This looks similar to parallelising the outer loop of the example
serial implementation.
%
The number of particles each process is assigned is roughly $N/P$.
%
Each of these $N/P$ particles are compared to $N$ other particles
using an $\bigO{1}$ operation.
%
The time for one of the $N/P$ particles to be updated is
\begin{equation}
    N\bigO{1} = \bigO{N}
\end  {equation}
and so, the time for all $N/P$ particles to be updated is
\begin{equation}
    \frac{N}{P}\bigO{N} = \bigO{\frac{N^2}{P}}
\end  {equation}
As such, a calculation term of $\bigO{N^2/P}$ is expected.

After the each process finishes updating it's section of the list,
the updated sections of lists are shared across processes using
an MPI\_Allgatherv.
This introduces a communication term of $\bigO{(N + l)\log{P}}$
where $l$ represents a constant latency.

The overall execution time of this method should therefore be
\begin{align}
    \bigO{\frac{N^2}{P}} + \bigO{(N+l)\log{P}}
        &\approx{} \bigO{\frac{N^2}{P} + (N+l)\log{P}} \\
        &\approx{} \bigO{\frac{N^2}{P} + (N+l)\log{P}} \\
        &\approx{} \bigO{\frac{N^2}{P} + N\log{P} + \log{P}} \\
        \label{eqn:v0_replicated_pair_operation_overall_time}
        &\approx{} \bigO{\frac{N^2}{P} + N\log{P}}
\end  {align}

\begin{figure}[!h]
    \input{parallel_implementation/v0/replicated.pair_operation.512.logtime.plt}
    \caption{\vZeroTimeCaption{Replicated Data}{\pairoperation{}}{512}}
    \label{fig:v0_replicated_pair_operation_512_logtime}
\end  {figure}

\begin{figure}[!h]
    \input{parallel_implementation/v0/replicated.pair_operation.4096.logtime.plt}
    \caption{\vZeroTimeCaption{Replicated Data}{\pairoperation{}}{4096}}
    \label{fig:v0_replicated_pair_operation_4096_logtime}
\end  {figure}

\begin{figure}[!h]
    \input{parallel_implementation/v0/replicated.pair_operation.32768.logtime.plt}
    \caption{\vZeroTimeCaption{Replicated Data}{\pairoperation{}}{32768}}
    \label{fig:v0_replicated_pair_operation_32768_logtime}
\end  {figure}

\vZeroTimeExplanation
    { \FIG{fig:v0_replicated_pair_operation_512_logtime} }
    { \FIG{fig:v0_replicated_pair_operation_4096_logtime} }
    { \FIG{fig:v0_replicated_pair_operation_32768_logtime} }
    { \pairoperation{} }

In \FIG{fig:v0_replicated_pair_operation_512_logtime},
\FIG{fig:v0_replicated_pair_operation_4096_logtime} and
\FIG{fig:v0_replicated_pair_operation_32768_logtime},
it is apparent that scaling begins to drop off as the number
of processes used is comparable to the number of particles in the system.
%
From \EQN{eqn:v0_replicated_pair_operation_overall_time} when $P \ll{} N$
\begin{equation}
    \bigO{\frac{N^2}{P} + N\log{P}} \sim{} \bigO{\frac{N^2}{P}}
\end  {equation}
suggesting good scaling in this regime.
%
Similarly when $P \sim{} N$
or $P > N$
\begin{align}
    \bigO{\frac{N^2}{P} + N\log{P}}
        &\sim{} \bigO{N + N\log{P}} \\
        &\sim{} \bigO{N\log{P}}
\end  {align}
%
suggesting the communication term eventually dominating and
the overall time increasing as a function of $P$.

Indeed, examining where
\FIG{fig:v0_replicated_pair_operation_512_logtime},
\FIG{fig:v0_replicated_pair_operation_4096_logtime},
\FIG{fig:v0_replicated_pair_operation_32768_logtime} and
begin straying away from linear scaling,
it can be seen that the minimum execution time
of 512 particles is approximately $10^{-4}$ seconds,
of 4096 particles is approximately $10^{-3}$ seconds and
of 32768 particles is approximately $10^{-2}$ seconds.
%
Thus as scaling begins dropping off at $P = N$,
the minimum execution time for this distribution,
for the system sizes tested,
scales as $N\log{N}$
where $N$ is the number of particles in the system.


\subsection{Systolic Loop}

The systolic loop scheme is initialised by allocating 3 arrays of
size $N/P$ on every process.
%
The system is then split roughly evenly across all the processes
and is held, updated and shared using these three lists.


%
% Systolic individual_operation v0
%

\subsubsection{Implementation of the \individualoperation{} Method}
This is implemented by having each process update its local list
of particles.

This should take $\bigO{N/P}$ time.

\begin{figure}[!h]
    \input{parallel_implementation/v0/systolic.individual_operation.512.logtime.plt}
    \caption{\vZeroTimeCaption{Systolic Loop}{\individualoperation{}}{512}}
    \label{fig:v0_systolic_individual_operation_512_logtime}
\end  {figure}

\begin{figure}[!h]
    \input{parallel_implementation/v0/systolic.individual_operation.4096.logtime.plt}
    \caption{\vZeroTimeCaption{Systolic Loop}{\individualoperation{}}{4096}}
    \label{fig:v0_systolic_individual_operation_4096_logtime}
\end  {figure}

\begin{figure}[!h]
    \input{parallel_implementation/v0/systolic.individual_operation.32768.logtime.plt}
    \caption{\vZeroTimeCaption{Systolic Loop}{\individualoperation{}}{32768}}
    \label{fig:v0_systolic_individual_operation_32768_logtime}
\end  {figure}

As seen in
\FIG{fig:v0_systolic_individual_operation_512_logtime},
\FIG{fig:v0_systolic_individual_operation_4096_logtime} and
\FIG{fig:v0_systolic_individual_operation_32768_logtime}
the current implementation satisfies this.

There appear to be a few unexpected data points in the mpi only timings,
but these are unlikely to be anything except an error.
%
As these are approaching nanosecond execution times, it is
unsurprising that they may pick up unexpected errors as the system clock
has at most a nanosecond resolution time.
%
Given the processor operates on about this time, it is also unsurprising
if some odd times may be picked up from taking measurements so
close to this time scale.


%
% Systolic pair_operation v0
%

\subsubsection{Implementation of the \pairoperation{} Method}
This is implemented by having each process use three lists of particles,
each of size $P/N$.

The first list is the processes local list of particles.

The second list will be referred to as the foreign list, and
represents a list originating from another process.

The third list is a swap list to allow a process to receive a new
foreign list list from the right
while also sending its old foreign list to the left
during a systolic pulse.

Initially, a process will copy its local list to the foreign list
and perform a partial force update on its local list using this
foreign list.
%
The system will then perform a systolic pulse.
After a systolic pulse, every foreign list should move one process
to the left in the systolic loop.
%
This is performed by copying the old foreign list into the
swap list, and using an MPI\_sendrecv to send the swap list to
the left process while receiving from
the right process into the foreign list.
%
When a new foreign list is received, another partial force update
is performed on the local list.
%
This process is repeated $P-1$ times.

Each list comparison between systolic pulses should take $\bigO{(N/P)^2}$ time.
%
For a given timestep, there should be $P$ of these list comparisons
performed, giving an over calculation time of $\bigO{N^2/P}$.

Given each pulse should be passing $N/P$ particles between two processes,
we expect this to take $\bigO{N/P + l}$ time where $l$ is a constant latency.
%
With $P$ pulses on each time step, we expect a communication time of
$\bigO{(N/P + l)P}$.

Combining our calculation and communication terms, the systolic loop approach
should run in $\bigO{N^2/P + dN + dlP}$ time
where $N$ is the number of particles in the system,
$P$ is the number of processes used and
$l$ is a constant latency and
$d$ is a constant.

\begin{figure}[!h]
    \input{parallel_implementation/v0/systolic.pair_operation.512.logtime.plt}
    \caption{\vZeroTimeCaption{Systolic Loop}{\pairoperation{}}{512}}
    \label{fig:v0_systolic_pair_operation_512_logtime}
\end  {figure}

\begin{figure}[!h]
    \input{parallel_implementation/v0/systolic.pair_operation.4096.logtime.plt}
    \caption{\vZeroTimeCaption{Systolic Loop}{\pairoperation{}}{4096}}
    \label{fig:v0_systolic_pair_operation_4096_logtime}
\end  {figure}

\begin{figure}[!h]
    \input{parallel_implementation/v0/systolic.pair_operation.32768.logtime.plt}
    \caption{\vZeroTimeCaption{Systolic Loop}{\pairoperation{}}{32768}}
    \label{fig:v0_systolic_pair_operation_32768_logtime}
\end  {figure}

We see in 
\FIG{fig:v0_systolic_pair_operation_512_logtime},
\FIG{fig:v0_systolic_pair_operation_4096_logtime} and
\FIG{fig:v0_systolic_pair_operation_32768_logtime}
that, much like in the replicated case, that the system scales
roughly as $N/P$ when $P \ll{} N$.

With a communication term scaling as $P$, we see the point in which
communications dominate comes much sooner.
%
However, we also note that it appears when $P \approx{} N$.
%
Taking our previous approach of finding an optimum $k$ for $P = N/k$,
we find $k \approx{} 8$.

This appears to hold for our three system sizes, although there does appear
to be an unexpected jump in
\FIG{fig:v0_systolic_pair_operation_32768_logtime}
between 512 and 1024 processes.
%
It is unclear whether this is a genuine effect, or simply an error in
measurement.

Ignoring the unexpected jump in 
\FIG{fig:v0_systolic_pair_operation_32768_logtime},
we may conclude that the minimum time to completion for out system
scales linearly with the number of particles.

It would appear the implimentation ultimately becomes slowed due to
communication latency, and in particular, due to a large number
of these communications.
%
The initial scaling of this term is due to the number of particles
in the system, however,
scaling with $P$ is a result of having $P$ communications
per time step.
%
Looking at our derivation for communication times,
we conclude that this must be an effect of latency.

\section{Shared and Replicated Data}

The \sharedandreplicateddata{} scheme is implemented in exactly the same
way as the \replicateddata{} scheme outlined in 
\SEC{sec:replicated_data_implementation},
except the update loop in the \pairoperation{} method is further
parallelised using \openmp{} directives, taking advantage of shared
memory between cores.
%
In fact, this distribution class inherits directly from the
\replicateddata{} distribution class, and overloads only the
\pairoperation{} method.


The primary motivation for this is to show that mixed mode MPI and \openmp{}
paralellism is just as viable as MPI only parallelism for a \replicateddata{}
scheme along with how easily the mixed mode parallelism may be implemented
on top of an MPI implementation.
%
The importance of this is that the maximum system size per core per node
can be increased in proportion to the number of \openmp{} threads
created per MPI process, which is of particular importance for nodes with
particularly high core counts.

The \sharedandreplicateddata{} distribution is initialised by
allocating a list of particles the size of the system of particles
on every MPI process.
%
The number of \openmp{} threads used per MPI processes is fixed at 8,
as this is the suggested number for \hector{} due to the arrangement
of the \numa{} regions into groups of 8.
%
As this is a rather low number compared to the overall number of MPI
processes, the emphasis here isn't to gain any perticular performance
improvement using \openmp{}, only to show that it is viable.
%
Indeed, the mere introduction of 8 threads per MPI process should increase
the maximum system size that can be run 8 fold.

In this section,
the implementation details and performance of
the \individualoperation{} and \pairoperation{} methods
will be presented and analysed.


\subsection{\individualoperation{}}

The \individualoperation{} method is inherited directly from
the implementation outlined in
\SEC{sec:replicated_data_individual_operation_implementation}.
%
As a result, the time to completion is also expected to scale as $\bigO{N}$.

%
% Overall speedup plot
%
\begin{figure}[!h]
    \input{parallel_implementation/v1/shared_and_replicated.individual_operation.logspeedup.plt}
    \caption{
        Speedup plots for the \individualoperation{} implemented with the \sharedandreplicateddata{} scheme for systems of particles of size 512, 4096 and 32768.
    }
    \label{fig:v1_shared_and_replicated_data_individual_operation_speedups}
\end{figure}


%
% Individual breakdowns
%
\begin{figure}[!h]
    \input{parallel_implementation/v1/shared_and_replicated.individual_operation.512.time.plt}
    \caption{\vZeroTimeCaption{\sharedandreplicateddata{}}{\individualoperation{}}{512}}
    \label{fig:v1_shared_and_replicated_individual_operation_512_time}
\end  {figure}

\begin{figure}[!h]
    \input{parallel_implementation/v1/shared_and_replicated.individual_operation.4096.time.plt}
    \caption{\vZeroTimeCaption{\sharedandreplicateddata{}}{\individualoperation{}}{4096}}
    \label{fig:v1_shared_and_replicated_individual_operation_4096_time}
\end  {figure}

\begin{figure}[!h]
    \input{parallel_implementation/v1/shared_and_replicated.individual_operation.32768.time.plt}
    \caption{\vZeroTimeCaption{\sharedandreplicateddata{}}{\individualoperation{}}{32768}}
    \label{fig:v1_shared_and_replicated_individual_operation_32768_time}
\end  {figure}

\vZeroTimeExplanation
    {\FIG{fig:v0_replicated_individual_operation_512_time}}
    {\FIG{fig:v0_replicated_individual_operation_4096_time}}
    {\FIG{fig:v0_replicated_individual_operation_32768_time}}
    {\individualoperation{}}
    {\replicateddata{}}

As can be seen from 
\FIG{fig:v0_replicated_individual_operation_512_time},
\FIG{fig:v0_replicated_individual_operation_4096_time} and
\FIG{fig:v0_replicated_individual_operation_32768_time}
the performace scaling does indeed scale as $\bigO{N}$, as expected
and exactly in line with the performance scaling results of the
\individualoperation{} method in the \replicateddata{} scheme.



\subsection{\pairoperation{}}

The \pairoperation{} method is implemented in a similar manner to the
\pairoperation{} method from the \replicateddata{} scheme with the
addition of \openmp{} directives to further parallelise the
force update loop.

A copy of the system is held on each MPI process, meaning there
are $P_{MPI}$ replicas of the system created.
%
Each MPI process is then assigned $N/P_{MPI}$ particles in that system
to determine forces for.
%
Given that list of $N/P_{MPI}$ particles,
an MPI process spawns $P_{OMP}$ threads
and assigns $1/(P_{MPI} P_{OMP})$ particles to each thread.

Writing $P = P_{MPI} \times{} P_{OMP}$,
each core therefore has $N/P$ particles
for which it must determine the forces.
%
As before, if each particles must be compared to $N$ other particles
using an $\bigO{1}$ update operation, the time to find the force for
a single particles is
\begin{equation}
    N\bigO{1} = \bigO{N}
\end{equation}
And so, the time to find the forces for $N/P$ particles is
\begin{equation}
    \frac{N}{P}\bigO{N} = \bigO{\frac{N^2}{P}}
\end{equation}

After the forces have been determined by each thread, the team of threads
shuts down.
%
This represents a synchronisation point for the threads within the
MPI process.
%
After this, an MPI\_Allgatherv is used over the $P_{MPI}$ processes
to synchronise the updated lists of particles, representing a
$\bigO{N\log{P_{OMP}}}$ operation.

%
% Overall speedup plot
%
\begin{figure}[!h]
    \input{parallel_implementation/v1/shared_and_replicated.pair_operation.logspeedup.plt}
    \caption{
        Speedup plots for the \pairoperation{} implemented with the \sharedandreplicateddata{} scheme for systems of particles of size 512, 4096 and 32768.
    }
    \label{fig:v1_shared_and_replicated_data_pair_operation_speedups}
\end{figure}


%
% Individual breakdowns
%
\begin{figure}[!h]
    \input{parallel_implementation/v1/shared_and_replicated.pair_operation.512.logtime.plt}
    \caption{\vZeroTimeCaption{\sharedandreplicateddata{}}{\pairoperation{}}{512}}
    \label{fig:v1_shared_and_replicated_pair_operation_512_logtime}
\end  {figure}

\begin{figure}[!h]
    \input{parallel_implementation/v1/shared_and_replicated.pair_operation.4096.logtime.plt}
    \caption{\vZeroTimeCaption{\sharedandreplicateddata{}}{\pairoperation{}}{4096}}
    \label{fig:v1_shared_and_replicated_pair_operation_4096_logtime}
\end  {figure}

\begin{figure}[!h]
    \input{parallel_implementation/v1/shared_and_replicated.pair_operation.32768.logtime.plt}
    \caption{\vZeroTimeCaption{\sharedandreplicateddata{}}{\pairoperation{}}{32768}}
    \label{fig:v1_shared_and_replicated_pair_operation_32768_logtime}
\end  {figure}

\section{Shared and Replicated Data}

\subsection{\individualoperation{}}

%
% Overall speedup plot
%
\begin{figure}[!h]
    \input{parallel_implementation/v1/replicated_systolic.individual_operation.logspeedup.plt}
    \caption{
        Speedup plots for the \individualoperation{} implemented with the \replicatedsystolicloop{} scheme for systems of particles of size 512, 4096 and 32768.
    }
    \label{fig:v1_replicated_systolic_loop_individual_operation_speedups}
\end{figure}


%
% Individual breakdowns
%
\begin{figure}[!h]
    \input{parallel_implementation/v1/replicated_systolic.individual_operation.512.logtime.plt}
    \caption{\vZeroTimeCaption{\replicatedsystolicloop{}}{\individualoperation{}}{512}}
    \label{fig:v1_replicated_systolic_individual_operation_512_time}
\end  {figure}

\begin{figure}[!h]
    \input{parallel_implementation/v1/replicated_systolic.individual_operation.4096.logtime.plt}
    \caption{\vZeroTimeCaption{\replicatedsystolicloop{}}{\individualoperation{}}{4096}}
    \label{fig:v1_replicated_systolic_individual_operation_4096_time}
\end  {figure}

\begin{figure}[!h]
    \input{parallel_implementation/v1/replicated_systolic.individual_operation.32768.logtime.plt}
    \caption{\vZeroTimeCaption{\replicatedsystolicloop{}}{\individualoperation{}}{32768}}
    \label{fig:v1_replicated_systolic_individual_operation_32768_time}
\end  {figure}




\subsection{\pairoperation{}}

%
% Overall speedup plot
%
\begin{figure}[!h]
    \input{parallel_implementation/v1/replicated_systolic.pair_operation.logspeedup.plt}
    \caption{
        Speedup plots for the \pairoperation{} implemented with the \replicatedsystolicloop{} scheme for systems of particles of size 512, 4096 and 32768.
    }
    \label{fig:v1_replicated_systolic_pair_operation_speedups}
\end{figure}


%
% Individual breakdowns
%
\begin{figure}[!h]
    \input{parallel_implementation/v1/replicated_systolic.pair_operation.512.logtime.plt}
    \caption{\vZeroTimeCaption{\replicatedsystolicloop{}}{\pairoperation{}}{512}}
    \label{fig:v1_replicated_systolic_pair_operation_512_logtime}
\end  {figure}

\begin{figure}[!h]
    \input{parallel_implementation/v1/replicated_systolic.pair_operation.4096.logtime.plt}
    \caption{\vZeroTimeCaption{\replicatedsystolicloop{}}{\pairoperation{}}{4096}}
    \label{fig:v1_replicated_systolic_pair_operation_4096_logtime}
\end  {figure}

\begin{figure}[!h]
    \input{parallel_implementation/v1/replicated_systolic.pair_operation.32768.logtime.plt}
    \caption{\vZeroTimeCaption{\replicatedsystolicloop{}}{\pairoperation{}}{32768}}
    \label{fig:v1_replicated_systolic_pair_operation_32768_logtime}
\end  {figure}



\section{Initial Implementations}
Here, the implementation details and performance of the
\replicateddata{} and \systolicloop{} schemes will be presented.

\subsection{Replicated Data}

The replicated scheme allocates a list of particles the size of
the entire MD system of particles
on each process.
%
It uses this list to keep an up-to-date copy of the system of particles
on every process.

In this section, the implementation details and performance of
of the \individualoperation{} and \pairoperation{} methods
for the replicated data scheme using only MPI will be analysed and discussed.


%
% Replicated individual_operation v0
%

\subsubsection{Implementation of the \individualoperation{} Method}

The \individualoperation{} method as outlined in
\SEC{sec:the_individual_operation_method}
is implemented by having each process update its entire local list.
%
As such, it closely resembles the example serial implementation.
%
This approach involves more computation than having each process
evaluate a subsection of the list and share the result with the
other processes, but it avoids a global synchronization.
%
As each process performs an $\bigO{1}$ operation on $N$ particles with
no communications,
this implementation is expected to take a time
\begin{equation}
\label{eqn:v0_replicated_individual_operation_overall_time}
    N\bigO{1} = \bigO{N}
\end  {equation}

\begin{figure}[!h]
    \input{parallel_implementation/v0/replicated.individual_operation.512.time.plt}
    \caption{\vZeroTimeCaption{Replicated Data}{\individualoperation{}}{512}}
    \label{fig:v0_replicated_individual_operation_512_time}
\end  {figure}

\begin{figure}[!h]
    \input{parallel_implementation/v0/replicated.individual_operation.4096.time.plt}
    \caption{\vZeroTimeCaption{Replicated Data}{\individualoperation{}}{4096}}
    \label{fig:v0_replicated_individual_operation_4096_time}
\end  {figure}

\begin{figure}[!h]
    \input{parallel_implementation/v0/replicated.individual_operation.32768.time.plt}
    \caption{\vZeroTimeCaption{Replicated Data}{\individualoperation{}}{32768}}
    \label{fig:v0_replicated_individual_operation_32768_time}
\end  {figure}


\vZeroTimeExplanation
    { \FIG{fig:v0_replicated_individual_operation_512_time} }
    { \FIG{fig:v0_replicated_individual_operation_4096_time} }
    { \FIG{fig:v0_replicated_individual_operation_32768_time} }
    { \individualoperation{} }


From these, it is clear that
the time for the \individualoperation{} doesn't scale with the number
of cores and that MPI takes up no time as this method uses no MPI calls.

There is an interesting increase of time that occurs at 2 cores and again
at 8 cores.
%
Given that \hector{} has 4 NUMA regions of 8 cores and each of those
regions is further subdivided into 4 NUMA regions of 2 cores,
it is likely this is a cause for the jumps at 2 and 8 cores.
%
This is particularly noticeable in
\FIG{fig:v0_replicated_individual_operation_32768_time}
where the jump occurs at exactly the same core count, but is noticeably larger.

The jumps occur at the same numbers of processes in
\FIG{fig:v0_replicated_individual_operation_512_time},
\FIG{fig:v0_replicated_individual_operation_4096_time} and
\FIG{fig:v0_replicated_individual_operation_32768_time}
and after 8 processes, the time remains roughly constant.
%
This suggests either a latency effect with processes accessing memory
in other NUMA regions or a memory bandwith effect.
%
Given that the effect scales roughly with the number of particles
in the system, at fixed core counts, it is most likely due to
memory bandwidth saturation.
%
If bandwidth saturated, it is expected that the time taken for
data transfer would
scale linearly with the size of the data per core being transferred.
%
Therefore, it is likely that the bandwidth is not saturated in the 2 core NUMA
region when only 1 core is in use, and
similarly that the bandwidth for the 8 core NUMA region is not saturated
when only 7 cores are in use.

Above 8 cores, the overall execution time for
the \individualoperation{} method for the Replicated Data implementation
appears to scale as $\bigO{N}$
as predicted in \EQN{eqn:v0_replicated_individual_operation_overall_time}.


%
% Replicated pair_operation v0
%

\subsubsection{Implementation of the \pairoperation{} Method}

The \pairoperation{} method as outlined in
\SEC{sec:the_pair_operation_method}
is implemented by having each process evaluating the pair
comparisons for a subset of the particles and sharing the
results with the other processes.

Each process is designated a subset of the list of particles for which
it will evaluate the results of the comparison and reduction.
%
This looks similar to parallelising the outer loop of the example
serial implementation.
%
The number of particles each process is assigned is roughly $N/P$.
%
Each of these $N/P$ particles are compared to $N$ other particles
using an $\bigO{1}$ operation.
%
The time for one of the $N/P$ particles to be updated is
\begin{equation}
    N\bigO{1} = \bigO{N}
\end  {equation}
and so, the time for all $N/P$ particles to be updated is
\begin{equation}
    \frac{N}{P}\bigO{N} = \bigO{\frac{N^2}{P}}
\end  {equation}
As such, a calculation term of $\bigO{N^2/P}$ is expected.

After the each process finishes updating it's section of the list,
the updated sections of lists are shared across processes using
an MPI\_Allgatherv.
This introduces a communication term of $\bigO{(N + l)\log{P}}$
where $l$ represents a constant latency.

The overall execution time of this method should therefore be
\begin{align}
    \bigO{\frac{N^2}{P}} + \bigO{(N+l)\log{P}}
        &\approx{} \bigO{\frac{N^2}{P} + (N+l)\log{P}} \\
        &\approx{} \bigO{\frac{N^2}{P} + (N+l)\log{P}} \\
        &\approx{} \bigO{\frac{N^2}{P} + N\log{P} + \log{P}} \\
        \label{eqn:v0_replicated_pair_operation_overall_time}
        &\approx{} \bigO{\frac{N^2}{P} + N\log{P}}
\end  {align}

\begin{figure}[!h]
    \input{parallel_implementation/v0/replicated.pair_operation.512.logtime.plt}
    \caption{\vZeroTimeCaption{Replicated Data}{\pairoperation{}}{512}}
    \label{fig:v0_replicated_pair_operation_512_logtime}
\end  {figure}

\begin{figure}[!h]
    \input{parallel_implementation/v0/replicated.pair_operation.4096.logtime.plt}
    \caption{\vZeroTimeCaption{Replicated Data}{\pairoperation{}}{4096}}
    \label{fig:v0_replicated_pair_operation_4096_logtime}
\end  {figure}

\begin{figure}[!h]
    \input{parallel_implementation/v0/replicated.pair_operation.32768.logtime.plt}
    \caption{\vZeroTimeCaption{Replicated Data}{\pairoperation{}}{32768}}
    \label{fig:v0_replicated_pair_operation_32768_logtime}
\end  {figure}

\vZeroTimeExplanation
    { \FIG{fig:v0_replicated_pair_operation_512_logtime} }
    { \FIG{fig:v0_replicated_pair_operation_4096_logtime} }
    { \FIG{fig:v0_replicated_pair_operation_32768_logtime} }
    { \pairoperation{} }

In \FIG{fig:v0_replicated_pair_operation_512_logtime},
\FIG{fig:v0_replicated_pair_operation_4096_logtime} and
\FIG{fig:v0_replicated_pair_operation_32768_logtime},
it is apparent that scaling begins to drop off as the number
of processes used is comparable to the number of particles in the system.
%
From \EQN{eqn:v0_replicated_pair_operation_overall_time} when $P \ll{} N$
\begin{equation}
    \bigO{\frac{N^2}{P} + N\log{P}} \sim{} \bigO{\frac{N^2}{P}}
\end  {equation}
suggesting good scaling in this regime.
%
Similarly when $P \sim{} N$
or $P > N$
\begin{align}
    \bigO{\frac{N^2}{P} + N\log{P}}
        &\sim{} \bigO{N + N\log{P}} \\
        &\sim{} \bigO{N\log{P}}
\end  {align}
%
suggesting the communication term eventually dominating and
the overall time increasing as a function of $P$.

Indeed, examining where
\FIG{fig:v0_replicated_pair_operation_512_logtime},
\FIG{fig:v0_replicated_pair_operation_4096_logtime},
\FIG{fig:v0_replicated_pair_operation_32768_logtime} and
begin straying away from linear scaling,
it can be seen that the minimum execution time
of 512 particles is approximately $10^{-4}$ seconds,
of 4096 particles is approximately $10^{-3}$ seconds and
of 32768 particles is approximately $10^{-2}$ seconds.
%
Thus as scaling begins dropping off at $P = N$,
the minimum execution time for this distribution,
for the system sizes tested,
scales as $N\log{N}$
where $N$ is the number of particles in the system.


\subsection{Systolic Loop}

The systolic loop scheme is initialised by allocating 3 arrays of
size $N/P$ on every process.
%
The system is then split roughly evenly across all the processes
and is held, updated and shared using these three lists.


%
% Systolic individual_operation v0
%

\subsubsection{Implementation of the \individualoperation{} Method}
This is implemented by having each process update its local list
of particles.

This should take $\bigO{N/P}$ time.

\begin{figure}[!h]
    \input{parallel_implementation/v0/systolic.individual_operation.512.logtime.plt}
    \caption{\vZeroTimeCaption{Systolic Loop}{\individualoperation{}}{512}}
    \label{fig:v0_systolic_individual_operation_512_logtime}
\end  {figure}

\begin{figure}[!h]
    \input{parallel_implementation/v0/systolic.individual_operation.4096.logtime.plt}
    \caption{\vZeroTimeCaption{Systolic Loop}{\individualoperation{}}{4096}}
    \label{fig:v0_systolic_individual_operation_4096_logtime}
\end  {figure}

\begin{figure}[!h]
    \input{parallel_implementation/v0/systolic.individual_operation.32768.logtime.plt}
    \caption{\vZeroTimeCaption{Systolic Loop}{\individualoperation{}}{32768}}
    \label{fig:v0_systolic_individual_operation_32768_logtime}
\end  {figure}

As seen in
\FIG{fig:v0_systolic_individual_operation_512_logtime},
\FIG{fig:v0_systolic_individual_operation_4096_logtime} and
\FIG{fig:v0_systolic_individual_operation_32768_logtime}
the current implementation satisfies this.

There appear to be a few unexpected data points in the mpi only timings,
but these are unlikely to be anything except an error.
%
As these are approaching nanosecond execution times, it is
unsurprising that they may pick up unexpected errors as the system clock
has at most a nanosecond resolution time.
%
Given the processor operates on about this time, it is also unsurprising
if some odd times may be picked up from taking measurements so
close to this time scale.


%
% Systolic pair_operation v0
%

\subsubsection{Implementation of the \pairoperation{} Method}
This is implemented by having each process use three lists of particles,
each of size $P/N$.

The first list is the processes local list of particles.

The second list will be referred to as the foreign list, and
represents a list originating from another process.

The third list is a swap list to allow a process to receive a new
foreign list list from the right
while also sending its old foreign list to the left
during a systolic pulse.

Initially, a process will copy its local list to the foreign list
and perform a partial force update on its local list using this
foreign list.
%
The system will then perform a systolic pulse.
After a systolic pulse, every foreign list should move one process
to the left in the systolic loop.
%
This is performed by copying the old foreign list into the
swap list, and using an MPI\_sendrecv to send the swap list to
the left process while receiving from
the right process into the foreign list.
%
When a new foreign list is received, another partial force update
is performed on the local list.
%
This process is repeated $P-1$ times.

Each list comparison between systolic pulses should take $\bigO{(N/P)^2}$ time.
%
For a given timestep, there should be $P$ of these list comparisons
performed, giving an over calculation time of $\bigO{N^2/P}$.

Given each pulse should be passing $N/P$ particles between two processes,
we expect this to take $\bigO{N/P + l}$ time where $l$ is a constant latency.
%
With $P$ pulses on each time step, we expect a communication time of
$\bigO{(N/P + l)P}$.

Combining our calculation and communication terms, the systolic loop approach
should run in $\bigO{N^2/P + dN + dlP}$ time
where $N$ is the number of particles in the system,
$P$ is the number of processes used and
$l$ is a constant latency and
$d$ is a constant.

\begin{figure}[!h]
    \input{parallel_implementation/v0/systolic.pair_operation.512.logtime.plt}
    \caption{\vZeroTimeCaption{Systolic Loop}{\pairoperation{}}{512}}
    \label{fig:v0_systolic_pair_operation_512_logtime}
\end  {figure}

\begin{figure}[!h]
    \input{parallel_implementation/v0/systolic.pair_operation.4096.logtime.plt}
    \caption{\vZeroTimeCaption{Systolic Loop}{\pairoperation{}}{4096}}
    \label{fig:v0_systolic_pair_operation_4096_logtime}
\end  {figure}

\begin{figure}[!h]
    \input{parallel_implementation/v0/systolic.pair_operation.32768.logtime.plt}
    \caption{\vZeroTimeCaption{Systolic Loop}{\pairoperation{}}{32768}}
    \label{fig:v0_systolic_pair_operation_32768_logtime}
\end  {figure}

We see in 
\FIG{fig:v0_systolic_pair_operation_512_logtime},
\FIG{fig:v0_systolic_pair_operation_4096_logtime} and
\FIG{fig:v0_systolic_pair_operation_32768_logtime}
that, much like in the replicated case, that the system scales
roughly as $N/P$ when $P \ll{} N$.

With a communication term scaling as $P$, we see the point in which
communications dominate comes much sooner.
%
However, we also note that it appears when $P \approx{} N$.
%
Taking our previous approach of finding an optimum $k$ for $P = N/k$,
we find $k \approx{} 8$.

This appears to hold for our three system sizes, although there does appear
to be an unexpected jump in
\FIG{fig:v0_systolic_pair_operation_32768_logtime}
between 512 and 1024 processes.
%
It is unclear whether this is a genuine effect, or simply an error in
measurement.

Ignoring the unexpected jump in 
\FIG{fig:v0_systolic_pair_operation_32768_logtime},
we may conclude that the minimum time to completion for out system
scales linearly with the number of particles.

It would appear the implimentation ultimately becomes slowed due to
communication latency, and in particular, due to a large number
of these communications.
%
The initial scaling of this term is due to the number of particles
in the system, however,
scaling with $P$ is a result of having $P$ communications
per time step.
%
Looking at our derivation for communication times,
we conclude that this must be an effect of latency.


\section{Improved Implementations}
Here implementation details and performance results of
the \sharedandreplicateddata{} and \replicatedsystolicloop{}
schemes will be presented.
%
These are intended to be improvements upon the \replicateddata{}
and \systolicloop{} schemes respectively.
\section{Shared and Replicated Data}

The \sharedandreplicateddata{} scheme is implemented in exactly the same
way as the \replicateddata{} scheme outlined in 
\SEC{sec:replicated_data_implementation},
except the update loop in the \pairoperation{} method is further
parallelised using \openmp{} directives, taking advantage of shared
memory between cores.
%
In fact, this distribution class inherits directly from the
\replicateddata{} distribution class, and overloads only the
\pairoperation{} method.


The primary motivation for this is to show that mixed mode MPI and \openmp{}
paralellism is just as viable as MPI only parallelism for a \replicateddata{}
scheme along with how easily the mixed mode parallelism may be implemented
on top of an MPI implementation.
%
The importance of this is that the maximum system size per core per node
can be increased in proportion to the number of \openmp{} threads
created per MPI process, which is of particular importance for nodes with
particularly high core counts.

The \sharedandreplicateddata{} distribution is initialised by
allocating a list of particles the size of the system of particles
on every MPI process.
%
The number of \openmp{} threads used per MPI processes is fixed at 8,
as this is the suggested number for \hector{} due to the arrangement
of the \numa{} regions into groups of 8.
%
As this is a rather low number compared to the overall number of MPI
processes, the emphasis here isn't to gain any perticular performance
improvement using \openmp{}, only to show that it is viable.
%
Indeed, the mere introduction of 8 threads per MPI process should increase
the maximum system size that can be run 8 fold.

In this section,
the implementation details and performance of
the \individualoperation{} and \pairoperation{} methods
will be presented and analysed.


\subsection{\individualoperation{}}

The \individualoperation{} method is inherited directly from
the implementation outlined in
\SEC{sec:replicated_data_individual_operation_implementation}.
%
As a result, the time to completion is also expected to scale as $\bigO{N}$.

%
% Overall speedup plot
%
\begin{figure}[!h]
    \input{parallel_implementation/v1/shared_and_replicated.individual_operation.logspeedup.plt}
    \caption{
        Speedup plots for the \individualoperation{} implemented with the \sharedandreplicateddata{} scheme for systems of particles of size 512, 4096 and 32768.
    }
    \label{fig:v1_shared_and_replicated_data_individual_operation_speedups}
\end{figure}


%
% Individual breakdowns
%
\begin{figure}[!h]
    \input{parallel_implementation/v1/shared_and_replicated.individual_operation.512.time.plt}
    \caption{\vZeroTimeCaption{\sharedandreplicateddata{}}{\individualoperation{}}{512}}
    \label{fig:v1_shared_and_replicated_individual_operation_512_time}
\end  {figure}

\begin{figure}[!h]
    \input{parallel_implementation/v1/shared_and_replicated.individual_operation.4096.time.plt}
    \caption{\vZeroTimeCaption{\sharedandreplicateddata{}}{\individualoperation{}}{4096}}
    \label{fig:v1_shared_and_replicated_individual_operation_4096_time}
\end  {figure}

\begin{figure}[!h]
    \input{parallel_implementation/v1/shared_and_replicated.individual_operation.32768.time.plt}
    \caption{\vZeroTimeCaption{\sharedandreplicateddata{}}{\individualoperation{}}{32768}}
    \label{fig:v1_shared_and_replicated_individual_operation_32768_time}
\end  {figure}

\vZeroTimeExplanation
    {\FIG{fig:v0_replicated_individual_operation_512_time}}
    {\FIG{fig:v0_replicated_individual_operation_4096_time}}
    {\FIG{fig:v0_replicated_individual_operation_32768_time}}
    {\individualoperation{}}
    {\replicateddata{}}

As can be seen from 
\FIG{fig:v0_replicated_individual_operation_512_time},
\FIG{fig:v0_replicated_individual_operation_4096_time} and
\FIG{fig:v0_replicated_individual_operation_32768_time}
the performace scaling does indeed scale as $\bigO{N}$, as expected
and exactly in line with the performance scaling results of the
\individualoperation{} method in the \replicateddata{} scheme.



\subsection{\pairoperation{}}

The \pairoperation{} method is implemented in a similar manner to the
\pairoperation{} method from the \replicateddata{} scheme with the
addition of \openmp{} directives to further parallelise the
force update loop.

A copy of the system is held on each MPI process, meaning there
are $P_{MPI}$ replicas of the system created.
%
Each MPI process is then assigned $N/P_{MPI}$ particles in that system
to determine forces for.
%
Given that list of $N/P_{MPI}$ particles,
an MPI process spawns $P_{OMP}$ threads
and assigns $1/(P_{MPI} P_{OMP})$ particles to each thread.

Writing $P = P_{MPI} \times{} P_{OMP}$,
each core therefore has $N/P$ particles
for which it must determine the forces.
%
As before, if each particles must be compared to $N$ other particles
using an $\bigO{1}$ update operation, the time to find the force for
a single particles is
\begin{equation}
    N\bigO{1} = \bigO{N}
\end{equation}
And so, the time to find the forces for $N/P$ particles is
\begin{equation}
    \frac{N}{P}\bigO{N} = \bigO{\frac{N^2}{P}}
\end{equation}

After the forces have been determined by each thread, the team of threads
shuts down.
%
This represents a synchronisation point for the threads within the
MPI process.
%
After this, an MPI\_Allgatherv is used over the $P_{MPI}$ processes
to synchronise the updated lists of particles, representing a
$\bigO{N\log{P_{OMP}}}$ operation.

%
% Overall speedup plot
%
\begin{figure}[!h]
    \input{parallel_implementation/v1/shared_and_replicated.pair_operation.logspeedup.plt}
    \caption{
        Speedup plots for the \pairoperation{} implemented with the \sharedandreplicateddata{} scheme for systems of particles of size 512, 4096 and 32768.
    }
    \label{fig:v1_shared_and_replicated_data_pair_operation_speedups}
\end{figure}


%
% Individual breakdowns
%
\begin{figure}[!h]
    \input{parallel_implementation/v1/shared_and_replicated.pair_operation.512.logtime.plt}
    \caption{\vZeroTimeCaption{\sharedandreplicateddata{}}{\pairoperation{}}{512}}
    \label{fig:v1_shared_and_replicated_pair_operation_512_logtime}
\end  {figure}

\begin{figure}[!h]
    \input{parallel_implementation/v1/shared_and_replicated.pair_operation.4096.logtime.plt}
    \caption{\vZeroTimeCaption{\sharedandreplicateddata{}}{\pairoperation{}}{4096}}
    \label{fig:v1_shared_and_replicated_pair_operation_4096_logtime}
\end  {figure}

\begin{figure}[!h]
    \input{parallel_implementation/v1/shared_and_replicated.pair_operation.32768.logtime.plt}
    \caption{\vZeroTimeCaption{\sharedandreplicateddata{}}{\pairoperation{}}{32768}}
    \label{fig:v1_shared_and_replicated_pair_operation_32768_logtime}
\end  {figure}

\section{Shared and Replicated Data}

\subsection{\individualoperation{}}

%
% Overall speedup plot
%
\begin{figure}[!h]
    \input{parallel_implementation/v1/replicated_systolic.individual_operation.logspeedup.plt}
    \caption{
        Speedup plots for the \individualoperation{} implemented with the \replicatedsystolicloop{} scheme for systems of particles of size 512, 4096 and 32768.
    }
    \label{fig:v1_replicated_systolic_loop_individual_operation_speedups}
\end{figure}


%
% Individual breakdowns
%
\begin{figure}[!h]
    \input{parallel_implementation/v1/replicated_systolic.individual_operation.512.logtime.plt}
    \caption{\vZeroTimeCaption{\replicatedsystolicloop{}}{\individualoperation{}}{512}}
    \label{fig:v1_replicated_systolic_individual_operation_512_time}
\end  {figure}

\begin{figure}[!h]
    \input{parallel_implementation/v1/replicated_systolic.individual_operation.4096.logtime.plt}
    \caption{\vZeroTimeCaption{\replicatedsystolicloop{}}{\individualoperation{}}{4096}}
    \label{fig:v1_replicated_systolic_individual_operation_4096_time}
\end  {figure}

\begin{figure}[!h]
    \input{parallel_implementation/v1/replicated_systolic.individual_operation.32768.logtime.plt}
    \caption{\vZeroTimeCaption{\replicatedsystolicloop{}}{\individualoperation{}}{32768}}
    \label{fig:v1_replicated_systolic_individual_operation_32768_time}
\end  {figure}




\subsection{\pairoperation{}}

%
% Overall speedup plot
%
\begin{figure}[!h]
    \input{parallel_implementation/v1/replicated_systolic.pair_operation.logspeedup.plt}
    \caption{
        Speedup plots for the \pairoperation{} implemented with the \replicatedsystolicloop{} scheme for systems of particles of size 512, 4096 and 32768.
    }
    \label{fig:v1_replicated_systolic_pair_operation_speedups}
\end{figure}


%
% Individual breakdowns
%
\begin{figure}[!h]
    \input{parallel_implementation/v1/replicated_systolic.pair_operation.512.logtime.plt}
    \caption{\vZeroTimeCaption{\replicatedsystolicloop{}}{\pairoperation{}}{512}}
    \label{fig:v1_replicated_systolic_pair_operation_512_logtime}
\end  {figure}

\begin{figure}[!h]
    \input{parallel_implementation/v1/replicated_systolic.pair_operation.4096.logtime.plt}
    \caption{\vZeroTimeCaption{\replicatedsystolicloop{}}{\pairoperation{}}{4096}}
    \label{fig:v1_replicated_systolic_pair_operation_4096_logtime}
\end  {figure}

\begin{figure}[!h]
    \input{parallel_implementation/v1/replicated_systolic.pair_operation.32768.logtime.plt}
    \caption{\vZeroTimeCaption{\replicatedsystolicloop{}}{\pairoperation{}}{32768}}
    \label{fig:v1_replicated_systolic_pair_operation_32768_logtime}
\end  {figure}


\chapter{Conclusion}

Parallel implementations are required to be functionally equivalent
to the example serial implementations outlined in \SEC{sec:methodology:subsec:implementation}.
%
Tests may then be implemented straightforwardly for the serial implementation,
and are then easily extended to the parallel implementations.

The prescribed interface discourages the use of optimisations which make
assumptions about the MD algorithm being implemented.
%
If it were known ahead of time, for example, that
only forces would be updated during a \pairoperation{},
as is the case in the \velocityverlet{} algorithm,
or that inter-atomic forces dropped to zero after a given distance,
the implementation may be able to use this information to improve
performance through specialized data layouts or communications patterns.

Optimizations for particular algorithms and MD systems are
beyond the scope of this dissertation.
%
The focus here is on general communication patterns rather than
the optimization of particular MD algorithms.

Implementations of the \replicateddata{} and \systolicloop{} schemes
using only MPI will be analysed and discussed by focusing on
the scaling results of the \individualoperation{} and \pairoperation{} methods.


\subsection{Replicated Data}

The replicated scheme allocates a list of particles the size of
the entire MD system of particles
on each process.
%
It uses this list to keep an up-to-date copy of the system of particles
on every process.

In this section, the implementation details and performance of
of the \individualoperation{} and \pairoperation{} methods
for the replicated data scheme using only MPI will be analysed and discussed.


%
% Replicated individual_operation v0
%

\subsubsection{Implementation of the \individualoperation{} Method}

The \individualoperation{} method as outlined in
\SEC{sec:the_individual_operation_method}
is implemented by having each process update its entire local list.
%
As such, it closely resembles the example serial implementation.
%
This approach involves more computation than having each process
evaluate a subsection of the list and share the result with the
other processes, but it avoids a global synchronization.
%
As each process performs an $\bigO{1}$ operation on $N$ particles with
no communications,
this implementation is expected to take a time
\begin{equation}
\label{eqn:v0_replicated_individual_operation_overall_time}
    N\bigO{1} = \bigO{N}
\end  {equation}

\begin{figure}[!h]
    \input{parallel_implementation/v0/replicated.individual_operation.512.time.plt}
    \caption{\vZeroTimeCaption{Replicated Data}{\individualoperation{}}{512}}
    \label{fig:v0_replicated_individual_operation_512_time}
\end  {figure}

\begin{figure}[!h]
    \input{parallel_implementation/v0/replicated.individual_operation.4096.time.plt}
    \caption{\vZeroTimeCaption{Replicated Data}{\individualoperation{}}{4096}}
    \label{fig:v0_replicated_individual_operation_4096_time}
\end  {figure}

\begin{figure}[!h]
    \input{parallel_implementation/v0/replicated.individual_operation.32768.time.plt}
    \caption{\vZeroTimeCaption{Replicated Data}{\individualoperation{}}{32768}}
    \label{fig:v0_replicated_individual_operation_32768_time}
\end  {figure}


\vZeroTimeExplanation
    { \FIG{fig:v0_replicated_individual_operation_512_time} }
    { \FIG{fig:v0_replicated_individual_operation_4096_time} }
    { \FIG{fig:v0_replicated_individual_operation_32768_time} }
    { \individualoperation{} }


From these, it is clear that
the time for the \individualoperation{} doesn't scale with the number
of cores and that MPI takes up no time as this method uses no MPI calls.

There is an interesting increase of time that occurs at 2 cores and again
at 8 cores.
%
Given that \hector{} has 4 NUMA regions of 8 cores and each of those
regions is further subdivided into 4 NUMA regions of 2 cores,
it is likely this is a cause for the jumps at 2 and 8 cores.
%
This is particularly noticeable in
\FIG{fig:v0_replicated_individual_operation_32768_time}
where the jump occurs at exactly the same core count, but is noticeably larger.

The jumps occur at the same numbers of processes in
\FIG{fig:v0_replicated_individual_operation_512_time},
\FIG{fig:v0_replicated_individual_operation_4096_time} and
\FIG{fig:v0_replicated_individual_operation_32768_time}
and after 8 processes, the time remains roughly constant.
%
This suggests either a latency effect with processes accessing memory
in other NUMA regions or a memory bandwith effect.
%
Given that the effect scales roughly with the number of particles
in the system, at fixed core counts, it is most likely due to
memory bandwidth saturation.
%
If bandwidth saturated, it is expected that the time taken for
data transfer would
scale linearly with the size of the data per core being transferred.
%
Therefore, it is likely that the bandwidth is not saturated in the 2 core NUMA
region when only 1 core is in use, and
similarly that the bandwidth for the 8 core NUMA region is not saturated
when only 7 cores are in use.

Above 8 cores, the overall execution time for
the \individualoperation{} method for the Replicated Data implementation
appears to scale as $\bigO{N}$
as predicted in \EQN{eqn:v0_replicated_individual_operation_overall_time}.


%
% Replicated pair_operation v0
%

\subsubsection{Implementation of the \pairoperation{} Method}

The \pairoperation{} method as outlined in
\SEC{sec:the_pair_operation_method}
is implemented by having each process evaluating the pair
comparisons for a subset of the particles and sharing the
results with the other processes.

Each process is designated a subset of the list of particles for which
it will evaluate the results of the comparison and reduction.
%
This looks similar to parallelising the outer loop of the example
serial implementation.
%
The number of particles each process is assigned is roughly $N/P$.
%
Each of these $N/P$ particles are compared to $N$ other particles
using an $\bigO{1}$ operation.
%
The time for one of the $N/P$ particles to be updated is
\begin{equation}
    N\bigO{1} = \bigO{N}
\end  {equation}
and so, the time for all $N/P$ particles to be updated is
\begin{equation}
    \frac{N}{P}\bigO{N} = \bigO{\frac{N^2}{P}}
\end  {equation}
As such, a calculation term of $\bigO{N^2/P}$ is expected.

After the each process finishes updating it's section of the list,
the updated sections of lists are shared across processes using
an MPI\_Allgatherv.
This introduces a communication term of $\bigO{(N + l)\log{P}}$
where $l$ represents a constant latency.

The overall execution time of this method should therefore be
\begin{align}
    \bigO{\frac{N^2}{P}} + \bigO{(N+l)\log{P}}
        &\approx{} \bigO{\frac{N^2}{P} + (N+l)\log{P}} \\
        &\approx{} \bigO{\frac{N^2}{P} + (N+l)\log{P}} \\
        &\approx{} \bigO{\frac{N^2}{P} + N\log{P} + \log{P}} \\
        \label{eqn:v0_replicated_pair_operation_overall_time}
        &\approx{} \bigO{\frac{N^2}{P} + N\log{P}}
\end  {align}

\begin{figure}[!h]
    \input{parallel_implementation/v0/replicated.pair_operation.512.logtime.plt}
    \caption{\vZeroTimeCaption{Replicated Data}{\pairoperation{}}{512}}
    \label{fig:v0_replicated_pair_operation_512_logtime}
\end  {figure}

\begin{figure}[!h]
    \input{parallel_implementation/v0/replicated.pair_operation.4096.logtime.plt}
    \caption{\vZeroTimeCaption{Replicated Data}{\pairoperation{}}{4096}}
    \label{fig:v0_replicated_pair_operation_4096_logtime}
\end  {figure}

\begin{figure}[!h]
    \input{parallel_implementation/v0/replicated.pair_operation.32768.logtime.plt}
    \caption{\vZeroTimeCaption{Replicated Data}{\pairoperation{}}{32768}}
    \label{fig:v0_replicated_pair_operation_32768_logtime}
\end  {figure}

\vZeroTimeExplanation
    { \FIG{fig:v0_replicated_pair_operation_512_logtime} }
    { \FIG{fig:v0_replicated_pair_operation_4096_logtime} }
    { \FIG{fig:v0_replicated_pair_operation_32768_logtime} }
    { \pairoperation{} }

In \FIG{fig:v0_replicated_pair_operation_512_logtime},
\FIG{fig:v0_replicated_pair_operation_4096_logtime} and
\FIG{fig:v0_replicated_pair_operation_32768_logtime},
it is apparent that scaling begins to drop off as the number
of processes used is comparable to the number of particles in the system.
%
From \EQN{eqn:v0_replicated_pair_operation_overall_time} when $P \ll{} N$
\begin{equation}
    \bigO{\frac{N^2}{P} + N\log{P}} \sim{} \bigO{\frac{N^2}{P}}
\end  {equation}
suggesting good scaling in this regime.
%
Similarly when $P \sim{} N$
or $P > N$
\begin{align}
    \bigO{\frac{N^2}{P} + N\log{P}}
        &\sim{} \bigO{N + N\log{P}} \\
        &\sim{} \bigO{N\log{P}}
\end  {align}
%
suggesting the communication term eventually dominating and
the overall time increasing as a function of $P$.

Indeed, examining where
\FIG{fig:v0_replicated_pair_operation_512_logtime},
\FIG{fig:v0_replicated_pair_operation_4096_logtime},
\FIG{fig:v0_replicated_pair_operation_32768_logtime} and
begin straying away from linear scaling,
it can be seen that the minimum execution time
of 512 particles is approximately $10^{-4}$ seconds,
of 4096 particles is approximately $10^{-3}$ seconds and
of 32768 particles is approximately $10^{-2}$ seconds.
%
Thus as scaling begins dropping off at $P = N$,
the minimum execution time for this distribution,
for the system sizes tested,
scales as $N\log{N}$
where $N$ is the number of particles in the system.


\subsection{Systolic Loop}

The systolic loop scheme is initialised by allocating 3 arrays of
size $N/P$ on every process.
%
The system is then split roughly evenly across all the processes
and is held, updated and shared using these three lists.


%
% Systolic individual_operation v0
%

\subsubsection{Implementation of the \individualoperation{} Method}
This is implemented by having each process update its local list
of particles.

This should take $\bigO{N/P}$ time.

\begin{figure}[!h]
    \input{parallel_implementation/v0/systolic.individual_operation.512.logtime.plt}
    \caption{\vZeroTimeCaption{Systolic Loop}{\individualoperation{}}{512}}
    \label{fig:v0_systolic_individual_operation_512_logtime}
\end  {figure}

\begin{figure}[!h]
    \input{parallel_implementation/v0/systolic.individual_operation.4096.logtime.plt}
    \caption{\vZeroTimeCaption{Systolic Loop}{\individualoperation{}}{4096}}
    \label{fig:v0_systolic_individual_operation_4096_logtime}
\end  {figure}

\begin{figure}[!h]
    \input{parallel_implementation/v0/systolic.individual_operation.32768.logtime.plt}
    \caption{\vZeroTimeCaption{Systolic Loop}{\individualoperation{}}{32768}}
    \label{fig:v0_systolic_individual_operation_32768_logtime}
\end  {figure}

As seen in
\FIG{fig:v0_systolic_individual_operation_512_logtime},
\FIG{fig:v0_systolic_individual_operation_4096_logtime} and
\FIG{fig:v0_systolic_individual_operation_32768_logtime}
the current implementation satisfies this.

There appear to be a few unexpected data points in the mpi only timings,
but these are unlikely to be anything except an error.
%
As these are approaching nanosecond execution times, it is
unsurprising that they may pick up unexpected errors as the system clock
has at most a nanosecond resolution time.
%
Given the processor operates on about this time, it is also unsurprising
if some odd times may be picked up from taking measurements so
close to this time scale.


%
% Systolic pair_operation v0
%

\subsubsection{Implementation of the \pairoperation{} Method}
This is implemented by having each process use three lists of particles,
each of size $P/N$.

The first list is the processes local list of particles.

The second list will be referred to as the foreign list, and
represents a list originating from another process.

The third list is a swap list to allow a process to receive a new
foreign list list from the right
while also sending its old foreign list to the left
during a systolic pulse.

Initially, a process will copy its local list to the foreign list
and perform a partial force update on its local list using this
foreign list.
%
The system will then perform a systolic pulse.
After a systolic pulse, every foreign list should move one process
to the left in the systolic loop.
%
This is performed by copying the old foreign list into the
swap list, and using an MPI\_sendrecv to send the swap list to
the left process while receiving from
the right process into the foreign list.
%
When a new foreign list is received, another partial force update
is performed on the local list.
%
This process is repeated $P-1$ times.

Each list comparison between systolic pulses should take $\bigO{(N/P)^2}$ time.
%
For a given timestep, there should be $P$ of these list comparisons
performed, giving an over calculation time of $\bigO{N^2/P}$.

Given each pulse should be passing $N/P$ particles between two processes,
we expect this to take $\bigO{N/P + l}$ time where $l$ is a constant latency.
%
With $P$ pulses on each time step, we expect a communication time of
$\bigO{(N/P + l)P}$.

Combining our calculation and communication terms, the systolic loop approach
should run in $\bigO{N^2/P + dN + dlP}$ time
where $N$ is the number of particles in the system,
$P$ is the number of processes used and
$l$ is a constant latency and
$d$ is a constant.

\begin{figure}[!h]
    \input{parallel_implementation/v0/systolic.pair_operation.512.logtime.plt}
    \caption{\vZeroTimeCaption{Systolic Loop}{\pairoperation{}}{512}}
    \label{fig:v0_systolic_pair_operation_512_logtime}
\end  {figure}

\begin{figure}[!h]
    \input{parallel_implementation/v0/systolic.pair_operation.4096.logtime.plt}
    \caption{\vZeroTimeCaption{Systolic Loop}{\pairoperation{}}{4096}}
    \label{fig:v0_systolic_pair_operation_4096_logtime}
\end  {figure}

\begin{figure}[!h]
    \input{parallel_implementation/v0/systolic.pair_operation.32768.logtime.plt}
    \caption{\vZeroTimeCaption{Systolic Loop}{\pairoperation{}}{32768}}
    \label{fig:v0_systolic_pair_operation_32768_logtime}
\end  {figure}

We see in 
\FIG{fig:v0_systolic_pair_operation_512_logtime},
\FIG{fig:v0_systolic_pair_operation_4096_logtime} and
\FIG{fig:v0_systolic_pair_operation_32768_logtime}
that, much like in the replicated case, that the system scales
roughly as $N/P$ when $P \ll{} N$.

With a communication term scaling as $P$, we see the point in which
communications dominate comes much sooner.
%
However, we also note that it appears when $P \approx{} N$.
%
Taking our previous approach of finding an optimum $k$ for $P = N/k$,
we find $k \approx{} 8$.

This appears to hold for our three system sizes, although there does appear
to be an unexpected jump in
\FIG{fig:v0_systolic_pair_operation_32768_logtime}
between 512 and 1024 processes.
%
It is unclear whether this is a genuine effect, or simply an error in
measurement.

Ignoring the unexpected jump in 
\FIG{fig:v0_systolic_pair_operation_32768_logtime},
we may conclude that the minimum time to completion for out system
scales linearly with the number of particles.

It would appear the implimentation ultimately becomes slowed due to
communication latency, and in particular, due to a large number
of these communications.
%
The initial scaling of this term is due to the number of particles
in the system, however,
scaling with $P$ is a result of having $P$ communications
per time step.
%
Looking at our derivation for communication times,
we conclude that this must be an effect of latency.

\section{Shared and Replicated Data}

The \sharedandreplicateddata{} scheme is implemented in exactly the same
way as the \replicateddata{} scheme outlined in 
\SEC{sec:replicated_data_implementation},
except the update loop in the \pairoperation{} method is further
parallelised using \openmp{} directives, taking advantage of shared
memory between cores.
%
In fact, this distribution class inherits directly from the
\replicateddata{} distribution class, and overloads only the
\pairoperation{} method.


The primary motivation for this is to show that mixed mode MPI and \openmp{}
paralellism is just as viable as MPI only parallelism for a \replicateddata{}
scheme along with how easily the mixed mode parallelism may be implemented
on top of an MPI implementation.
%
The importance of this is that the maximum system size per core per node
can be increased in proportion to the number of \openmp{} threads
created per MPI process, which is of particular importance for nodes with
particularly high core counts.

The \sharedandreplicateddata{} distribution is initialised by
allocating a list of particles the size of the system of particles
on every MPI process.
%
The number of \openmp{} threads used per MPI processes is fixed at 8,
as this is the suggested number for \hector{} due to the arrangement
of the \numa{} regions into groups of 8.
%
As this is a rather low number compared to the overall number of MPI
processes, the emphasis here isn't to gain any perticular performance
improvement using \openmp{}, only to show that it is viable.
%
Indeed, the mere introduction of 8 threads per MPI process should increase
the maximum system size that can be run 8 fold.

In this section,
the implementation details and performance of
the \individualoperation{} and \pairoperation{} methods
will be presented and analysed.


\subsection{\individualoperation{}}

The \individualoperation{} method is inherited directly from
the implementation outlined in
\SEC{sec:replicated_data_individual_operation_implementation}.
%
As a result, the time to completion is also expected to scale as $\bigO{N}$.

%
% Overall speedup plot
%
\begin{figure}[!h]
    \input{parallel_implementation/v1/shared_and_replicated.individual_operation.logspeedup.plt}
    \caption{
        Speedup plots for the \individualoperation{} implemented with the \sharedandreplicateddata{} scheme for systems of particles of size 512, 4096 and 32768.
    }
    \label{fig:v1_shared_and_replicated_data_individual_operation_speedups}
\end{figure}


%
% Individual breakdowns
%
\begin{figure}[!h]
    \input{parallel_implementation/v1/shared_and_replicated.individual_operation.512.time.plt}
    \caption{\vZeroTimeCaption{\sharedandreplicateddata{}}{\individualoperation{}}{512}}
    \label{fig:v1_shared_and_replicated_individual_operation_512_time}
\end  {figure}

\begin{figure}[!h]
    \input{parallel_implementation/v1/shared_and_replicated.individual_operation.4096.time.plt}
    \caption{\vZeroTimeCaption{\sharedandreplicateddata{}}{\individualoperation{}}{4096}}
    \label{fig:v1_shared_and_replicated_individual_operation_4096_time}
\end  {figure}

\begin{figure}[!h]
    \input{parallel_implementation/v1/shared_and_replicated.individual_operation.32768.time.plt}
    \caption{\vZeroTimeCaption{\sharedandreplicateddata{}}{\individualoperation{}}{32768}}
    \label{fig:v1_shared_and_replicated_individual_operation_32768_time}
\end  {figure}

\vZeroTimeExplanation
    {\FIG{fig:v0_replicated_individual_operation_512_time}}
    {\FIG{fig:v0_replicated_individual_operation_4096_time}}
    {\FIG{fig:v0_replicated_individual_operation_32768_time}}
    {\individualoperation{}}
    {\replicateddata{}}

As can be seen from 
\FIG{fig:v0_replicated_individual_operation_512_time},
\FIG{fig:v0_replicated_individual_operation_4096_time} and
\FIG{fig:v0_replicated_individual_operation_32768_time}
the performace scaling does indeed scale as $\bigO{N}$, as expected
and exactly in line with the performance scaling results of the
\individualoperation{} method in the \replicateddata{} scheme.



\subsection{\pairoperation{}}

The \pairoperation{} method is implemented in a similar manner to the
\pairoperation{} method from the \replicateddata{} scheme with the
addition of \openmp{} directives to further parallelise the
force update loop.

A copy of the system is held on each MPI process, meaning there
are $P_{MPI}$ replicas of the system created.
%
Each MPI process is then assigned $N/P_{MPI}$ particles in that system
to determine forces for.
%
Given that list of $N/P_{MPI}$ particles,
an MPI process spawns $P_{OMP}$ threads
and assigns $1/(P_{MPI} P_{OMP})$ particles to each thread.

Writing $P = P_{MPI} \times{} P_{OMP}$,
each core therefore has $N/P$ particles
for which it must determine the forces.
%
As before, if each particles must be compared to $N$ other particles
using an $\bigO{1}$ update operation, the time to find the force for
a single particles is
\begin{equation}
    N\bigO{1} = \bigO{N}
\end{equation}
And so, the time to find the forces for $N/P$ particles is
\begin{equation}
    \frac{N}{P}\bigO{N} = \bigO{\frac{N^2}{P}}
\end{equation}

After the forces have been determined by each thread, the team of threads
shuts down.
%
This represents a synchronisation point for the threads within the
MPI process.
%
After this, an MPI\_Allgatherv is used over the $P_{MPI}$ processes
to synchronise the updated lists of particles, representing a
$\bigO{N\log{P_{OMP}}}$ operation.

%
% Overall speedup plot
%
\begin{figure}[!h]
    \input{parallel_implementation/v1/shared_and_replicated.pair_operation.logspeedup.plt}
    \caption{
        Speedup plots for the \pairoperation{} implemented with the \sharedandreplicateddata{} scheme for systems of particles of size 512, 4096 and 32768.
    }
    \label{fig:v1_shared_and_replicated_data_pair_operation_speedups}
\end{figure}


%
% Individual breakdowns
%
\begin{figure}[!h]
    \input{parallel_implementation/v1/shared_and_replicated.pair_operation.512.logtime.plt}
    \caption{\vZeroTimeCaption{\sharedandreplicateddata{}}{\pairoperation{}}{512}}
    \label{fig:v1_shared_and_replicated_pair_operation_512_logtime}
\end  {figure}

\begin{figure}[!h]
    \input{parallel_implementation/v1/shared_and_replicated.pair_operation.4096.logtime.plt}
    \caption{\vZeroTimeCaption{\sharedandreplicateddata{}}{\pairoperation{}}{4096}}
    \label{fig:v1_shared_and_replicated_pair_operation_4096_logtime}
\end  {figure}

\begin{figure}[!h]
    \input{parallel_implementation/v1/shared_and_replicated.pair_operation.32768.logtime.plt}
    \caption{\vZeroTimeCaption{\sharedandreplicateddata{}}{\pairoperation{}}{32768}}
    \label{fig:v1_shared_and_replicated_pair_operation_32768_logtime}
\end  {figure}

\section{Shared and Replicated Data}

\subsection{\individualoperation{}}

%
% Overall speedup plot
%
\begin{figure}[!h]
    \input{parallel_implementation/v1/replicated_systolic.individual_operation.logspeedup.plt}
    \caption{
        Speedup plots for the \individualoperation{} implemented with the \replicatedsystolicloop{} scheme for systems of particles of size 512, 4096 and 32768.
    }
    \label{fig:v1_replicated_systolic_loop_individual_operation_speedups}
\end{figure}


%
% Individual breakdowns
%
\begin{figure}[!h]
    \input{parallel_implementation/v1/replicated_systolic.individual_operation.512.logtime.plt}
    \caption{\vZeroTimeCaption{\replicatedsystolicloop{}}{\individualoperation{}}{512}}
    \label{fig:v1_replicated_systolic_individual_operation_512_time}
\end  {figure}

\begin{figure}[!h]
    \input{parallel_implementation/v1/replicated_systolic.individual_operation.4096.logtime.plt}
    \caption{\vZeroTimeCaption{\replicatedsystolicloop{}}{\individualoperation{}}{4096}}
    \label{fig:v1_replicated_systolic_individual_operation_4096_time}
\end  {figure}

\begin{figure}[!h]
    \input{parallel_implementation/v1/replicated_systolic.individual_operation.32768.logtime.plt}
    \caption{\vZeroTimeCaption{\replicatedsystolicloop{}}{\individualoperation{}}{32768}}
    \label{fig:v1_replicated_systolic_individual_operation_32768_time}
\end  {figure}




\subsection{\pairoperation{}}

%
% Overall speedup plot
%
\begin{figure}[!h]
    \input{parallel_implementation/v1/replicated_systolic.pair_operation.logspeedup.plt}
    \caption{
        Speedup plots for the \pairoperation{} implemented with the \replicatedsystolicloop{} scheme for systems of particles of size 512, 4096 and 32768.
    }
    \label{fig:v1_replicated_systolic_pair_operation_speedups}
\end{figure}


%
% Individual breakdowns
%
\begin{figure}[!h]
    \input{parallel_implementation/v1/replicated_systolic.pair_operation.512.logtime.plt}
    \caption{\vZeroTimeCaption{\replicatedsystolicloop{}}{\pairoperation{}}{512}}
    \label{fig:v1_replicated_systolic_pair_operation_512_logtime}
\end  {figure}

\begin{figure}[!h]
    \input{parallel_implementation/v1/replicated_systolic.pair_operation.4096.logtime.plt}
    \caption{\vZeroTimeCaption{\replicatedsystolicloop{}}{\pairoperation{}}{4096}}
    \label{fig:v1_replicated_systolic_pair_operation_4096_logtime}
\end  {figure}

\begin{figure}[!h]
    \input{parallel_implementation/v1/replicated_systolic.pair_operation.32768.logtime.plt}
    \caption{\vZeroTimeCaption{\replicatedsystolicloop{}}{\pairoperation{}}{32768}}
    \label{fig:v1_replicated_systolic_pair_operation_32768_logtime}
\end  {figure}




\bibliographystyle{unsrt}
\bibliography{bibliography}

\end{document}
