% % % % % % % % % % % % % % % % % % % %
%                                     %
% Towards Exascale Molecular Dynamics %
%                                     %
% % % % % % % % % % % % % % % % % % % %
%          Padraig O Conbhui          %
% % % % % % % % % % % % % % % % % % % %

\documentclass[12pt,a4paper]{report}

\ifx\macrosHeader\undefined

%
% Styles
%
\newcommand{\subsubsubsection}[1]{{\bf #1} \vspace{0.5\baselineskip}}


%
% Copywriting
%
\newcommand{\bigO}[1]{\mathcal{O}{\left({ #1 }\right)}}

\newcommand{\hector}{HECToR}
\newcommand{\velocityverlet}{velocity Verlet}
\newcommand{\verletlist}{Verlet list}
\newcommand{\twobody}{two body}
\newcommand{\LennardJones}{Lennard-Jones}
\newcommand{\numa}{NUMA}

\newcommand{\openmp}{OpenMP}

\newcommand{\replicateddata}{replicated data}
\newcommand{\systolicloop}{systolic loop}
\newcommand{\sharedandreplicateddata}{shared and replicated data}
\newcommand{\replicatedsystolicloop}{replicated systolic loop}

\newcommand{\individualoperation}{
    \texttt{indi}\discretionary{-}{}{}\texttt{vidual\_}\discretionary{-}{}{}\texttt{operation}
}
\newcommand{\pairoperation}{
    \texttt{pair\_}\discretionary{-}{}{}\texttt{operation}
}

\newcommand{\EQN}[1]{Eqn.~(\ref{#1})}
\newcommand{\LST}[1]{Listing~\ref{#1}}
\newcommand{\FIG}[1]{Figure~\ref{#1}}
\newcommand{\SEC}[1]{Section~\ref{#1}}


\newcommand{\vZeroSpeedupCaption}[2]{
    {\bf Speedup vs. Cores} for the {\bf #1} implementation of the
    {\bf #2} method for systems of particles of size
    $N=512$ (red), $N=4096$ (green) and $N=32768$ (blue) and the
    function $f(x) = x$ (light red).
}

\newcommand{\vZeroSpeedupExplanation}[3]{
    #1 shows how the speedup of the #2 implementation of the
    #3 method scales with the number of cores available to the
    implementation.
    %
    This is presented for systems of particles of size
    $N=512$ shown in red, $N=4096$ shown in green
    and $N=32769$ shown in blue along with the function
    $f(x) = x$ shown in light red.
}

\newcommand{\vZeroTimeCaption}[3]{
    {\bf Time vs. Cores} for the {\bf #1} implementation of the
    {\bf #2} method for the total execution time (red),
    calculation execution time without MPI (green) and
    MPI execution time without calculations (blue)
    for an MD system of particles of size ${\bf N = {#3}}$.
}

\newcommand{\vOneSRTimeCaption}[3]{
    \vZeroTimeCaption{#1}{#2}{#3}
    %
    %The number cores presented is the number of MPI
    %processes used multiplied by the number of \openmp{} threads
    %available per MPI process.
}

\newcommand{\vZeroTimeExplanation}[5]{
    {#1}, {#2} and {#3}
    show how the execution time of the {#4} method
    for the {#5} scheme varies with
    the number of cores used for the simulation
    where $N$, the number of particles in the system, is
    512, 4096 and 32768 respectively.
    They present the cases where
    both MPI and calculations are performed (red)
        representing the total execution time;
    calculations are performed without MPI (green)
        representing the execution time of the simulation
        without communication effects; and
    MPI is performed without calculations (blue)
        representing the time taken up solely by the communications.
}

\newcommand{\vOneSRTimeExplanation}[5]{
    \vZeroTimeExplanation{#1}{#2}{#3}{#4}{#5}
    %
    In this graph, the cores listed represent the total possible cores
    $P = P_{MPI} \times{} P_{OMP}$ available to the implementation.
}


\def\macrosHeader{0}
\fi

\ifx\packagesHeader\undefined

\usepackage{amsmath}

\usepackage{hyperref}

\usepackage{graphics}
\usepackage{graphicx}

\usepackage{epcc}

\usepackage[usenames, svgnames]{xcolor}
\usepackage{listings}

\usepackage{enumitem}

\usepackage{tikz}
\usepackage{tikz-uml}
\usetikzlibrary{shapes, arrows}


\def\packagesHeader{0}
\fi

\ifx\configurationHeader\undefined

%
% Set author, title and date
%
\author{\padraigoconbhui{}}
\title{\towardsexascalemd{}}
\date{\today}

\hypersetup{
    pdfauthor={\plainpadraigoconbhui{}},
    pdftitle={\towardsexascalemd{}}
}

%
% Set up hyperref package
%
\hypersetup{
    hidelinks
}



\lstset{frame=single, language=Fortran}
\lstset{
    commentstyle=\color{Blue},
    keywordstyle=\bf \color{Green},
    numbers=left,
    stepnumber=1
}
\lstset{morecomment=[s][\color{LimeGreen}]{PURE}{\ }}
\lstset{morecomment=[s][\color{LimeGreen}]{&}{\ }}


\usepackage{parskip}

\usepackage[font=footnotesize]{caption}



% Styles gratefully stolen from
% http://www.texample.net/tikz/examples/simple-flow-chart/
\tikzstyle{redcolor} = [red!80]
\tikzstyle{bluecolor} = [blue!20]
\tikzstyle{redfill} = [fill=red!80]
\tikzstyle{bluefill} = [fill=blue!20]
\tikzstyle{decision} = [diamond, draw, bluefill, 
    text width=4.5em, text badly centered, node distance=3cm, inner sep=0pt]
\tikzstyle{block} = [rectangle, draw, bluefill, 
    text width=5em, text centered, rounded corners, minimum height=4em]
\tikzstyle{line} = [draw, -latex', rounded corners, thick]


\def\configurationHeader{0}
\fi


\begin{document}


%
% Begin roman numerals
%

\pagenumbering{roman}


%
% Make front page
%

\makeEPCCtitle

\thispagestyle{empty}

\vspace{12cm}

\begin{center}

\large{MSc in High Performance Computing}

\large{The University of Edinburgh}

\large{Year of Presentation: \the\year}

\end{center}

\newpage

%
% Empty page
%
\thispagestyle{empty}
\clearpage\mbox{}\clearpage


%
% Project Abstract
%

\begin{abstract}
Computer experiments are becoming increasingly important in
scientific research.
%
In fields such as material science and biophysics, computer models
of atomic and molecular systems present a rich source of
information.
%
Molecular Dynamics (MD) is one tool for generating these computer models.

There is a constant strive to simulate ever larger molecular systems
using MD.
%
Given a simple formulation of an atomic system suggests a
$\bigO{N^2}$ algorithm, increasing simulation sizes is not
a straightforward task.
%
To this end, many approximations of they physics involved exist that
can dramatically reduce the time complexity of these simulations.
%
However, these approximations represent potential sources of error
in the already chaotic equations of motion for these systems.

This work attempts to determine just how far parallelisation alone
can improve an MD application.
%
In particular, it strives to find where the bottlenecks exist for
four parallel schemes which integrate the equations of motion for
an MD simulation without approximations.
%
Two of these schemes are the familiar \replicateddata{} and
\systolicloop{} schemes.
%
The third is referred to in this work as the \sharedandreplicateddata{}
scheme, representing a hybrid replicated and shared memory scheme.
%
The fourth, presented here, is referred to as the \replicatedsystolicloop{},
which is based on the \systolicloop{} scheme, taking inspiration from
the \replicateddata{} scheme by using replica systolic loops to
reduce the number of pulses any one systolic loop performs.

The bottleneck in the \replicateddata{} scheme was found to be
the amount of data being transferred and the number of MPI processes
being communicated to.
%
This was addressed by the \sharedandreplicateddata{} scheme by
using shared memory to reduce the number of MPI communications
necesary.
%
This scheme also addressed some of the memory usage issues involved
with the \replicateddata{} scheme.

The bottleneck in the \systolicloop{} scheme was found to be the
number of pulses performed, as a result of communication latency.
%
This was addressed by the \replicatedsystolicloop{} scheme by
using replica loops to reduce the number of systolic pulses each
loop performed.

\end{abstract}


% Restart numbering
\pagenumbering{roman}


%
% Content list pages
%

\tableofcontents
\listoftables
\listoffigures


%
% Title page
%

\begin{titlepage}
\vspace*{2in}
% an acknowledgements section is completely optional but if you decide
% not to include it you should still include an empty {titlepage}
% environment as this initialises things like section and page numbering.

%
% Acknowledgements
%

\section*{Acknowledgements}

I would like to thank my supervisor, Dr. Antonia Collis, without whom
I would still be completely incapable of writing a reasonable scientific
document.
%
Her guidance throughout the project, steering me back on track when I
needed steering, and giving me the advice I needed when I needed it,
were instrumental to the existance of this document.

Thanks go out to friends and family for their love and support
over the last 12 months. 

I must also thank my parents, in particular,
for bankrolling me all the way here.



\end{titlepage}


% Start regular page numbering
\pagenumbering{arabic}


%
% Include chapters
%

\chapter{Introduction}

We require that all parallel implementations are functionally equivalent
to the above codes.
%
This allows us to easily implement tests.
%
We simply implement tests for the serial implementation involving
performing transformations on the code and outputting the data to disk
and then analysing this output.

Further, it discourages us from making optimisations that make
assumptions about our MD algorithm.
%
We must focus solely upon parallel list comparison and updates.

Both the replicated data and systolic loop schemes were initially implemented
using MPI for parallelism.



\subsection{Replicated Data}

The replicated data scheme is initialised by allocating a list the
size of the whole system on every process.


\subsubsection{individual\_operation}

Individual operations are performed by having each process update
its entire local list.
This approach involves more computation than having each process
evaluate a subsection of the list and share the result with the
other processes, but it avoids a global synchronization.

This approach should take $\bigO{N}$ time.



\subsubsection{pair\_operation}

We parallelise the outer loop across the processes.
That is, each process is assigned a set of particles for which
it will determine the forces.
So, each process should be comparing $N/P$ particles to $N$ other
particles, suggesting a calculation term of $\bigO{N^2/P}$.

After the loop, we synchronise the updated list across processes.
This was implemented using an MPI\_Allgatherv, which
should introduce a communication term of $\bigO{N\log{P}}$.

Overall, this should take $\bigO{N^2/P + lN\log{P}}$ time
where $N$ is the number of particles,
$P$ is the number of processes and
$l$ is a constant.

\begin{figure}
    \label{fig:v0_replicated_512_logspeedup}
    \input{parallel_implementation/v0/replicated.pair_operation.512.logspeedup.plt}
    \caption{Replicated Data pair\_operation Speedup for $N = 512$}
\end  {figure}

In \FIG{fig:v0_replicated_512_logspeedup}, we see strong scaling results
for the pair\_operation performed normally begin to drop off as the number
of processes used is comparable to the number of particles in the system.
%
This is consistent with our predictions.
%
If $P = N/k$, then our time should be $\bigO{kN + lN\log{kN}}$.
%
We see the communication term begins to dominate our time, and
the system doesn't scale as well.

Similarly, for small $P << N$, we see our calculation term dominating
and the system appears to scale quite well.

This suggests the primary bottleneck in the replicated data approach
is in communications.

We expect to see this approach scale relatively well
with problem size.
%
Where a larger system is used, we may use more cores to solve the problem.
%
In particular, if there is an optimum $k$ for $P = N/k$, we see that
if we place no particular importance on core count, the time to solution
should scale roughly linearly with the problem size.




\subsection{Systolic Loop}

The systolic loop scheme is initialised by allocating 3 arrays of
size $N/P$ on every process.

\begin{description}[style=nextline]
\item[individual\_operation]
    This is implemented by having each process update its local list
    of particles.

    This should take $\bigO{N/P}$ time.

\item[pair\_operation]
    This is implemented by having each process use three lists of particles.

    The first list is the processes local list of particles.

    The second list will be referred to as the foreign list, and
    represents a list originating from another process.

    The third list is a swap list to allow a process to receive a new
    foreign list list from the right
    while also sending its old foreign list to the left
    during a systolic pulse.

    Initially, a process will copy its local list to the foreign list
    and perform a partial force update on its local list using this
    foreign list.
    The system will then perform a systolic pulse.
    After a systolic pulse, every foreign list should move one process
    to the left in the systolic loop.
    This is performed by copying the old foreign list into the
    swap list, and sending the swap list to the left process while
    receiving from the right process into the foreign list.
    When a new foreign list is received, another partial force update
    is performed on the local list.
    This process is repeated $P-1$ times.

    This approach should take $\bigO{N^2/P}$ time.

\end  {description}


\chapter{Background}

%
% Molecular Dynamics
%
\subsection{Molecular Dynamics}


Here, we consider only classical MD simulations.
%
%Q: What exactly is an MD simulation?
A classical MD simulation is performed by
solving the Newtonian equations of motion for
a molecular system.
%
As there are typically more than 3 bodies interacting in the simulation,
it is not possible to solve these equations analytically.
%
The system must be solved numerically.


%
%Q: What bits do we need to make a simulation?
A parallel MD simulation typically consists of three parts:
a particle distribution scheme,
a force evaluation scheme and
a numerical integration scheme.
%
The particle distribution and force finding schemes are
the parts we will be most interested in here.
%
In this study, we use the \velocityverlet{} integration scheme.


%
%Q: What do we need to do with those bits to get data?
An MD simulation typically consists of three phases:
setting up the particle distribution,
simulating the system until it comes to thermodynamic equilibrium and
taking measurements of the system.
%
%Q: Why do we need the system to reach thermodynamic equilibrium?
The system must be allowed to reach thermodynamic equilibrium because
the initial distribution of particles may be a statistically unlikely
configuration.
%
If measurements of the system are taken before this stage, statistically
unlikely ensembles will be sampled more frequently than they should be when
performing measurements.
%
Measurements such as temperature and energy will not be representative
of the system being studied.



%
%Q: What physical constraints does an MD simulation conform to?
An MD system,
assuming homogeneity of time,
homogeneity of space and
isotropy of space
should conserve energy, linear momentum and angular momentum.
%
%Q: Can we use these to tell anything about our simulation?
Measurements of these values can be used to
determine the ``health'' of a simulation.
%
%Q: What does the conformance to these tell us about our simulation?
If any of these values change over the course of a simulation,
it is an indicator that either the system is unphysical or
significant numerical error is creeping into the simulation.
%
%Q: If it necessarily conforms to these, is it still a valid simulation?
However, simply because these symmetries are conserved does not mean
the simulation is producing a correct result.
%
Indeed, a simulation may proceed perfectly, but not accurately reflect
the system it supposes to model.




\subsubsection{Equations Of Motion}

%
%Q: What is a potential?
In a molecular system, each particle will generate a potential field
which permeates throughout the system.
%
A given particle will have a potential energy as a sum of these fields.
%
The gradient of that potential energy then determines
the force felt by that particle.
%
%Q: How do potentials differ for different systems?
The exact form of the potential depends on the system being studied.
%
Systems studied using Coulomb interactions, for example,
will have potentials that drop to zero at relatively long distances
from a particle.
%
Systems studied using Van der Waals interactions will have
potentials that drop to zero at relatively short distances from
a particle.
%
Several inter-molecular and intra-molecular forces may
calculated in a simulation to more accurately model underlying behaviours.


%
%Q: What are the equations of motion for a 2-body potential?
The Hamiltonian for a system of $N$ bodies
interacting under a 2-body potential can be written
\begin{equation}
    H = \sum_{i=1}^N \frac{\vec{p}_i^2}{2 m}
        + \sum_{i=1}^N \sum_{j<i}^N V(\vec{x}_i, \vec{x}_j)
\end  {equation}
where $H$ is the Hamiltonian,
$N$ is the number of bodies,
$m_i$ is the mass of body $i$,
$p_i$ is the momentum of body $i$,
$x_i$ is the position of body $i$ and
$V(\vec{x}_i, \vec{x}_j)$ is the interaction potential between two bodies.
This formula assumes a scalar potential and
that the potential is solely a function of distance.
%
This may be easily extended include terms involving charges,
but is ignored here.
%
We may then derive the equations of motion
\begin{equation}
    m_i \vec{a}_i = -\sum_{\substack{j=1\\j\ne{}i}}^N
                    \vec{\nabla}_i V(\vec{x}_i, \vec{x}_j)
\end  {equation}


%
%Q: What is the Velocity Verlet integration scheme?
The equations of motion may be discretise and solved numerically
by using an appropriate integration scheme.
The integration scheme used in this dissertation is
the \velocityverlet{} scheme
\begin{align}
\label{eqn:velocity_verlet_scheme}
    \vec{v}_i(t + \tfrac{1}{2} h) &=
        \vec{v}_i(t) + \tfrac{1}{2}\vec{a}_i h
    \\
    \vec{x}_i(t + h) &=
        \vec{x}_i(t) + \vec{v}_i(t + \tfrac{1}{2} h) h
    \\
    m_i \vec{a}_i(t + h) &=
        - \sum_{\substack{j=1\\j\ne{}i}}^N
            \vec{\nabla}_i V(\vec{x}_i(t+h), \vec{x}_j(t+h))
    \\
    \vec{v}_i(t+h) &=
        \vec{v}_i(t + \tfrac{1}{2} h) + \tfrac{1}{2} \vec{a}_i(t + h) h
\end  {align}
where $t$ is the time,
$h$ is the step size,
$\vec{v}_i(t)$ is the velocity of body $i$ at time $t$,
$m_i$ is the mass of body $i$,
$\vec{a}_i(t)$ is the acceleration of body $i$ at time $t$,
$\vec{x}_i(t)$ is the position of body $i$ at time $t$ and
$V$ and $N$ are as described above.
%
%Q:What is the error in the VV scheme?
The local error for this scheme is $\bigO{h^4}$ while the local error
is $\bigO{h^3}$.
%
%Q: What is the time to solution for the VV scheme?
As can be seen, for a 2-body potential,
this produces an $\bigO{N^2}$ algorithm.

%
%Q: What time step do we need for the VV scheme?
The timescale at which atomic forces manifest compared to
the timescale at which properties of a molecular system may emerge
requires out numerical integrator use very short steps and be run for
a very large number of them.
%
This in itself introduces a significant amount of challenge to computations.
%
As a classical molecular system can be considered a chaotic one,
we see that any significant error introduced as a result of
increasing time steps may result in
vastly different results in a simulation.
%
Therefore, great care must be taken when choosing time steps, and
indeed a numerical integrator,
that it does not introduce so much error to a simulation as to render it
an unreliable source of information.
%
Some schemes attempt to address this issue by
using different time steps for phenomenon that occur on
different time scales within the same simulation.
%
However, these won’t be discussed here.
%
Indeed, beyond ensuring our numerical integrator and time steps provide
reasonable results for our simulation,
they will be outside the scope of this discussion.


%
%Q: What is the Lennard-Jones potential?
In this dissertation, we will consider the two body Lennard-Jones potential
\begin{equation}
    V_{LJ}(r) = 4\epsilon \left[
        \left( \frac{\delta}{r} \right)^{12}
        - \left( \frac{\delta}{r} \right)^{6}
    \right]
\end  {equation}
where $V_{LJ}(r)$ is the Lennard-Jones potential as
a function of the distance $r$,
$r$ is the distance between two bodies,
$\epsilon$ is the depth of the potential well and
$\delta$ is the distance at which
the potential switches from being attractive to repulsive.
Typically, $\epsilon$ and $\delta$ are determined for a system by
fitting them to experimental data.
%
%Q: What is the LJ potential useful for?
The Lennard-Jones is useful for modelling simple particle interactions.
%
Due to the simplicity of this potential,
it is well studied and there exists good data on
reasonable values for $\epsilon$ and $\delta$ for
different molecular systems.
%
\begin{figure}
    \label{fig:lennard_jones_potential}
    \input{background/lennard_jones_potential.plt}
    \caption{The Lennard-Jones Potential with $\epsilon = 1$ and $\delta = 1$}
\end  {figure}

%
%Q: What are the equations of motion for the LJ potential under VV?
Using the Lennard Jones potential, we can write
\begin{equation}
    V(\vec{x}_i(t), \vec{x}_j(t)) = V_{LJ}(|\vec{x}_i(t) - \vec{x}_j(t)|)
\end  {equation}
and writing $\vec{r}_{ij} = \vec{x}_j - \vec{x}_i$,
$r_{ij} = |\vec{r}_{ij}|$, $\hat{r}_{ij} = \frac{\vec{r}_{ij}}{r_{ij}}$
\begin{equation}
    \vec{\nabla}_i V(r_{ij}) = 4\epsilon \left[
        - \frac{12}{r_{ij}} \left( \frac{\delta}{r_{ij}} \right)^{12}
        + \frac{6}{r_{ij}} \left( \frac{\delta}{r_{ij}} \right)^{6}
    \right]
\end  {equation}
which may be used in \EQN{eqn:velocity_verlet_scheme} to describe
the discretised equations of motion of $N$ bodies interacting under the
Lennard-Jones potential.
%
%Q: How do these equations affect our time to solution?
One may imagine a situation where two bodies are distant enough
that the potential difference between them is negligible compared
to the contributions from closer bodies.
%
In this case, it may be possible to approximate this potential difference
to zero, and in fact, not calculate it.
%
By doing this, we may reduce the time complexity of our simulation.
%
This process is called truncation.


\subsubsection{Truncation of Forces}
%
%Q: What is the importance of the effective length of our potentials?
As can be seen from the form of the Lennard-Jones potential,
after a given distance from a particle, the potential tends towards
zero.
%
This presents the opportunity to consider only interactions between
``close'' particles.
%
This is achieved by setting a cutoff distance and evaluating potentials
between particles within this distance from each other.
%
Several schemes for tailoring a potential to a cutoff exist, and several
schemes for limiting calculations performed to ``close'' particles
exist.
%
The \verletlist{} scheme, for example, involves using some bookkeeping to
maintain lists of particles that are within the cutoff.
%
Typically, \verletlist{} implementations scale near $\bigO{N}$ with the
number of particles in the simulation, but $\bigO{N^2}$ in memory used.
%
Another example is a domain distributed parallel decomposition.
%
Particles that are physically close are placed on processes that are ``close''.
%
In this manner, processes may limit communications to other close processes.
%
Finally, direct solutions are performed between these close particles.

%
%Q: When we truncate our forces, what are the effects?
Overzealous truncation of forces may overly approximate the forces in play.
%
In particular, some molecular systems may be of interest, but the forces
are too long ranged to allow for simulations to be performed in
a reasonable time frame.
% W%here is this ok and where is it not ok?

%
%Q: What happens if we have periodic systems?
Truncation may be used to effectively simulate very large
systems by implementing periodic boundary conditions, but ensuring
that no particle interacts with the same particle twice.
%
With this approach, one hopes to simulate what looks like an infinite system
without introducing periodic effects.
%
Of course, as we have done this by using periodic boundary conditions,
we can not completely remove periodic effects.
%
%Q: When we shrink our system sizes, what are the effects?
These periodic systems must be of a minimum size to avoid introducing
large finite size effects.
%
A system will often, however, experience some finite size effects
regardless of the system size simply because the system is not actually
infinite.

%
%Q: What happens if we have long range potentials?
Of course, we can't reasonably truncate our potentials if they are long ranged.
%
For non-periodic systems, we must directly solve the equations of motion.
%
Further, we must approach the problem from a very different angle if we
are using a periodic system.
%
Schemes, such as Ewald summation, exist to tackle the problem.
%
This will not be discussed here, but it is worth noting the limitations of
long ranged potentials and periodic systems as we are effectively using
truncation to describe short ranged potentials in periodic systems as
long ranged forces in many non-periodic systems.


\subsubsection{Approximations and Sensitivity}

Classical MD simulations do not account for quantum effects,
and as such make some false predictions.
%
For example, A quantum approach would yield
a different distribution of
molecular vibrational frequencies to a classical approach.
%
Typically, a classical simulation will yield incorrect results
where dealing with high speeds or low temperatures.
%
This limits the areas where it may be used for study.

There are several potential issues regarding accuracy in MD simulations.
%
An $N$ body system is a chaotic one.
%
As a result, a small change in accuracy may lead to
largely different results.
%
Errors may be introduced by
the numerical integration scheme used,
the time step used and
approximations of the forces used.
%
While a final answer may not be accurate,
the simulation may still be useful.
%
The system may be used to derive information on
thermodynamic variables or to simulate general trends.

Indeed, here, we seek to improve the time to solution for direct solutions
to allow us to increase the cutoff distance to two-body potentials.
%
By increasing this cutoff distance, we may slightly improve the accuracy
of our simulation.
%
However, given the system is chaotic, we will
still get an ``incorrect'' answer.
%
We hope only to reduce this particular source of error.


\subsection{Parallel Schemes}

%
%Q: Can we parallelize the force calculation?
As discussed, the \velocityverlet{} scheme
immediately provides an algorithm for
performing MD simulations of long ranged forces in isolated systems.
%
We would like to explore parallel implementations of this algorithm.
%
In this dissertation, we begin by focusing on two particular schemes
that directly evaluate the \velocityverlet{} algorithm:
\replicateddata{} and the \systolicloop{}.



\subsubsection{Replicated Data}
\label{sec:background:subsec:replicated_data}

%
%Q: What is the replicated data scheme?
In the \replicateddata{} scheme,
the entire system of particles is replicated across all processors.
%
This is useful for the force update, as each particle must have a view of
every other particle to determine its instantaneous force.
%
The general approach is, in $P$ replicated processes, for each processor
to determine the force for $1/P$ of the particles.
%
Each process may then either update the particles whose forces it has
determined and then share them, or share them and then update the entire
list of particles.
%
In this way, it may provide an $\bigO{N^2/P}$ algorithm.

%
%Q: What are the scaling challenges of replicated data?
Ideally, we would like to simply use $P = N^2$ processes and produce an
$\bigO{1}$ algorithm, however, this is not possible for several reasons.
%
An immediate concern is that each process must have the same view of the
system at all times.
%
As a result, all processes must share their updated particles at
every time step and receive updated particles from other processes.
%
This results in a large amount of communications occurring very frequently,
with a size growing as $\bigO{N}$ and as $\bigO{\log{P}}$,
yielding a communication time of $\bigO{N \log{P}}$.

Further problems arising from this algorithm is that every process has
an individual copy of the system.
%
For large systems running on a large number of cores,
this can stretch the memory of the machine it is running on.
%
Imagining a system of particles where each particle contains double
precision floating point (8~B) data for position, velocity, force and mass,
amounting to 80~B per particle, a 32~GB system could hold around $10^8$
particles.
%
However, if the system has around 10 cores per 32~GB of memory,
with the replicated scheme,
it can now hold a maximum of around $10^7$ particles
as each core will be holding a full system.
%
Essentially, the maximum size of the system becomes inversely proportional
to the number of cores used per GB of memory.
%
This increased use of memory can cause other issues such as poor caching
due to the increased memory use.

We further encounter the problem that all processes must communicate
with every other process at every time step.
%
This puts global synchronisation points into the simulation.
%
Frequent global synchronisation and global data exchange can incur
significant overhead for the calculation.

%
%Q: Why are the scaling challenges of replicated data concerning for exascale?
These drawbacks are particularly important when considering exascale systems
where we expect large numbers of cores per node.
%
As we expect memory per node to grow at a much lower rate, a \replicateddata{}
scheme may have difficulties utilising every core on a node.
%
Indeed, even in current systems, nodes are often requested and only a
subset of cores used due to memory per core considerations in applications.
%
Even if all the cores in the node are utilised, global synchronisation
points do not scale well to large numbers of processes.



\subsubsection{Systolic Loop}
\label{sec:background:subsec:systolic_loop}

%
%Q: What is the systolic loop scheme?
In the \systolicloop{} scheme, the system is split up and distributed among
all the processes.
%
Force updates are performed by passing list data around a ring in pulses.

Initially, a process will make a copy of its list of particles,
and then use that to perform a partial force update on its local list.
%
It will then pass the copied list to the ``right'' process and
receive a new list from the ``left'' process and use that to perform
another partial force update on its local list.
%
This process continues until a process receives its own system again from
the ``left'' process, signifying that it has at some stage received
a copy of every list on every process.
%
At this point, it should have added the force contributions from
every particle in the system to the particles in its local list.
%
From here, a process may update the position and velocity of these local
particles, thereby stepping the system forward.
%
As we are performing an $\bigO{\left(\frac{N}{P}\right)^2}$ operation
when comparing particles and perform this $\bigO{P}$ times, we
get an $\bigO{N^2/P}$ algorithm.


%
%Q: What are the scaling challenges of systolic loop?
As the system is distributed over a number of processes, this allows
us to work on very large systems.
%
Taking the systems and figures from the previous section,
rather than being limited to $10^7$ particles for the system,
we are limited to $10^7$ particles per core.
%
As a result, we have an upper limit on particles per core, but no
upper limit on the actual system being studied.
%
Further, communications between processes is more limited than the
replicated case.
%
Here, we have each process communicating with only two other processes
and sending only $N/P$ particles.
%
By this mechanism, we simultaneously reduce the volume of data
being communicated between every process and
remove a hard global synchronisation
when compared to the \replicateddata{} scheme.
%
However, we must perform $P$ communications.

Ideally, we could set $P = N$ and have a $\bigO{1}$ operation performed
$\bigO{P = N}$ times.
%
The primary limitation here is the overhead of performing the systolic
pulses.
%
As communications have a given latency, we would like to reduce the number
of communications performed.
%
In a \systolicloop{}, the number of operations performed reduces with
the square of the processes, but the communications performed scales linearly.
%
As such, we quickly reach a point where latency dominates over computation.
%
Even for very large $N$, a modest $P$ can incur large latency overheads.
%
%Q: Why are the scaling challenges of systolic loop concerning for exascale?
As increasing the number of processes to a large value
can quickly result in calculations having significant latency,
we see an immediate concern for exascale system.


\chapter{Methodology}

We require that all parallel implementations are functionally equivalent
to the above codes.
%
This allows us to easily implement tests.
%
We simply implement tests for the serial implementation involving
performing transformations on the code and outputting the data to disk
and then analysing this output.

Further, it discourages us from making optimisations that make
assumptions about our MD algorithm.
%
We must focus solely upon parallel list comparison and updates.

Both the replicated data and systolic loop schemes were initially implemented
using MPI for parallelism.



\subsection{Replicated Data}

The replicated data scheme is initialised by allocating a list the
size of the whole system on every process.


\subsubsection{individual\_operation}

Individual operations are performed by having each process update
its entire local list.
This approach involves more computation than having each process
evaluate a subsection of the list and share the result with the
other processes, but it avoids a global synchronization.

This approach should take $\bigO{N}$ time.



\subsubsection{pair\_operation}

We parallelise the outer loop across the processes.
That is, each process is assigned a set of particles for which
it will determine the forces.
So, each process should be comparing $N/P$ particles to $N$ other
particles, suggesting a calculation term of $\bigO{N^2/P}$.

After the loop, we synchronise the updated list across processes.
This was implemented using an MPI\_Allgatherv, which
should introduce a communication term of $\bigO{N\log{P}}$.

Overall, this should take $\bigO{N^2/P + lN\log{P}}$ time
where $N$ is the number of particles,
$P$ is the number of processes and
$l$ is a constant.

\begin{figure}
    \label{fig:v0_replicated_512_logspeedup}
    \input{parallel_implementation/v0/replicated.pair_operation.512.logspeedup.plt}
    \caption{Replicated Data pair\_operation Speedup for $N = 512$}
\end  {figure}

In \FIG{fig:v0_replicated_512_logspeedup}, we see strong scaling results
for the pair\_operation performed normally begin to drop off as the number
of processes used is comparable to the number of particles in the system.
%
This is consistent with our predictions.
%
If $P = N/k$, then our time should be $\bigO{kN + lN\log{kN}}$.
%
We see the communication term begins to dominate our time, and
the system doesn't scale as well.

Similarly, for small $P << N$, we see our calculation term dominating
and the system appears to scale quite well.

This suggests the primary bottleneck in the replicated data approach
is in communications.

We expect to see this approach scale relatively well
with problem size.
%
Where a larger system is used, we may use more cores to solve the problem.
%
In particular, if there is an optimum $k$ for $P = N/k$, we see that
if we place no particular importance on core count, the time to solution
should scale roughly linearly with the problem size.




\subsection{Systolic Loop}

The systolic loop scheme is initialised by allocating 3 arrays of
size $N/P$ on every process.

\begin{description}[style=nextline]
\item[individual\_operation]
    This is implemented by having each process update its local list
    of particles.

    This should take $\bigO{N/P}$ time.

\item[pair\_operation]
    This is implemented by having each process use three lists of particles.

    The first list is the processes local list of particles.

    The second list will be referred to as the foreign list, and
    represents a list originating from another process.

    The third list is a swap list to allow a process to receive a new
    foreign list list from the right
    while also sending its old foreign list to the left
    during a systolic pulse.

    Initially, a process will copy its local list to the foreign list
    and perform a partial force update on its local list using this
    foreign list.
    The system will then perform a systolic pulse.
    After a systolic pulse, every foreign list should move one process
    to the left in the systolic loop.
    This is performed by copying the old foreign list into the
    swap list, and sending the swap list to the left process while
    receiving from the right process into the foreign list.
    When a new foreign list is received, another partial force update
    is performed on the local list.
    This process is repeated $P-1$ times.

    This approach should take $\bigO{N^2/P}$ time.

\end  {description}


\chapter{Parallel Implementations}

We require that all parallel implementations are functionally equivalent
to the above codes.
%
This allows us to easily implement tests.
%
We simply implement tests for the serial implementation involving
performing transformations on the code and outputting the data to disk
and then analysing this output.

Further, it discourages us from making optimisations that make
assumptions about our MD algorithm.
%
We must focus solely upon parallel list comparison and updates.

Both the replicated data and systolic loop schemes were initially implemented
using MPI for parallelism.



\subsection{Replicated Data}

The replicated data scheme is initialised by allocating a list the
size of the whole system on every process.


\subsubsection{individual\_operation}

Individual operations are performed by having each process update
its entire local list.
This approach involves more computation than having each process
evaluate a subsection of the list and share the result with the
other processes, but it avoids a global synchronization.

This approach should take $\bigO{N}$ time.



\subsubsection{pair\_operation}

We parallelise the outer loop across the processes.
That is, each process is assigned a set of particles for which
it will determine the forces.
So, each process should be comparing $N/P$ particles to $N$ other
particles, suggesting a calculation term of $\bigO{N^2/P}$.

After the loop, we synchronise the updated list across processes.
This was implemented using an MPI\_Allgatherv, which
should introduce a communication term of $\bigO{N\log{P}}$.

Overall, this should take $\bigO{N^2/P + lN\log{P}}$ time
where $N$ is the number of particles,
$P$ is the number of processes and
$l$ is a constant.

\begin{figure}
    \label{fig:v0_replicated_512_logspeedup}
    \input{parallel_implementation/v0/replicated.pair_operation.512.logspeedup.plt}
    \caption{Replicated Data pair\_operation Speedup for $N = 512$}
\end  {figure}

In \FIG{fig:v0_replicated_512_logspeedup}, we see strong scaling results
for the pair\_operation performed normally begin to drop off as the number
of processes used is comparable to the number of particles in the system.
%
This is consistent with our predictions.
%
If $P = N/k$, then our time should be $\bigO{kN + lN\log{kN}}$.
%
We see the communication term begins to dominate our time, and
the system doesn't scale as well.

Similarly, for small $P << N$, we see our calculation term dominating
and the system appears to scale quite well.

This suggests the primary bottleneck in the replicated data approach
is in communications.

We expect to see this approach scale relatively well
with problem size.
%
Where a larger system is used, we may use more cores to solve the problem.
%
In particular, if there is an optimum $k$ for $P = N/k$, we see that
if we place no particular importance on core count, the time to solution
should scale roughly linearly with the problem size.




\subsection{Systolic Loop}

The systolic loop scheme is initialised by allocating 3 arrays of
size $N/P$ on every process.

\begin{description}[style=nextline]
\item[individual\_operation]
    This is implemented by having each process update its local list
    of particles.

    This should take $\bigO{N/P}$ time.

\item[pair\_operation]
    This is implemented by having each process use three lists of particles.

    The first list is the processes local list of particles.

    The second list will be referred to as the foreign list, and
    represents a list originating from another process.

    The third list is a swap list to allow a process to receive a new
    foreign list list from the right
    while also sending its old foreign list to the left
    during a systolic pulse.

    Initially, a process will copy its local list to the foreign list
    and perform a partial force update on its local list using this
    foreign list.
    The system will then perform a systolic pulse.
    After a systolic pulse, every foreign list should move one process
    to the left in the systolic loop.
    This is performed by copying the old foreign list into the
    swap list, and sending the swap list to the left process while
    receiving from the right process into the foreign list.
    When a new foreign list is received, another partial force update
    is performed on the local list.
    This process is repeated $P-1$ times.

    This approach should take $\bigO{N^2/P}$ time.

\end  {description}


\chapter{Conclusion}

We require that all parallel implementations are functionally equivalent
to the above codes.
%
This allows us to easily implement tests.
%
We simply implement tests for the serial implementation involving
performing transformations on the code and outputting the data to disk
and then analysing this output.

Further, it discourages us from making optimisations that make
assumptions about our MD algorithm.
%
We must focus solely upon parallel list comparison and updates.

Both the replicated data and systolic loop schemes were initially implemented
using MPI for parallelism.



\subsection{Replicated Data}

The replicated data scheme is initialised by allocating a list the
size of the whole system on every process.


\subsubsection{individual\_operation}

Individual operations are performed by having each process update
its entire local list.
This approach involves more computation than having each process
evaluate a subsection of the list and share the result with the
other processes, but it avoids a global synchronization.

This approach should take $\bigO{N}$ time.



\subsubsection{pair\_operation}

We parallelise the outer loop across the processes.
That is, each process is assigned a set of particles for which
it will determine the forces.
So, each process should be comparing $N/P$ particles to $N$ other
particles, suggesting a calculation term of $\bigO{N^2/P}$.

After the loop, we synchronise the updated list across processes.
This was implemented using an MPI\_Allgatherv, which
should introduce a communication term of $\bigO{N\log{P}}$.

Overall, this should take $\bigO{N^2/P + lN\log{P}}$ time
where $N$ is the number of particles,
$P$ is the number of processes and
$l$ is a constant.

\begin{figure}
    \label{fig:v0_replicated_512_logspeedup}
    \input{parallel_implementation/v0/replicated.pair_operation.512.logspeedup.plt}
    \caption{Replicated Data pair\_operation Speedup for $N = 512$}
\end  {figure}

In \FIG{fig:v0_replicated_512_logspeedup}, we see strong scaling results
for the pair\_operation performed normally begin to drop off as the number
of processes used is comparable to the number of particles in the system.
%
This is consistent with our predictions.
%
If $P = N/k$, then our time should be $\bigO{kN + lN\log{kN}}$.
%
We see the communication term begins to dominate our time, and
the system doesn't scale as well.

Similarly, for small $P << N$, we see our calculation term dominating
and the system appears to scale quite well.

This suggests the primary bottleneck in the replicated data approach
is in communications.

We expect to see this approach scale relatively well
with problem size.
%
Where a larger system is used, we may use more cores to solve the problem.
%
In particular, if there is an optimum $k$ for $P = N/k$, we see that
if we place no particular importance on core count, the time to solution
should scale roughly linearly with the problem size.




\subsection{Systolic Loop}

The systolic loop scheme is initialised by allocating 3 arrays of
size $N/P$ on every process.

\begin{description}[style=nextline]
\item[individual\_operation]
    This is implemented by having each process update its local list
    of particles.

    This should take $\bigO{N/P}$ time.

\item[pair\_operation]
    This is implemented by having each process use three lists of particles.

    The first list is the processes local list of particles.

    The second list will be referred to as the foreign list, and
    represents a list originating from another process.

    The third list is a swap list to allow a process to receive a new
    foreign list list from the right
    while also sending its old foreign list to the left
    during a systolic pulse.

    Initially, a process will copy its local list to the foreign list
    and perform a partial force update on its local list using this
    foreign list.
    The system will then perform a systolic pulse.
    After a systolic pulse, every foreign list should move one process
    to the left in the systolic loop.
    This is performed by copying the old foreign list into the
    swap list, and sending the swap list to the left process while
    receiving from the right process into the foreign list.
    When a new foreign list is received, another partial force update
    is performed on the local list.
    This process is repeated $P-1$ times.

    This approach should take $\bigO{N^2/P}$ time.

\end  {description}



\bibliographystyle{unsrt}
\bibliography{bibliography}

\end{document}
