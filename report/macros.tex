\ifx\macrosHeader\undefined

%
% Styles
%
\newcommand{\subsubsubsection}[1]{{\bf #1} \vspace{0.5\baselineskip}}


%
% Copywriting
%
\newcommand{\bigO}[1]{\mathcal{O}{\left({ #1 }\right)}}

\newcommand{\hector}{HECToR}
\newcommand{\velocityverlet}{velocity Verlet}
\newcommand{\verletlist}{Verlet list}
\newcommand{\twobody}{two body}
\newcommand{\LennardJones}{Lennard-Jones}

\newcommand{\replicateddata}{replicated data}
\newcommand{\systolicloop}{systolic loop}

\newcommand{\individualoperation}{\texttt{individual\_operation}}
\newcommand{\pairoperation}{\texttt{pair\_operation}}

\newcommand{\EQN}[1]{Eqn.~(\ref{#1})}
\newcommand{\LST}[1]{Listing~\ref{#1}}
\newcommand{\FIG}[1]{Figure~\ref{#1}}
\newcommand{\SEC}[1]{Section~\ref{#1}}


\newcommand{\vZeroTimeCaption}[3]{
    {\bf Time vs. Cores} for the {\bf #1} implementation of the
    {\bf #2} method for the total execution time (red),
    calculation execution time without MPI (green) and
    MPI execution time without calculations (blue)
    for an MD system of particles of size ${\bf N = {#3}}$.
}

\newcommand{\vZeroTimeExplanation}[5]{
    {#1}, {#2} and {#3}
    show how the execution time of the {#4} method
    for the {#5} scheme varies with
    the number of cores used for the simulation
    where $N$, the number of particles in the system, is
    512, 4096 and 32768 respectively.
    They present the cases where
    both MPI and calculations are performed (red)
        representing the total execution time;
    calculations are performed without MPI (green)
        representing the execution time of the simulation
        without communication effects; and
    MPI is performed without calculations (blue)
        representing the time taken up solely by the communications.
}


\def\macrosHeader{0}
\fi
