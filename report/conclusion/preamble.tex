%
%Q: What were the objectives and findings of the dissertation?
This dissertation aimed to evaluate the scaling bottlenecks of
some parallel schemes used to perform \twobody{} MD simulations.
%
Two base schemes were chosen:
the \replicateddata{} and \systolicloop{} schemes.
Two improvements upon these schemes were also proposed:
the \sharedandreplicateddata{} and \replicatedsystolicloop{} schemes.

The steps taken to determine the bottlenecks were:
\begin{itemize}
\item
    Proposing an implementation pattern capable of representing
    all four of the parallel schemes to ensure all implementations
    were comparable.

\item
    Implementing the four parallel schemes using the prescribed
    implementation pattern.

\item
    Determining the expected time complexity of the
    \individualoperation{} and \pairoperation{} proposed by
    the pattern, taking
    account of expected calculation and communication patterns.

\item
    Writing a benchmark for the
    \individualoperation{} and \pairoperation{} methods
    to run each for a minimum of 5 iterations and minimum of 1 second
    in 3 regimes:
    \begin{itemize}
        \item Normal operation
        \item MD calculations only
        \item MPI calls only
    \end{itemize}

\item
    Running this benchmark on \hector{} Phase 3, using 32 cores per node
    where possible, on a range from 1 to 32768 cores.

\item
    Matching the graphs found to the expected time complexities.
    Determining where communications become a significant overhead
    to the calculation.

\end{itemize}

It was found for the \replicateddata{} based implementations that
data transfer between MPI processes was the greatest bottleneck.
%
It was found for the \systolicloop{} based implementations that
communication latency between MPI processes was the greatest bottleneck.
