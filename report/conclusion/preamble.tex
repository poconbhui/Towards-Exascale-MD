%
%Q: What were the objectives and findings of the dissertation?
This dissertation aimed to evaluate the scaling bottlenecks of
some parallel schemes used to perform \twobody{} MD simulations.
%
Two base schemes were chosen:
the \replicateddata{} and \systolicloop{} schemes.
Two improvements upon these schemes were also proposed:
the \sharedandreplicateddata{} and \replicatedsystolicloop{} schemes.

The steps taken to determine the bottlenecks were:
\begin{itemize}
\item
    Proposing an implementation pattern capable of representing
    all four of the parallel schemes to ensure all implementations
    were comparable.

\item
    Implementing the four parallel schemes using the prescribed
    implementation pattern.

\item
    Determining the expected time complexity of the
    \individualoperation{} and \pairoperation{} proposed by
    the pattern, taking
    account of expected calculation and communication patterns.

\item
    Writing a benchmark for the
    \individualoperation{} and \pairoperation{} methods
    to run each for a minimum of 5 iterations and minimum of 1 second
    in 3 regimes:
    \begin{itemize}
        \item Normal operation
        \item MD calculations only
        \item MPI calls only
    \end{itemize}

\item
    Running this benchmark on \hector{} Phase 3, using 32 cores per node
    where possible, on a range from 1 to 32768 cores.

\item
    Matching the graphs found to the expected time complexities.
    Determining where communications become a significant overhead
    to the calculation, and whether this was an effect of communication
    times or of latency.
\end{itemize}


\section{Findings}

It was found for the \replicateddata{} and the \sharedandreplicateddata{}
implementations that
data transfer between MPI processes was the greatest bottleneck.
%
It was found for the \systolicloop{} and the \replicatedsystolicloop{}
implementations that
communication latency between MPI processes was the greatest bottleneck.
%
The \replicatedsystolicloop{} also noticed some overhead in the total
calculation time at core counts
nearing $P = 32768$ that could not be explained by the calculation time
and the MPI communications time on their own, suggesting potential
synchronisation effects.

The proposed \sharedandreplicateddata{} scheme was found to perform at
least as well as the \replicateddata{} scheme.
%
It was  shown to require
an amount of memory inversely proportional to the number of
\openmp{} threads per MPI process used.
%
It also displayed a small improvement in performance due to
reduced MPI communications for similar total core counts.

The proposed \replicatedsystolicloop{} scheme was found to directly
improve upon both the \replicateddata{} and \systolicloop{} schemes.
%
It was shown that the memory requirements were comparable to
the \systolicloop{} scheme, with memory per core as $N/\sqrt{P}$,
meaning it was suitable for running data sets that are to large to
fit on any one node.
%
It also displayed a marked improvement in communication times,
approaching $\log{\sqrt{P}}$ at large $P$.


\section{Recommendations}

This dissertation presents a very simple improvement that can be
made to an MPI implementation of the \replicateddata{} scheme to 
immediately reduce the overheads of data storage and MPI
communications.
%
It would be surprising to find any reasonable implementation of the
\replicateddata{} scheme without this optimisation in place.
%
Considering only one unoptimised \openmp{} directive was added to
one loop, it is a very minimal improvement to make.

The dissertation also proposes the \replicatedsystolicloop{}.
%
Given the $\log{\sqrt{P}}$ scaling of this implementation, and the
improvement on memory requirements per node over the \replicateddata{}
schemes, this method improves upon both the \replicateddata{} and
\systolicloop{} schemes.
%
It also passes dramatically smaller messages across the machine
in a much more limited fashion.
%
Systolic pulses are limited to neighbouring processes, and the final
reduction is performed across distinct sets of communicators.
%
It may be possible to take advantage of this topology to gain
improvements in communication times.
%
However, given the relative conceptual complexity over the \replicateddata{}
scheme, implementation is less obvious.

Where a \systolicloop{} implementation exists, much of the code can
be reused, requiring only minor changes.
%
The \replicatedsystolicloop{} adds only two extra
communicators, one of which is used as in the \systolicloop{} case
for performing systolic pulses.
%
It then determines pulse chunks to be performed, and performs
an initial swap.
%
After that, the usual \systolicloop{} code can be used, albeit performing
fewer pulses.
%
The \replicatedsystolicloop{} then includes an \mpiallreduce{} to determine
the final value.


\section{Future Work}

The \replicatedsystolicloop{} presented here may be of perticular
interest for future work.
%
In this dissertation, the placement of processes to take advantage
of communication patterns presented by this scheme has been neglected.
%
Considerations of the nearest neighbour placement of nodes during the
systolic pulse may be of interest.
%
So, too, might optimising for the reduction operation.

An interesting area of study may be optimising the balance of
systolic elements to replica loops.
%
In this dissertation \mpidimscreate{} was used to determine the
number of systolic elements per loop and the number of loops.
%
This function strove to keep these two numbers equal.
%
This resulted in the number of systolic pulses performed remaining
near constant at around 1.
%
However, this lead to the eventually dominating $\log{\sqrt{P}}$
communication term, which was in fact $\log{R}$.

By reducing the number of replicas and increasing the number of
systolic elements, the number of systolic pulses will increase.
%
This will lead to an increase in the $S/R$ communication term.
%
However, before the $S/R$ term dominates, the $\log{R}$ term will
decrease.
%
The decrease in this dominant term may result in an improved
communication time.


\section{Evaluation}

This dissertation presented some scaling results for some implementations
and variations of the \replicateddata{} and \systolicloop{} schemes.
%
The lack of comparisons to any real-world code makes it difficult
to place the overall performance of these codes.
%
This presents a difficulty when attempting to place the timings
made in terms of realistic values.

What it does present, however, is how the implementations assembled here
compare to each other in terms of calculation times and communication times.
%
In particular, it identified that the scaling of a similarly implemented
\systolicloop{} scheme will be worse than that of a \replicateddata{}
scheme at large core counts.
%
As the focus here was not on implementing particularly performant
MD simulations, but rather evaluating the bottlenecks of particular
parallel schemes, any comparison to the times of real-world code
would not be interesting.
%
As a result, the overall time of the code is less interesting than
the trends produced by the timings.

The communication times presented here take only data transfer and
latency times into account.
%
Indeed, in the strong scaling graphs for the \replicatedsystolicloop{}
scheme, it was seen that the total time began to deviate from the
calculation time for large $P$, dispite the fact that the measured
communication times were almost an order of magnitude lower than
either time.
%
It was suggested that this may be due to processes waiting at
synchronisation barriers, namely at the \mpiallreduce{} used
by this scheme.
%
However, it is difficult to make any solid conclusions on this based
on the data presented.

As different simulations may require different functions to be
performed, resulting in calculations that take different amounts
of time, taking synchronisation effects as a result of these calculations
into account in the communication timing may not be
representitive of the general communication complexity.
%
It may therefore be more informative to present the communication
times neglecting synchronisation times, as has been done here,
remarking that situations where the total time is greater than
the sum of the calculation times and the communication times
that it may be due to synchronisation effects.
%
A similar argument can be made for particle sizes, where an implementation
may use a particle containing more information than the ones used here.
%
As MPI will be simply sending $x$ bytes of data representing an
array of particles, the issue of larger particle data types can be addressed,
to a certain extent, by looking at how
communication times vary for different numbers of the particle types used here.
%
Communication times for, say $4096$ of the 80~B particles presented here
may be equivalent to communication times for $512$ 640~B particles.
%
Of course, the scope of this application is rather limited.

This dissertation presented the \replicatedsystolicloop{} which is,
to the best of the author's knowledge, a new approach to the problem.
%
It proposes a data distribution scheme that is suitable for working
with data that exceeds the memory of any one node.
%
It also proposes a communications scheme that improve upon that
of the \replicateddata{} scheme.
%
The data gathered suggests that the communication time at high core
counts is not directly proportional to the number of particles in
the system, and may be mostly latency based.
