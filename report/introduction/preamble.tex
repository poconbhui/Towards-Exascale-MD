% introduction/preamble
%
% Preamble
%

%
%Q: What is a Molecular system?
The world around us contains
a rich assortment of molecular systems in
highly complex arrangements.
%
The way molecules in these systems interact determine various properties
of the system such as heat capacity, structure, rigidity and reactivity.
%
%Q: Why is modelling molecular systems useful?
The ability to study complex molecular systems
is advantageous in areas or research
such as the development of novel drugs
\cite{perryman2004hiv} and
the creation interesting new materials \cite{rao2001science}.

%
%Q: What modelling methods are there?
There are many tools to study systems of interacting particles.
These range from from molecular modelling methods such as
energy minimisation \cite[p.~306]{schlick2010molecular},
to molecular dynamics methods such as 
integrating the classical equations of motion of a
system of particles
\cite[p.~71]{frenkel2001understanding},
or ab initio methods to solve for quantum systems
using density functional theory \cite{parr1995density}.


%
%Q: What is MD?
Molecular dynamics (MD) aims to
gather information about a molecular system by
simulating the time evolution of that system
\cite[p.~63]{frenkel2001understanding}.
%
%Q: Why is MD useful?
By doing this, it is possible to
gather spatial and temporal information on the system.
%
%Q: What is the basic MD problem?
A classical MD simulation is performed by integrating
the classical equations of motion.
%
For the equations of motion of a \twobody{} potential,
this may require $\bigO{N^2}$ operations per time step
\cite[p.~67]{frenkel2001understanding}.
where $N$ is the number of particles in the MD system.


Many approaches exist to reduce the time complexity of
the simulation by altering the potential used.
%
One such approach is to truncate the effective length of inter-molecular
potential and to avoid computing interactions between particles outside of a
potential-dependant cutoff distance.
%
This can be used to produce $\bigO{N^{\frac{3}{2}}}$ algorithms
\cite[p.~545]{frenkel2001understanding}.
%
%Q: Why is truncation bad?
This truncation, however, can have adverse effects on the accuracy
of the simulation \cite{patra2003molecular}.
%
The truncation necessary to achieve an efficient implementation
for a given molecular system may be
too short to accurately model the physics involved.

%
%Q: Why so we want to look at direct solutions?
This dissertation seeks to implement and benchmark two parallel schemes
that can be used to integrate the equations of motion without
approximations.
%
These are the \replicateddata{} and \systolicloop{} schemes
\cite{smith1991molecular}.
%
These schemes present algorithms that would ideally scale as $\bigO{N^2/P}$
where $P$ is the number of cores used to solve the problem.
%
At first glance, it appears that when $P = N$, the algorithm should
scale as $\bigO{N}$.
%
However, this $\bigO{N^2/P}$ scaling tends to fall short at large $P$.
%
This dissertation seeks to identify the strengths and weaknesses of
these schemes, and suggest some potential workarounds.


%
%Q: What will we be doing here?
This dissertation will attempt to:
\begin{itemize}
\item
    Study parallel force evaluation schemes using
    the \twobody{} \LennardJones{} potential with
    non-periodic boundary conditions.

\item
    Study four parallelisation schemes:
    \replicateddata{}, the \systolicloop{},
    \sharedandreplicateddata{} and the \replicatedsystolicloop{}.

\item
    Attempt to evaluate the scalability and
    identify the bottlenecks of these schemes
    by looking at strong scaling performance
    on systems of containing
    $2^{9}$, $2^{12}$ and $2^{15}$ particles.
\end  {itemize}
