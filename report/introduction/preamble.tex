% introduction/preamble
%
% Preamble
%

%
%Q: What is a Molecular system?
The world around us contains
a rich assortment of molecular systems in
highly complex arrangements.
%
The way molecules in these systems interact determine various properties
of the system such as heat capacity, structure, rigidity, reactivity.
%
%Q: Why is modelling molecular systems useful?
With a deeper understanding of these molecular systems,
researchers are able to develop novel drugs,
develop a better understanding of protein folding and
do ever more exciting things in genetic engineering.

%
%Q: What modelling methods are there?
There are many tools to study systems of interacting particles.
These range from from molecular modelling methods such as performing
Monte Carlo over solutions to energy minimisation problems,
to molecular dynamics methods such as using the \velocityverlet{}
algorithm to integrate the classical equations of motion of a
system of particles, or ab initio methods to solve for quantum systems
using density functional theory.
%
Each of these tools is useful for
deriving some information on the system being modelled and
each has their difficulties and limitations.
%
Here, we are interested primarily in molecular dynamics.


%
%Q: What is MD?
Molecular dynamics (MD) aims to
gather information about a molecular system by
simulating the time evolution of that system.
%
%Q: Why is MD useful?
By doing this, it is possible to
gather spatial and temporal information on the system.


%
%Q: What is the basic MD problem?
A classical MD simulation determines
the force on each particle as a result of
every other particle in the system.
%
It will then step the system forward in time by a small amount.
%
The force calculation may involve $\bigO{N^2}$ terms or higher.
%
The time the system may be stepped forward is
quite small compared to the length of time it must be run for.
%
Both the force calculation and the time step represent a unique difficulty.
%
This dissertation focuses on the force calculation.


Many approaches exist to reduce the time complexity of
the force term.
%
One such approach is to truncate the effective length of inter-molecular
forces and to avoid computing interactions between particles outside of a
force-dependant cutoff distance.
This can be used to produce $\bigO{N}$ algorithms.
%
%Q: Why is truncation bad?
This truncation, however, can have adverse effects on the accuracy
of the simulation.
%
The truncation necessary to achieve an efficient implementation
for a given molecular system may be
too short to accurately model the physics involved.

%
%Q: Why so we want to look at direct solutions?
One popular approach which uses force truncation is the
domain distribution scheme.
It will be shown how one such implementation can be cast as
a series of isolated systems
directly solving the equations of motion without truncations.
%
By finding good direct solutions, it may be possible to use reduce the
time taken for these isolated systems to run, and so longer
force truncations may become feasible to use.


%
%Q: What will we be doing here?
This dissertation will attempt to:
\begin{itemize}
\item
    Study parallel force evaluation schemes using
    the \twobody{} \LennardJones{} potential with
    non-periodic boundary conditions.

\item
    Study two main parallelisation schemes:
    \replicateddata{} and the \systolicloop{}.

\item
    Attempt to evaluate the scalability and
    identify the bottlenecks of these schemes
    by looking at strong scaling performance
    on systems of size
    $2^{9}$, $2^{12}$ and $2^{15}$, which correspond
    to systems of approximate size $10^2$, $10^3$ and $10^4$.

\item
    Present a formulation of a parallel domain distributed code using
    force truncations in the form of
    a series of isolated parallel simulations
    without force truncations.
\end  {itemize}
