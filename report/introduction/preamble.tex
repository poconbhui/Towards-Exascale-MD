% introduction/preamble
%
% Preamble
%

%
%Q: What is a Molecular system?
The world around us contains
a rich assortment of molecular systems in
highly complex arramgements.
%
The way molecules in these systems interact determine various properties
of the system such as heat capacity, structure, rigidity, reactivity.
%
%Q: Why is modelling molecular systems useful?
With a deeper understanding of these molecular systems,
they may be used in novel applications such as
the development of new drugs,
the understanding of protein folding and
genetic engineering.
%
%Q: What modelling methods are there?
There are many tools to study these systems,
from molecular modelling by solving energy minimization problems,
to molecular dynamics by integrating equations of motion.
%
Each of these tools is useful for
deriving some information on the system being modelled and
each has their difficulties and limitations.
%
Here, we are interested primarily in molecular dynamics.

%Q: What is MD?
Molecular dynamics (MD) aims to
gather information about a molecular system by
simulating the time evolution of that system.
%
%Q: Why is MD useful?
By doing this, it is possible to
gather spatial and temporal information for the system.
%
Compare this to static modelling which may gather only spatial information.
%
As the system is allowed to run naturally, rather than being solved,
MD also avoids the multiple minimum problems
sometimes present in molecular modelling.
%
However, it presents significant new problems.


%
%Q: What is the basic MD problem?
A classical MD simulation determines
the force on each particle as a result of
every other particle in the system.
%
It will then step the system forward in time by a small amount.
%
The force calculation may involve $\bigO{N^2}$ terms or higher.
%
The time the system may be stepped forward is
quite small compared to the length of time it must be run for.
%
Each of these represent a unique difficulty.
%
Here, we are interested mainly in the force calculation.


There are many approaches taken to reduce the time complexity of
the $\bigO{N^2}$ term.
%
This is often done by truncating the effective length of intermolecular
forces, leading to $\bigO{N}$ algorithms.
%
%Q: Why is truncation bad?
These trade accuracy for improved complexity.
%
Furthermore, the truncation necessary to achieve an efficient implementation
may be shorter than can be justified for a realistic simulation.

%
%Q: Why so we want to look at direct solutions?
With truncation in place, we can liken a simulation to solving a
large number of isolated systems with direct solutions.
%
By finding good direct solutions, it may be possible to use longer force
truncations without worrying about these isolated system sizes.
%
As future HPC architectures are expected to be many core systems,
we would like to see how well the overall time to solution for
$\bigO{N^2}$ algorithms can be scaled.


%
%Q: What will we be doing here?
In this dissertation we will
\begin{itemize}
\item
    Study parallel force evaluation schemes using
    the 2 body Lennard-Jones potential with
    non-periodic boundary conditions.

\item
    Study 2 parallelization schemes:
    replicated data and the systolic loop.

\item
    Attempt to optimize these schemes and
    evaluate their performance by looking at
    strong scaling data on system sizes of
    $10^4$ , $10^5$ , $10^6$ and $10^7$ particles.
    %
    In all cases, the minimum time to solution will be
    considered the most important metric,
    regardless of the number of cores used.
\end  {itemize}
