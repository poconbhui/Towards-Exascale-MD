% introduction/preamble
%
% Preamble
%


The world around us contains
a rich assortment of molecular systems
in highly complex arramgements.
%
With a deeper understanding of these molecular systems,
they may be used in novel applications
such as the development of new drugs,
the understanding of protein folding and
genetic engineering.
%
There are many tools to study these systems,
from molecular modelling by solving energy minimization problems,
to molecular dynamics by integrating equations of motion.
%
Each of these tools is useful for deriving some information on
the system being modelled and each has their difficulties and limitations.
%
Here, we are interested primarily in molecular dynamics.

Molecular dynamics (MD) aims to gather information
about a molecular system by simulating the time evolution of
that system.
%
By doing this, it is possible to to gather
spatial and temporal information for the system.
%
Compare this to static modelling which may gather only spatial information.
%
As the system is allowed to run naturally, rather than being "solved",
MD also avoids the multiple minimum problems
sometimes present in molecular modelling.
%
However, it presents significant new problems.

The timescale at which atomic forces manifest compared to the timescale at
which properties of a molecular system may emerge requires an MD simulation
takes very many very short time steps.
%
This in itself introduces a significant amount of challenge to computations.
%
As a classical molecular system can be considered a chaotic one, we see
that any significant error introduced as a result of increasing time steps
may result in vastly different results in a simulation.
%
Therefore, great care must be taken when choosing time steps, and indeed
a numerical integrator, that it does not introduce so much error to a
simulation as to render it an unreliable source of information.
%
Some schemes attempt to address this issue by using different time steps
for phenomenon that occur on different time scales within the same simulation.
%
However, these won't be discussed here.
%
Indeed, beyond ensuring our numerical integrator and time steps provide
reasonable results for our simulation, they will be outside the scope
of this discussion.

The next area of interest is
the calculation of intermolecular forces that happens on each step.
%
An initial look at the calculation of intermolecular forces suggests
a problem whose solution time scales as $\bigO{N^2}$, where $N$ is the
number of particles in the system.
%
In this dissertation, we will be interested in tackling the calculation
of these forces.
%
Approximations exist providing solutions that scale as $\bigO{N}$.
%
Here, however, we are interested primarily in studying the direct
solutions to these equations of motions, and the study of parallel schemes
which calculate them.

In this dissertation we will
\begin{itemize}
\item
    Study 3 parallelization schemes: replicated data, domain distribution
    and the systolic loop.

\item
    We will attempt to optimize these schemes as much as possible
    and evaluate their performance by looking at strong scaling
    data for each of them on system sizes of $10^4$, $10^5$, $10^6$ and
    $10^7$ particles. In all cases, or most important metric is the minimum
    time to solution, regardless of the number of cores used.

\item
    As the obvious route for optimizing the domain distributed case is to
    truncate short range forces, we will use short range forces in all
    parallelization schemes.

\item
    We will ignore the case of periodic boundary conditions to
    simplify this study.
\end  {itemize}
