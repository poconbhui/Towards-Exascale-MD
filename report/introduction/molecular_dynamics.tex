% introduction/molecular_dynamics.tex
%
% Molecular Dynamics
%
\subsection{Molecular Dynamics}

% General information

Here, we consider only a classical Molecular Dynamics (MD) simulation.
%
A classical MD simulation is performed by
solving the Newtonian equations of motion for that system.
%
As there are typically more than 3 bodies interacting in the simulation,
it is not possible to solve these equations analytically.
%
The system must be solved numerically.
%
The numerical solutions to the equations of motion
typically integrate over a timestep $h$.
%
To simulate the system for a time $t$, one must perform $t/h$ steps.
%
An initial attempt at solving the equations of motion yield a set of equations
requiring $\bigO{N^2}$ steps to solve for each time step.
%
Some approximations may lead to equations that may be solved in $\bigO{N}$
steps, at the cost of reduced accuracy.


Classical MD simulations do not account for quantum effects,
and as such make some false predictions.
%
For example, A quantum approach would yield
a different distribution of molecular vibrational frequencies
to a classical approach.
%
Typically, a classical simulation will yield incorrect results where
dealing with high speeds or low temperatures.
%
This limits the areas where it may be used for study.


% Some caveats and limitations

There are several potential issues regarding
accuracy in MD simulations.
%
An N body system is a chaotic one.
%
As a result, a small changes in accuracy may lead to largely different results.
%
Errors may be introduced by the numerical integration scheme used,
the description of force used and
approximations used in the equations of motion.
%
While a final answer may not be accurate, the simulation may still be useful.
%
The system may be used to derive information on thermodynamic variables
or to simulate general trends.


% Pieces of an MD code

An MD simulation typically consists of three parts:
a particle distribution scheme,
a force finding scheme and
a numerical integration scheme.
%
The particle distribution and force finding schemes are
the parts we will be most interested in here.
%
In particular, the potential particle distribution schemes possible
tend to be limited by the force finding scheme in use.
%
The particle distribution scheme also dictates how the force update scheme
may be parallelized.
%
In this study, we use the velocity Verlet integration scheme.


